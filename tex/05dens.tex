\section{Výpočet hustoty stavov Thoulesovým ansatzom}
V tejto kapitole vypočítame hustotu stavov pomocou Thoulesovho ansatzu. 
Podobne ako v kapitole 2, vychádzame z rovnice pre elektrón interagujúci s disorderom je v rámci modelu {\it želé} popísaný Schrodingerovou rovnicou
\begin{equation}
\label{eq:05schr_dis}
\bigl(-\frac{\hbar^2}{2m}\laplace + V_{dis}(\vr)\bigr)\phi_m^{(0)}(\vr)=\E_m\phi_m^{(0)}(\vr) \text{,}
\end{equation}
kde index $(0)$ na vlnovej funkcii označuje absenciu e-e interakcie.

Do rovnice \eqref{eq:05schr_dis} pridáme e-e interakciu v Hartree-Fockovej aproximácii a Hartreeho príspevok k e-e interakcii zanedbáme, dostaneme Fockove rovnice
\begin{equation}
 \label{eq:05fock_dis}
 \bigl(-\frac{\hbar^2}{2m}\laplace + V_{dis}(\vr) \bigr)\phi_m(\vr)-\sum_{\forall m'} \int d\vrp \phi^*_{m'}(\vr)\phi_{m}(\vrp)V(\vr-\vrp)\phi_{m'}(\vr)=E_m\phi_m(\vr) \text{,}
\end{equation}
kde $\phi_m(\vr)$ a $E_m$ sú vlastné funkcie a vlastné energie elektrónov vo Fockovej aproximácii a $V(\vr-\vrp)$ je potenciálna energia e-e interakcie.

Fockove rovnice \eqref{eq:05fock_dis} budeme riešiť v prvom ráde poruchovej teórie, pretože predpokladáme, že e-e $V(\vr-\vrp)$ interakcia je slabá. Analogicky s kapitolou 2,
rovnice \eqref{eq:05fock_dis} prenásobíme zľava funkciou $\phi^*_m(\vr)$ a integrujeme $\int d\vr$. Problém riešime v prvom ráde poruchovej teórie, $\phi_m(\vr) \simeq \phi_m^{(0)}(\vr)$. Pre prehľadnosť textu funkcie vlnové funkcie $\phi_m^{(0)}(\vr)$ preznačíme  $\phi_m(\vr)$. Odteraz sú $\phi_m(\vr)$ vlastné funkcie neinteragujúceho problému \eqref{eq:05eqschrdis}. Dostávame vzťah pre energiu 
\begin{equation}
 \label{eq:05fock_dis_erg}
 E_m=\E_m-\sum_{\forall m'} \int d\vrp \int d\vr\ \phi^*_{m'}(\vrp) \phi_{m}(\vrp)\phi^*_{m}(\vr)\phi_{m'}(\vr)V(\vr-\vrp) \text{.}
\end{equation}

Potenciál $V(\vr-\vrp)$ je tienená Coulombovská interakcia. Do rovnice \eqref{eq:05fock_dis_erg} je vhodné dosadiť jej Fourierovu transformáciu
\begin{equation}
 \label{eq:05V_ft}
 V(\vr-\vrp)=\ftkvec{(\vr-\vrp)}{\vq}{V(q)}\text{.}
\end{equation}
Dosadením \eqref{eq:05V_ft} do \eqref{eq:05fock_dis_erg} dostaneme rovnicu 
\begin{equation}
\label{eq:05erg_V_ft}
 E_m=\E_m-\sum_{\forall m'}\frac{1}{(2\pi)^{3}} \int d\vq\ V(q) \ |\bra{\phi_m}e^{i\vq\cdot\vr}\ket{\phi_{m'}}|^2 \text{.}
\end{equation}

Rovnica \eqref{eq:05V_ft} popisuje jednu špecifickú vzorku kovu s disorderom.  Vystredujeme ju cez mnoho vzoriek, s makroskopicky rovnakým ale mikroskopicky rôznym disorderom
\begin{equation}
\label{eq:05erg_meandis}
 \overline{E_m}=\overline{\E_m}-\sum_{\forall m'} \frac{1}{(2\pi)^{3}} \int d\vq\ V(q) \overline{|\bra{\phi_m}e^{i\vq\cdot\vr}\ket{\phi_{m'}}|^2} \text{.}
\end{equation}

Výsledok \eqref{eq:05erg_meandis}  je potrebné poopraviť, analogicky s kapitolou 2. Elektrón elektrónová interakcia je v porovnaní s disorderom slabá, ale len v prípade, že medzielektrónové vzdialenosti $|\vr-\vrp|\gtrsim l$, kde $l$ je stredná voľná dráha zrážok elektrónu s disorderom. Pre medzielektrónové vzdialenosti $< l$ interakcia s disorderom sa nestihne uplatniť.

To znamená, že integračný interval musíme rozdeliť na intervale  $(0, q_{max})$ a interval $(q_{max}, \infty)$, kde $q_{max}=A/l$ a $A$ je číslo blízke jednotke.
Na intervale  $(0, q_{max})$ ponecháme \eqref{eq:05erg_V_ft} bez zmeny a na intervale  $(q_{max}, \infty)$ dosadíme namiesto funkcii $\phi_m(\vr)$ funkcie blízke rovinným vlnám  $\phi_m(\vr) \simeq \frac{1}{\sqrt \Omega}e^{i\vk_m\cdot\vr}$. Úpravu neuvádzame, ale uvažujeme ju implicitne pre každý integrál cez $dq$

Doteraz sme postupovali rovnako ako v kapitole 2, ktorá bola robená na základe úvah Altschulera a Aronova  \cite{Altshuler1},\cite{Altshuler3},\cite{Altshuler4}, teda elektrón v okolí Fermiho hladiny bol nahradený semiklasickým vlnovým balíkom, pohybujúcim sa podobne ako Brownowská častica s krokom dĺžky $l=v_F\tau$. 

Teraz však zvolíme iný prístup, ktorý je, narozdiel od Altschulera Aronova platný v oblasti {\it mimo} okolia Fermiho energie. Ešte predtým ale rovnicu upravíme do vhodného tvaru nasledovným postupom.

Vlnové funkcie $\phi_m(\vr)$ rozvinieme do systému rovinných vĺn. 
\begin{align}
\label{eq:05phi}
\phi_m(\vr) = \frac{1}{\sqrt{\Omega}}\sum_{\vk}c_{\vk}^me^{i\vk\vr}\text{,}
\end{align}
kde $\Omega$ je normovací objem Born Von Karmanovej krabice. Dosadením do \eqref{eq:05erg_meandis} dostaneme nasledovný vzťah, kde Diracovú bra-ket notáciu nahradíme integrálmi.
\begin{align}
\notag
\overline{E_m} = \E_m - \frac{1}{(2\pi)^3\Omega^2}&\int d\vq\ V(\vq)\overline{ \sum_{m'} \sum_{\vk_1} \sum_{\vk_3}\int d\vr c^{m*}_{\vk_1}c^{m'}_{\vk_3}e^{i\v\cdot q\vr}e^{i\vk_1\cdot \vr}e^{-i\vk_3\cdot \vr}}\\
&\overline{\sum_{\vk_2}\sum_{\vk_4}\int d\vrp c^{m'*}_{\vk_4}c^{m}_{\vk_2}e^{-i\vq\cdot \vrp}e^{-i\vk_4\cdot \vrp}e^{i\vk_2\cdot \vrp}} \text{.}
\end{align}
Priestor rovinných vĺn je ortonormálny, teda môžme využiť nasledovné vzťahy
\begin{align*}
\int d\vr\ e^{i(\vk_1+\vq-\vk_3)\vr}=\delta(\vk_3-\vk_1-\vq)\\ 
\int d\vrp e^{i(\vk_2-\vq-\vk_4)\vrp}=\delta(\vk_2-\vk_4-\vq)
\end{align*}
Vyžitím týchto vzťahov dostaneme
\begin{align}
\notag
\overline{E_m} = \E_m - \frac{1}{(2\pi)^3\Omega^2}&\int d\vq\ V(\vq)\overline{ \sum_{m'} \sum_{\vk_1} \sum_{\vk_3}\int d\vr c^{m*}_{\vk_1}c^{m'}_{\vk_3}\delta(\vk_3-\vk_1-\vq)}\\
&\overline{\sum_{\vk_2}\sum_{\vk_4}\int d\vrp c^{m'*}_{\vk_4}c^{m}_{\vk_2}\delta(\vk_2-\vk_4-\vq)} \text{.}
\end{align} 
Dve zo štyroch súm môžme nahradiť integrálmi a integrovať využitím Diracových delta funkcii, týmpádom sa týchto súm zbavíme Zvolíme si teda sumy cez $\vk_3$ a $\vk_4$.
\begin{align}
\overline{E_m}=\overline{\E_m} - \frac{1}{(2\pi)^3\Omega}\int d\vq\ V(\vq) \sum_{m'}\sum_{\vk}\sum_{\vkp}\overline{c^{*m}_{\vk+\vq}c^{m}_{\vk\ '+\vq}c^{*m'}_{\vk\ '}c^{m'}_{\vk}}\text{,}
\end{align}
kde sme sumačné indexy nahradili $k_1=k$ a $k_2=k'$. 

Pre ďalšie úpravy rovnice urobíme nasledovné aproximácie, ktoré vyplývajú z toho, že uvažujeme slabý disorder (platí $k_F l \gg 1$ ). Preto považujeme stavy $m$  a $m'$ za nekorelované, z čoho vyplýva že stredná hodnota súčinu je súčin stredných hodnôt. Preto dostávame
\begin{align}
\overline{E_m}=\overline{\E_m} - \frac{1}{(2\pi)^3\Omega}\int d\vq\ V(\vq) \sum_{m'}\sum_{\vk}\sum_{\vkp}\overline{c^{*m}_{\vk+\vq}c^{m}_{\vk\ '+\vq}}\overline{c^{*m'}_{\vk\ '}c^{m'}_{\vk}}\text{,}
\end{align}
Navyše predpokladáme, že stavy $\vk$ a $\vkp$ sú nekorelované, tak isto z dôvodu slabého disorderu, čo vieme matematicky znázorniť Kroneckerovým symbolom $\delta_{\vk\vkp}$. Preto dostávame nasledovnú rovnicu
\begin{align}
\overline{E_m}=\overline{\E_m} - \frac{1}{(2\pi)^3\Omega}\int d\vq\ V(\vq) \sum_{m'}\sum_{\vk}\sum_{\vkp}\overline{c^{*m}_{\vk+\vq}c^{m}_{\vk\ '+\vq}}\overline{c^{*m'}_{\vk\ '}c^{m'}_{\vk}}\delta_{\vk\vkp}\text{,}
\end{align}
Tým pádom sa zbavíme sumy cez $k'$ a dostávame
\begin{align}
\label{eq:05energy2}
\overline{E_m}=\overline{\E_m} - \frac{1}{(2\pi)^3\Omega} \int d\vq\ V(\vq) \sum_{m'}\sum_{\vk} \overline{c^{m*}_{\vk+\vq}c^{m}_{\vk+\vq}}\ \overline{c^{*m'}_{\vk}c^{m'}_{\vk}} \text{.}
\end{align}
Teraz sa dostávame k hlavnej myšlienke tejto kapitoly. Do rovnice \eqref{eq:05energy2} dosadíme nasledovný ansatz
\begin{align}
\label{eq:05thouless}
\overline{c^{m*}_{\vk}c^{m}_{\vk}}=\frac{1}{\pi \rho(\epsilon_m)}\lorenz{\epsilon_m}
\end{align}
Vzťah \eqref{eq:05thouless} je Thoulessov ansatz \cite{Thouless}, ktorý sa používa pri odvodení Kubovej formuly pre optickú vodivosť. Toto odvodenie sme urobili v rámci tejto práce tiež, spolu s fyzykálnym zdôvodnením ansatzu \eqref{eq:05thouless} (viď dodatok A).


Po dosadení Thoulesovho ansatzu \eqref{eq:05thouless} do \eqref{eq:05energy2} dostaneme  

\begin{align}
\label{eq:05energy3}
\overline{E_m}=\overline{\E_m} - \frac{1}{(2\pi)^3\Omega} \int d\vq\ V(\vq) \sum_{m'}\sum_{\vk}\frac{1}{\pi\rho(\epsilon_m)}\frac{\epsilon_\tau}{(\epsilon_m-\epsilon_k)^2+\epsilon_\tau^2}\frac{1}{\pi\rho(\epsilon_{m'})}\frac{\epsilon_\tau}{(\epsilon_{m'}-\epsilon_{|\vk+\vq|})^2+\epsilon_\tau^2} \text{.}
\end{align}
Kde sme zaviedli nasledovné označenia
\begin{align*}
\epsilon_\tau &\equiv \frac{\hbar}{2\tau} \\
\epsilon_{|\vk+\vq|}&\equiv\frac{\hbar^2|\vk+\vq\ |^2}{2m}
\end{align*}
Prejdeme od súm cez $m'$ a $\vk$ k integrálom v energetických súradniciach, čím nám úplne vypadne normovací objem $\Omega$ a do integrálu vstúpia hustoty stavov $\rho(\epsilon_{m'})$ a $\rho(\epsilon_k)$
\begin{align}
\overline{E_m}=\overline{\E_m} - \frac{1}{(2\pi)^3} \int d\vq\ V(\vq)\int \rho(\epsilon_k)d\epsilon_k \int \rho(\epsilon_{m'})d\epsilon_{m'}\frac{1}{\pi\rho(\epsilon_m)}\frac{\epsilon_\tau}{(\epsilon_m-\epsilon_k)^2+\epsilon_\tau^2}\frac{1}{\pi\rho(\epsilon_{m'})}\frac{\epsilon_\tau}{(\epsilon_{m'}-\epsilon_{|\vk+\vq|})^2+\epsilon_\tau^2} \text{.}
\end{align}


Hustota stavov $\rho(\epsilon_m)$ sa vykráti, ale hustoty stavov $\rho(\epsilon_{k})$ a $\rho(\epsilon_m)$ nie. Použitie aproximácie $\rho(\epsilon_m)\approx\rho(\epsilon_k)$ nám umožňuje ich umožňuje vykrátiť
\begin{align}
\label{eq:energy_vykratene}
\overline{E_m}=\overline{\E_m} - \frac{1}{(2\pi)^3} \int d\vq\ V(\vq)\int d\epsilon_k \int d\epsilon_{m'}\frac{1}{\pi}\frac{\epsilon_\tau}{(\epsilon_m-\epsilon_k)^2-\epsilon_\tau^2}\frac{1}{\pi}\frac{\epsilon_\tau}{(\epsilon_{m'}-\epsilon_{|\vk+\vq|})^2-\epsilon_\tau^2} \text{.}
\end{align}

Pri integrovaní cez $d\vq$  prejdeme k sférickým súradniciam

\begin{align}
\label{eq:05energy5}
\notag
\overline{E_m}=\overline{\E_m} - \frac{1}{(2\pi)^3}\frac{1}{\pi^2}\int_0^{2\pi}d\phi \int_0^\pi \sin\theta d\theta\  \int_0^{q_{max}} dq\ q^2 V(q)\\ 
\int_0^{E_F} d\epsilon_{m'} \int_0^{\infty} d \epsilon_k  \frac{\epsilon_\tau}{(\epsilon_m-\epsilon_k)^2+\epsilon_\tau^2}\frac{\epsilon_\tau}{(\epsilon_{k}+\epsilon_{q}+2\sqrt{\epsilon_{k}\epsilon_{q}}\cos(\theta)-\epsilon_{m'})^2+\epsilon_\tau^2} \text{.} 
\end{align}
Tu je vhodné explicitne uviesť integrál na intervale $(0,q_{max})$ ako sme spomenuli na začiatku kapitoly.
Integrál cez $d\phi$ je $2\pi$. Integrály cez $d\epsilon_m$ a $d\theta$ vieme vypočítať analyticky. Zvyšné dva budeme musieť rátať numericky, preto zavedieme bezrozmerné premenné a konštanty:
\begin{align*}
w=\frac{\epsilon_m}{\epsilon_\tau} \\
u=\frac{\epsilon_{m'}}{\epsilon_\tau} \\
x=\frac{\epsilon_k}{\epsilon_\tau} \\
y=\frac{\epsilon_q}{\epsilon_\tau}\\ 
\bar{y}=\frac{q}{k_s}\\
\bar{y}_max=\frac{q_max}{k_s}\\
u_{EF}=\frac{E_F}{\epsilon_\tau}
\end{align*}
V bezrozmerných premenných bude vzťah \eqref{eq:05energy5} 
\begin{align}
\notag
\overline{E_m}=\overline{\E_m} - \frac{1}{(2\pi)^3}\frac{1}{\pi^2}\int_0^{2\pi}d\phi \int_0^\pi \sin\theta d\theta\  \int_0^{\bar{y}_{max}} d\bar{y}\ k_s \frac{\bar{y}^2}{\epsilon_0(\bar{y}^2+1)}\\ 
\int_0^{u_{EF}} du \int_0^{\infty} dx  \frac{1}{(w-x)^2+1}\frac{1}{(x+y+2\sqrt{xy}\cos(\theta)-u)^2+1} \text{.} 
\end{align}
Analytické integrovanie cez $d\theta$ a $du$ je priamočiare a zdĺhavé, preto ho uvádzať nebudeme Po vykonaní všetkých analytických integrálov dostaneme
\begin{align}
\label{eq:05energy6}
\overline{E_m}=\overline{\E_m}-\frac{e^2}{4\pi^4\epsilon_0 k_s^{-1}} \int_0^{\bar y_{max}} d\bar{y}\ \frac{\bar{y}^2}{1+\bar{y}^2}\int_0^{\infty} dx \frac{1}{(w-x)^2+1}F(x,y) \text{,}
\end{align}
kde  $F(x,y)$ je výsledok analytických integrálov
\begin{align*}
F(x,y)=\frac{1}{\sqrt{4xy}}&\{ \\
&(x+y+2\sqrt{xy}-u_{EF})\arctan(x+y+2\sqrt{xy}-u_{EF}) \\
&-(x+y-2\sqrt{xy}-u_{EF})\arctan(x+y-2\sqrt{xy}-u_{EF}) \\
&-(x+y+2\sqrt{xy})\arctan(x+y+2\sqrt{xy}) \\
&+(x+y-2\sqrt{xy})\arctan(x+y-2\sqrt{xy}) \\
&-\frac{1}{2}ln(\frac{(x+y+2\sqrt{xy}-u_{EF})^2+1}{(x+y-2\sqrt{xy}-u_{EF})^2+1}\ \frac{(x+y+2\sqrt{xy})^2+1}{(x+y-2\sqrt{xy})^2+1})\\
&\}\text{.}
\end{align*}


Numerické riešenie \eqref{eq:05selfenergy} zjednodušíme poslednou aproximáciou. Integrál cez $dx$ môžme považovať za $\delta$ funkciu
\begin{align}
\pi\delta(x-w)=\int_0^{\infty}dx\ \frac{1}{(x-w)^2+1}
\end{align}
Finálny vzťah pre energiu, ktorý budeme počítať numericky, bude nasledovný
\begin{align}
\label{eq:05energy7}
\overline{E_m}=\overline{\E_m}-\frac{e^2}{4\pi^3\epsilon_0 k_s^{-1}} \int_0^{\bar y_{max}} d\bar{y}\ \frac{\bar{y}^2}{1+\bar{y}^2}F(w,y) \text{,}
\end{align} 

Teraz odvodíme vzťah pre hustotu stavov, analogicky ku kapitole 2. Rovnicu \eqref{eq:05energy7} si preznačíme
\begin{align}
\overline{E(\E)} = \overline{\E} - E_{self}(\E)
\end{align}
kde Fockova self-energia je v tomto prípade 
\begin{align}
\label{eq:05selfenergy}
E_{self}(\E)=-\frac{e^2}{4\pi^3\epsilon_0 k_s^{-1}} \int_0^{\bar y_{max}} d\bar{y}\ \frac{\bar{y}^2}{1+\bar{y}^2}F(w,y)
\end{align}
Keď už máme vzťah pre self energiu, vieme určiť hustotu stavov:
\begin{align}
\label{eq:05rho_final}
\rho(\E) \simeq \rho_0(\E_F)(1-\frac{dE_{self}(\E)}{d\E})
\end{align}

 

