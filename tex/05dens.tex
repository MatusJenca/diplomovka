\section{Altshuler - Aronovov jav, rozšírenie teórie na energie $|\E - \E_F| \gtrsim \hbar/\tau$}
Napíšme znovu Altshuler - Aronovov výsledok \eqref{eq:aa_dos3 fin},
\begin{equation}
 \label{eq:aa_dos3 fin new}
 \rho(\E)=\rho_0(\E_F) - \frac{q_{max}}{2\pi^3 \hbar D}
 +    \frac{1}{2\pi^2 (2\hbar D)^{3/2}}  \ \sqrt{|\E-\E_F|}  \text{.}
\end{equation}
Z výsledku \eqref{eq:aa_dos3 fin new} vidno, že $\rho(\E) = \rho_0$ pre $|\E-\E_F| =  U_{co}$,
kde
\begin{equation}
\label{eq:aa_U co korelenergia}
 U_{co} = \frac{8}{3\pi^{2}} q_{max}^2 l^2 \frac{\hbar}{\tau} \text{.}
\end{equation}
je tzv. korelačná energia. Keď zvolíme  $q_{max} =  1/l$, tak
$U_{co} = 0.27 \frac{\hbar}{\tau}$, pre $q_{max} =  1.92/l$ dostaneme $U_{co} = \frac{\hbar}{\tau}$. Z kapitoly 2 vieme, že
výsledok  \eqref{eq:aa_dos3 fin new} je teoreticky spoľahlivý pre $|\E-\E_F| \ll  {\hbar}{\tau}$. Experimentálne dáta na obr. \ref{fig:B2}
ukazujú, že výsledok  \eqref{eq:aa_dos3 fin new} platí rozumne pre  $|\E-\E_F| \lesssim  U_{co}$ a samozrejme, očividne neplatí pre $|\E-\E_F| > U_{co}$.

Pre $|\E-\E_F| \gtrsim  U_{co}$ experiment
na obr. \ref{fig:B2} ukazuje, že stavy vytlačené z oblasti $|\E-\E_F| \lesssim  U_{co}$ majú tendenciu sa nakopiť tesne
nad energiou $|\E-\E_F| =  U_{co}$ v oblasti veľkosti dva a až tri krát $U_{co}$.  Konkrétne, pre dáta ukázané v paneli (d) obrázku \ref{fig:B2} bolo ukázané \cite{Mazur}, že
v oblasti veľkosti niekoľko $U_{co}$ nad $U_{co}$ sa nakopilo najmenej $80$ percent stavov. To je argument pre tvrdenie \cite{Mazur}, že hustota stavov má tendenciu zachovávať stavy
podobne ako ich zachováva hustota stavov v okoli supravodivej medzery. Pokiaľ je nám známe,tak tento experimentálny výsledok (najmä fakt, že  pre $|E-E_F| > U_{co}$ hodnota $\rho(\E)$ hodnotu $\rho_0$
najprv prevýši a až potom k nej konverguje zhora) nemá oporu v dostupných teóriach.

V tomto odseku sa pokúsime teóriu rozšíriť do oblasti energií $|E-E_F| \gtrsim U_{co}$. Náš štartovací bod bude rovnica \eqref{eq:erg_meandis}, ktorú kvôli prehľadnosti ešte raz napíšeme:
\begin{equation}
\label{eq:erg_meandis nex}
 \overline{E_m}=\overline{\E_m}-\sum_{\forall m'} \frac{1}{(2\pi)^{3}} \int d\vq\ V(q) \overline{|\bra{\phi_m}e^{i\vq\cdot\vr}\ket{\phi_{m'}}|^2} \text{.}
\end{equation}
kde $V(q) = e^2/\epsilon_0(q^2+k_s^2)$.
Zopakujme, že energie $\E_m$ v  rovnici \eqref{eq:erg_meandis nex} sú energie neporušené e-e interakciou, t.j. energie elektrónov interagujúcich len s disorderom (pre slabý disorder $\overline{\E_m} \simeq \hbar^2 k_m^2/2m$). Tiež zopakujme, že
pre $q > q_{max} \simeq 1/l$ je treba vlnové funkcie $\phi_m(\vr)$ aproximovať rovinnými vlnami $\frac{1}{\sqrt V}e^{i\vk_m\cdot\vr}$, zatiaľčo pre $q < q_{max}$ treba
$\phi_m(\vr)$ považovať za vlnové funkcie elektrónov interagujúcich len s disorderom. Rovnicu \eqref{eq:erg_meandis nex} môžeme teda zapísať aj v tvare
\begin{equation}\label{eq:erg_meandis nex nex}
\begin{split}
 \overline{E_m}= &\frac{\hbar^2 k_m^2}{2m} - \sum_{\forall m'} \frac{1}{(2\pi)^{3}} \int_{|\vq| < q_{max}} d\vq\ V(q) \overline{|\bra{\phi_m}e^{i\vq\cdot\vr}\ket{\phi_{m'}}|^2}  \\
    &  - \sum_{\forall m'} \frac{1}{(2\pi)^{3}} \int_{|\vq| > q_{max}} d\vq\ V(q) |\bra{k_m}e^{i\vq\cdot\vr}\ket{k_{m'}}|^2 \text{.}
\end{split}
\end{equation}
kde $\ket{k_{m}}= \frac{1}{\sqrt V}e^{i\vk_m\cdot\vr}$, a potom upraviť nasledovne:
\begin{equation}\label{eq:erg_meandis nex prepis}
\begin{split}
 \overline{E_m}= &\frac{\hbar^2 k_m^2}{2m} - \sum_{\forall m'} \frac{1}{(2\pi)^{3}} \int d\vq\ V(q) |\bra{k_m}e^{i\vq\cdot\vr}\ket{k_{m'}}|^2  \\
    & -\sum_{\forall m'} \frac{1}{(2\pi)^{3}} \int_{|\vq| < q_{max}} d\vq\ V(q) \overline{|\bra{\phi_m}e^{i\vq\cdot\vr}\ket{\phi_{m'}}|^2} \\
    & + \sum_{\forall m'} \frac{1}{(2\pi)^{3}} \int_{|\vq| < q_{max}} d\vq\ V(q) |\bra{k_m}e^{i\vq\cdot\vr}\ket{k_{m'}}|^2\text{.}
\end{split}
\end{equation}
Poslednú rovnicu môžeme prepísať do tvaru
\begin{equation}\label{eq:erg_meandis nex prepis skratka}
 \overline{E_m} \ \ = \ \ \E_m + E_{self}^{AA}(m) - E_{self}^{free}(m)  \text{,}
\end{equation}
kde
\begin{equation}\label{eq:erg_volna castica s ee}
\E_m = \frac{\hbar^2 k_m^2}{2m} - \sum_{\forall m'} \frac{1}{(2\pi)^{3}} \int d\vq\ V(q) |\bra{k_m}e^{ivq\cdot\vr}\ket{k_{m'}}|^2  \text{}
\end{equation}
je energia voľného elektrónu interagujúceho s ostatnými cez tienenú Fockovu e-e interakciu,
\begin{equation}\label{eq:erg_self AA}
 E_{self}^{AA}(m) \ \ = \ \  -\sum_{\forall m'} \frac{1}{(2\pi)^{3}} \int_{|\vq| < q_{max}} d\vq\ V(q) \overline{|\bra{\phi_m}e^{i\vq\cdot\vr}\ket{\phi_{m'}}|^2} \\
    \text{}
\end{equation}
je self-energia, ktorú Altshuler a Aronov počítali v limite $|\E_m - \E_m'|\ll \hbar/ \tau $ v difúznej aproximácii (výpočet v kapitole 2),
a
\begin{equation}\label{eq:erg_self free}
 E_{self}^{free}(m) \ \ =  \ \  - \sum_{\forall m'} \frac{1}{(2\pi)^{3}} \int_{|\vq| < q_{max}} d\vq\ V(q) |\bra{k_m}e^{i\vq\cdot\vr}\ket{k_{m'}}|^2\text{,}
\end{equation}
je self-energia pochádzajúca z Fockovho príspevku od rovinných vĺn. Z vyššie uvedených rovníc vidno, že  člen $E_{self}^{free}(m)$ vznikne preto, že za neporušená energiu považujeme
energiu $\E_m$ danú vzťahom \eqref{eq:erg_volna castica s ee}
Naozaj, pre kov bez disorderu ($l \rightarrow \infty$, $q_{max} \rightarrow 0$)) sú oba integrály cez premennú $\vq$ v rovniciach \eqref{eq:erg_self AA} a \eqref{eq:erg_self free}
nulové, takže zostane len neporušená energia \eqref{eq:erg_volna castica s ee}.

V kapitole 1 sme rovnicu \eqref{eq:erg_volna castica s ee}
dopočítali až na konečný analytický výsledok \eqref{eq:fock_screen_final}, disperzný zákon voľného elektrónu s Fockovou self enegiou,
z ktorého sme vypočítali aj zodpovedajúcu hustotu stavov (krivka ukázaná na obr. \ref{fig:screening_dos} červenou farbou).
Táto hustota stavov (hustota stavov voľných interagujúcich elektrónov) je neporušená hustota stavov.

Príspevok $E_{self}^{free}(m)$ daný vzťahom \eqref{eq:erg_self free} Altshuler a Aronov neuvažovali, pretože ich zaujímalo len blízke okolie Fermiho energie.
Neskôr uvidíme, že práve tento príspevok spôsobí nakopenie elektrónových stavov nad energiou $U_{co}$, podobné ako ukazujú experimentálne dáta na Obr.  \ref{fig:B2}

Teraz ideme vypočítať člen $E_{self}^{AA}(m)$. Na rozdiel od Altshulera a Aronova (a kapitoly 2) sa o to pokúsime v limite $|\E_m - \E_m'| \gtrsim \hbar/\tau$.
Začneme tým, že vlnové funkcie $\phi_m(\vr)$ rozvinieme do úplneho systému rovinných vĺn:
\begin{align}
\label{eq:05phi}
\phi_m(\vr) = \frac{1}{\sqrt{\Omega}}\sum_{\vk}c_{\vk}^me^{i\vk\vr}\text{,}
\end{align}
kde $\Omega$ je normovací objem. Dosadením do rovnice \eqref{eq:erg_self AA} dostaneme
\begin{multline}
E_{self}^{AA}(m)=  - \frac{1}{(2\pi)^3\Omega^2}     \\
\times  \int_{|\vq| < q_{max}} d\vq\ V(\vq)\overline{ \sum_{m'} \sum_{\vk_1} \sum_{\vk_3}\int d\vr c^{m*}_{\vk_1}c^{m'}_{\vk_3}e^{i\vq\cdot q\vr}e^{i\vk_1\cdot \vr}e^{-i\vk_3\cdot \vr} \sum_{\vk_2}\sum_{\vk_4}\int d\vrp c^{m'*}_{\vk_4}c^{m}_{\vk_2}e^{-i\vq\cdot \vrp}e^{-i\vk_4\cdot \vrp}e^{i\vk_2\cdot \vrp}} \text{.}
\end{multline}
Keď využijeme vzťahy
\begin{equation}
\frac{1}{\Omega}\ \int d\vr\ e^{i(\vk_1+\vq-\vk_3)\vr}=\delta_{\vk_3,\vk_1+\vq}
\  \   \  \text{,}
\ \
\frac{1}{\Omega} \ \int d\vrp e^{i(\vk_2-\vq-\vk_4)\vrp}=\delta_{\vk_2,\vk_4+\vq}
\ \ \  \text{,}
\end{equation}
dostaneme
\begin{equation}
E_{self}^{AA}(m) = - \frac{1}{(2\pi)^3} \int_{|\vq| < q_{max}} d\vq\ V(\vq)\overline{ \sum_{m'} \sum_{\vk_1} \sum_{\vk_3}\int d\vr c^{m*}_{\vk_1}c^{m'}_{\vk_3}\delta_{\vk_3,\vk_1+\vq} \sum_{\vk_2}\sum_{\vk_4}\int d\vrp c^{m'*}_{\vk_4}c^{m}_{\vk_2}\delta_{\vk_2,\vk_4+\vq}} \text{.}
\end{equation}
S pomocou Kroneckerových symbolov vysumujeme cez $\vk_3$ a $\vk_4$ a prichádzame k vzťahu
\begin{align}
E_{self}^{AA}(m)= - \frac{1}{(2\pi)^3}\int_{|\vq| < q_{max}} d\vq\ V(\vq) \sum_{m'}\sum_{\vk}\sum_{\vkp}\overline{c^{*m}_{\vk+\vq}c^{m}_{\vk\ '+\vq}c^{*m'}_{\vk\ '}c^{m'}_{\vk}}\text{,}
\end{align}
kde sumačné indexy $\vk_1$ a $\vk_2$ sú nahradené vzťahmi $\vk_1=\vk$ a $\vk_2=\vk'$.

Začíname robiť aproximácie. Tak ako pri odvádzaní klasickej Drudeho vodivosti v dodatku B, považujeme stavy $m$ a $m'$ za nekorelované. Tým sa posledná rovnica zjednoduší na tvar
\begin{align}
E_{self}^{AA}(m)= - \frac{1}{(2\pi)^3}\int_{|\vq| < q_{max}} d\vq\ V(\vq) \sum_{m'}\sum_{\vk}\sum_{\vkp}\overline{c^{*m}_{\vk+\vq}c^{m}_{\vk\ '+\vq}}\overline{c^{*m'}_{\vk\ '}c^{m'}_{\vk}}\text{,}
\end{align}
Navyše predpokladáme, že aj stavy $\vk$ a $\vkp$ sú nekorelované. To znamená, že $\overline{c^{*m'}_{\vk\ '}c^{m'}_{\vk}} = \overline{c^{*m'}_{\vk}c^{m'}_{\vk}}\delta_{\vk\vkp}$ a
tiež $\overline{c^{*m}_{\vk+\vq}c^{m}_{\vk\ '+\vq}} = \overline{c^{*m}_{\vk+\vq}c^{m}_{\vk +\vq}} \delta_{\vk\vkp}$.
To umožňuje ľahko vysumovať cez $k'$ a dostávame
\begin{align}
\label{eq:05energy2}
E_{self}^{AA}(m)= - \frac{1}{(2\pi)^3} \int_{|\vq| < q_{max}} d\vq\ V(\vq) \sum_{m'}\sum_{\vk} \overline{c^{m*}_{\vk+\vq}c^{m}_{\vk+\vq}}\ \overline{c^{*m'}_{\vk}c^{m'}_{\vk}} \text{.}
\end{align}

Teraz sa dostávame k hlavnej myšlienke tejto kapitoly. V dodatku C je ukázané, že pre objekt $\overline{c^{m*}_{\vk}c^{m}_{\vk}}$ sa dá v self-konzistentnej Bornovej aproximácii  odvodiť vzťah
\begin{align}
\label{eq:05thouless}
\overline{c^{m*}_{\vk}c^{m}_{\vk}}=\frac{1}{\pi \rho(\epsilon_m)}\lorenz{\epsilon_m} \text{.}
\end{align}
Thouless \cite{Thouless} použil aproximáciu \eqref{eq:05thouless} pri opíse vlnových funkcií neusporiadaného elektrónového systému v kvantovej vodivosti Kuba - Greenwooda.
Ukázal, že aproximácia \eqref{eq:05thouless} spôsobí, že kvantová vodivosť Kuba - Greenwooda prejde na klasickú Drudeho vodivosť. Thoulessov postup je podrobne prepočítaný v dodatku B.
Naše odvodenie self-energie  $E_{self}^{AA}(m)$ sa opiera o tie isté aproximácie, ako použil Thouless.


Keď dosadíme aproximáciu \eqref{eq:05thouless} do rovnice \eqref{eq:05energy2}, dostaneme
\begin{align}
\label{eq:05energy3}
E_{self}^{AA}(\epsilon_m)= - \frac{1}{(2\pi)^3} \int_{|\vq| < q_{max}} d\vq\ V(\vq) \sum_{m'}\sum_{\vk}\frac{1}{\pi\rho(\epsilon_m)}\frac{\epsilon_\tau}{(\epsilon_m-\epsilon_k)^2+\epsilon_\tau^2}\frac{1}{\pi\rho(\epsilon_{m'})}\frac{\epsilon_\tau}{(\epsilon_{m'}-\epsilon_{|\vk+\vq|})^2+\epsilon_\tau^2} \text{,}
\end{align}
kde $\epsilon_\tau \equiv \frac{\hbar}{2\tau}$ a $\epsilon_{|\vk+\vq|} \equiv\frac{\hbar^2|\vk+\vq\ |^2}{2m}$.
Prejdeme od sumy cez $m'$ k intergálu $\int_0^{\epsilon_F} d\epsilon_{m'} \rho(\epsilon_{m'})$ a od sumy cez $\vk$ k integrálu  $\int d\epsilon_k \rho(\epsilon_k)$, kde $\rho(\epsilon_{m'})$ a $\rho(\epsilon_k)$
sú hustoty stavov. Dostaneme
\begin{multline}
E_{self}^{AA}(\epsilon_m)= - \frac{1}{(2\pi)^3} \\
\int_{|\vq| < q_{max}} d\vq\ V(\vq)\int d\epsilon_k \rho(\epsilon_k) \int_0^{\epsilon_F} d\epsilon_{m'} \rho(\epsilon_{m'})\frac{1}{\pi\rho(\epsilon_m)}\frac{\epsilon_\tau}{(\epsilon_m-\epsilon_k)^2+
\epsilon_\tau^2}\frac{1}{\pi\rho(\epsilon_{m'})}\frac{\epsilon_\tau}{(\epsilon_{m'}-\epsilon_{|\vk+\vq|})^2+\epsilon_\tau^2} \text{.}
\end{multline}
Hustota stavov $\rho(\epsilon_m')$ sa vykráti, ale hustoty stavov $\rho(\epsilon_{k})$ a $\rho(\epsilon_m)$ nie. Napriek tomu ich kvôli jednoduchosti približne vykrátime a dostávame
\begin{align}
\label{eq:energy_vykratene}
E_{self}^{AA}(\epsilon_m)= - \frac{1}{(2\pi)^3} \int_{|\vq| < q_{max}} d\vq\ V(\vq)\int d\epsilon_k \int_0^{\epsilon_F} d\epsilon_{m'}\frac{1}{\pi}\frac{\epsilon_\tau}{(\epsilon_m-\epsilon_k)^2-\epsilon_\tau^2}\frac{1}{\pi}\frac{\epsilon_\tau}{(\epsilon_{m'}-\epsilon_{|\vk+\vq|})^2-\epsilon_\tau^2} \text{.}
\end{align}
Pri integrovaní cez $d\vq$ prejdeme k sférickým súradniciam:
\begin{align}
\label{eq:05energy5}
\notag
E_{self}^{AA}(\epsilon_m)= - \frac{1}{(2\pi)^3}\frac{1}{\pi^2}\int_0^{2\pi}d\phi \int_0^\pi d\theta \sin\theta \ \int_0^{q_{max}} dq\ q^2 V(q)\\
\int_0^{\E_F} d\epsilon_{m'} \int_0^{\infty} d \epsilon_k \frac{\epsilon_\tau}{(\epsilon_m-\epsilon_k)^2+\epsilon_\tau^2}\frac{\epsilon_\tau}{(\epsilon_{k}+\epsilon_{q}+2\sqrt{\epsilon_{k}\epsilon_{q}}\cos(\theta)-\epsilon_{m'})^2+\epsilon_\tau^2} \text{.}
\end{align}
Integrál cez $d\phi$ je $2\pi$, integrály cez $d\epsilon_m$ a $d\theta$ vieme vypočítať analyticky, a zvyšné dva budeme rátať numericky. Zavedieme bezrozmerné premenné a konštanty:
\begin{equation}
w=\frac{\epsilon_m}{\epsilon_\tau} \ \ \text{,} \ \
u=\frac{\epsilon_{m'}}{\epsilon_\tau} \ \ \text{,} \ \
x=\frac{\epsilon_k}{\epsilon_\tau} \ \ \text{,} \ \
y=\frac{\epsilon_q}{\epsilon_\tau} \ \ \text{,} \ \
\bar{y}=\frac{q}{k_s}  \ \   \text{,} \ \
\bar{y}_max=\frac{q_{max}}{k_s}  \ \ \text{,} \ \
u_{EF}=\frac{\E_F}{\epsilon_\tau}  \ \ \text{.}
\end{equation}
V bezrozmerných premenných nadobudne vzťah \eqref{eq:05energy5} tvar
\begin{align}
\notag
E_{self}^{AA}(\epsilon_m)= - \frac{1}{(2\pi)^3}\frac{1}{\pi^2}\int_0^{2\pi}d\phi \int_0^\pi d\theta \sin\theta \ \int_0^{\bar{y}_{max}} d\bar{y}\ k_s \frac{\bar{y}^2}{\epsilon_0(\bar{y}^2+1)}\\
\int_0^{u_{EF}} du \int_0^{\infty} dx \frac{1}{(w-x)^2+1}\frac{1}{(x+y+2\sqrt{xy}\cos(\theta)-u)^2+1} \text{.}
\end{align}
Analytické integrovanie cez $d\theta$ a $du$ je zdĺhavé, takže uvedieme len výsledok Po vykonaní analytických integrálov dostaneme
\begin{align}
\label{eq:05energy6}
E_{self}^{AA}(\epsilon_m)= -\frac{e^2}{4\pi^4\epsilon_0 k_s^{-1}} \int_0^{\bar y_{max}} d\bar{y}\ \frac{\bar{y}^2}{1+\bar{y}^2}\int_0^{\infty} dx \frac{1}{(w-x)^2+1}F(x,y) \text{,}
\end{align}
kde
\begin{align*}
F(x,y)=\frac{1}{\sqrt{4xy}}& \{(x+y+2\sqrt{xy}-u_{EF})\arctan(x+y+2\sqrt{xy}-u_{EF}) \\
&-(x+y-2\sqrt{xy}-u_{EF})\arctan(x+y-2\sqrt{xy}-u_{EF}) \\
&-(x+y+2\sqrt{xy})\arctan(x+y+2\sqrt{xy}) \\
&+(x+y-2\sqrt{xy})\arctan(x+y-2\sqrt{xy}) \\
&-\frac{1}{2}ln(\frac{(x+y+2\sqrt{xy}-u_{EF})^2+1}{(x+y-2\sqrt{xy}-u_{EF})^2+1}\ \frac{(x+y+2\sqrt{xy})^2+1}{(x+y-2\sqrt{xy})^2+1})\}\text{.}
\end{align*}

