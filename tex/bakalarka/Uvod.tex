


Altshuler a Aronov (AA) ukázali teoreticky \cite{Altshuler1},\cite{Altshuler3},\cite{Altshuler4}, že elektrón-elektrónová (e-e) interakcia v slabo neusporiadanom trojrozmernom (3D) kove potláča jednoelektrónovú hustotu
 stavov v okolí Fermiho energie ($E_F$). Stručne, v teórii \cite{Altshuler1},\cite{Altshuler3},\cite{Altshuler4} ide o kvantový výpočet hustoty
stavov v elektrónovom plyne, v ktorom elektróny interagujú cez tienenú Coulombovskú interakciu a zároveň sú vystavené pôsobeniu slabého
náhodného potenciálu pochádzajúceho od disorderu. Konkrétnejšie, e-e interakcia je započítana v prvom ráde poruchovej teórie vo Fockovej aproximácii a
vplyv disorderu na vlnové funkcie elektrónov je zahrnutý v difúznej semiklasickej aproximácii.

 AA teória \cite{Altshuler1},\cite{Altshuler3},\cite{Altshuler4} predpovedá, že hustota stavov závisí od energie elektrónu ($E$) ako $\sqrt{|E-E_F|}$, kde $|E-E_F| \lesssim U_{co}$ a $U_{co}$ je charakteristická korelačná energia. (Ako uvidíme neskôr, v typických kovoch s Fermiho energiou rádove jednotky elektrónvoltov je $U_{co} \simeq \hbar/\tau$, kde $\hbar$ je Planckova konštanta a $1/\tau$ je frekvencia elastických elektrónových zrážok s disorderom.) Hustota stavov $\propto \sqrt{|E-E_F|}$ pre energie $|E-E_F| \lesssim U_{co}$ bola pozorovaná metódami tunelovej spektroskopie \cite{Abeles},\cite{Dynes},\cite{McMillan2}
\cite{ImryOvadyahu}, \cite{Schmitz1}, \cite{Schmitz2}, \cite{Escudero}, \cite{Teizer}, \cite{Mazur},\cite{Luna2014}, \cite{Luna2015} a fotoelektrónovej spektroskopie \cite{Kobayashi}.


%V niektorých experimentálnych prácach \cite{Schmitz1}, \cite{Schmitz2}, \cite{Escudero},\cite{Mazur}, \cite{Kobayashi} bola pozorovaná interakciou a disorderom modifikovaná hustota stavov aj pre energie $|E-E_F| > U_{co}$, teda %aj pre energie kde AA teória už neplatí. Cieľom práce \cite{Mazur} bolo ukázať experimentálne, že hustota stavov v prítomnosti AA javu vykazuje pre $|E-E_F| > U_{co}$ tzv. stavy zachovávajúcu závislosť od energie, podobnú tej %ktorá sa pozoruje po oboch stranách
%energetickej medzery v supravodiči. Experiment \cite{Mazur} naozaj ukázal,
%že stavy vypudené AA efektom z oblasti $|E-E_F| \lesssim U_{co}$ majú tendenciu sa nakopiť hneď nad energiou $U_{co}$ v oblasti veľkosti dva a až tri krát $U_{co}$.

%Pozorovaná \cite{Mazur} stavy zachovávajúca hustota stavov však nebola porovnaná s teóriou, pretože známe teórie  \cite{Altshuler1}, \cite{Altshuler3}, \cite{Altshuler4}, \cite{LeeRamakrishnan}, \cite{Imry} sú obmedzené
%na AA efekt pri energiách $|E-E_F| \lesssim U_{co}$.
%Nedávno bolo ukázané \cite{Moskova}, že stavy zachovávajúca hustota stavov pozorovaná v práci \cite{Mazur} bola prítomná (nepovšimnutá) aj v starších experimentoch \cite{Schmitz1}, \cite{Schmitz2}, \cite{Escudero}.
%Bola odvodená aj stavy zachovávajúca hustota stavov \cite{Moskova}, avšak odvodenie bolo len heuristické, bez základov v mikrosopickej teórii.

Cieľom tejto bakalárskej práce je
odvodiť hustotu stavov Altshulera=Aronova pre $|E-E_F| \lesssim U_{co}$ a zreprodukovať ich výsledok. Na rozdiel od Altshulera a Aronova ktorí použili metódu Greenových funkcií a Feynmannových diagramov,
v tejto práci budeme prezentovať technicky iné odvodenie, ktoré sa autor tejto práce naučil z prednášok svojej konzultantky a vedúceho práce \cite{MoskovaUnpub1}. Ide o odvodenie založené na základnej kvantovej mechanike, ktoré je zvládnuteľné na bakalárskej úrovni a pritom obsahuje tú istú fyziku (tienená e-e interakcia vo Fockovej aproximácii, vplyv disorderu v difúznej aproximácii) ako AA teória.

%Druhým cieľom našej práce bude opísať istú aproximáciu \cite{MoskovaUnpub2}, vďaka ktorej sa
%mikroskopická teória stáva aspoň hrubo použiteľnou aj pre energie $|E-E_F| > U_{co}$. Úspechom bude, keď takto rozšírená mikroskopická teória kvalitatívne napodobní stavy zachovávajúcu hustotu stavov, pozorovanú vo vyššie %spomenutých experimentoch.

Text práce je zostavený nasledovne. Kapitola 1 je venovaná najjednoduchšiemu modelu vodivostných elektrónov v kove, volnému neinteragujúcemu elektrónovému plynu.
Kapitola 2 obsahuje úvod do Hartree-Fockovej aproximácie a odvodenie Fockovej aproximácie pre disperzný zákon elektrónu vo voľnom plyne (model želé) elektrónov, ktoré interagujú
cez holý Coulombov potenciál. Ukážeme numerické výsledky pre spektrum energie aj hustotu stavov. V kapitole 3 vezmeme do úvahy tienenie Coulombovho potenciálu a opäť vypočítame vo Fockovej aproximácii disperzný zákon
  a hustotu stavov. Výsledky porovnáme s výsledkami pre holý Coulombov potenciál. Kľúčové časti práce sú kapitoly 4 a 5.
V kapitole 4 opíšeme mikroskopické odvodenie pôvodného výsledku Altshulera-Aronova. V kapitole 5 opíšeme fyzikálny princíp metódy tunelovej spektroskopie, pomocou ktorej bol AA efekt v minulosti pozorovaný. Uvedieme niekoľko
experimentálnych výsledkov a spomenieme niektoré otvorené problémy, ktoré by sme chceli skúmať v budúcnosti v rámci diplomovej práce.  Na konci textu sme pridali dodatok, ktorý obsahuje technické detaily. 