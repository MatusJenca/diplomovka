\renewcommand{\thesection}{B.\arabic{section}}
\renewcommand{\theequation}{B.\arabic{equation}}
\renewcommand{\thefigure}{B.\arabic{figure}}
\setcounter{equation}{0}
\subsection*{Poissonova rovnica pre tienený potenciál}
Nech náš vložený náboj má hustotu $\rho_{ext}$. Pre jeho potenciál platí:
\begin{equation}
 \label{eq:poisext}
 \laplace \Phi_{ext}(\vr)=-\frac{\rho_{ext}(\vr)}{\epsilon_0} \text{.}
\end{equation} 

Pre celkový potenciál, aj po započítaní vloženého náboja bude 
\begin{equation}
 \label{eq:poistot}
 \laplace \Phi_{tot}(\vr)=-\frac{\rho_{tot}(\vr)}{\epsilon_0} \text{.}
\end{equation}
Hustota vloženého náboja bude $\rho_{ext}(\vr)=-e\delta(\vr)$. Hustota indukovaného náboja je daná hustotou elektrónov každý má náboj $e$, teda $\rho_{ind}(\vr)=-en_{ind}(\vr)$,
kde,

\begin{equation}
 \label{eq:n_ind}
 n_{ind}(\vr)=n(r)-n_0 \text{,}
\end{equation} 

kde $n_0$ je pôvodná hustota elektrónov a $n(\vr)$ je hustota vzniknutých kladných častíc 
\begin{equation}
 \label{eq:fermidirac0}
 n_0=2\frac{1}{(2\pi)^3}\int d\vk \frac{1}{e^{(\frac{E(\vk)-\mu}{k_BT})}+1} \text{,}
\end{equation} 
 z Ferami-Diracovho rozdelenia - v prípade elektrónov v kove môžme za chemický  potenciál $\mu$ dosadiť Fermiho Energiu $E_F$.
 V prípade $n(\vr)$ za $\mu$ dosadíme $E_F-e\Phi_{tot}(\vr)$, čo bude nová Fermiho energia.
 
 Veličinu $n_0$ poznáme, je to pôvodná hustota elektrónov v kove. Veličinu $n(\vr)$ vypočítame pri nulovej teplote, kde nám Fermi-Diracovo rozdelenie
 prejde na $\Theta$-funkciu:
 
 \begin{equation}
  \label{eq:limLT}
  \lim_{T\to 0} \frac{1}{e^{(\frac{E(\vk)-\mu}{k_BT})}+1}=\Theta(E(\vk)-\mu) \text{.}
 \end{equation} 
 Po prejdení do súradníc energie, a vykonaní limity \eqref{eq:limLT} nám vzťah \eqref{eq:fermidirac0} prejde na 
 \begin{equation}
  \label{eq:nr}
  n(\vr)=\int_{0}^{E_F-e\Phi_{tot}{\vr}}dE \rho(E)=\int_{0}^{E_F}dE \rho(E) + \int_{E_F}^{E_F-e\Phi_{tot}{\vr}}dE \rho(E) \text{,}
 \end{equation} 
 kde $\rho(E)$ je hustota stavov z \eqref{eq:rho_par}. Integrál možno rozložiť na súčet, kde prvý člen je rovný $n_0$. 
 Keďže $e\Phi_{tot}{\vr}$ je malé, môžme druhý integrál napísať ako $\rho(E_F)(-e\Phi_{tot}(\vr))$. Po dosadení do \eqref{eq:n_ind} dostaneme  
 \begin{equation}
  \label{eq:n_ind_final}
n_{ind}(\vr)=\rho(E_F)(-e\Phi_{tot}(\vr))\text{.}
 \end{equation} 
  
 Toto vieme dosadiť do Poisonvej rovnice \eqref{eq:poistot},
 \begin{equation}
  \label{eq:poistot_final}
   \laplace \Phi_{tot}(\vr)=-\frac{\rho(E_F)(-e^2\Phi_{tot}(\vr))-e\delta(\vr)}{\epsilon_0} \text{.}
 \end{equation} 
 %\subsection{Riešenie Poisonvej rovnice pre tienený potenciál pomocou Fourierovej transformácie}
 
 Rovnicu \eqref{eq:poistot_final} budeme riešiť pomocou Fourierovej transformácie. Obe strany transformujeme, a na ľavej strane zameníme laplace za integrál.
 \begin{align*}
  \laplace \ftk{\vr}{\vq}{\Phi_{tot}(\vq)}&=-\frac{\ftk{\vr}{\vq}{\rho(E_F) e^2 \Phi_{tot}(\vr)}-e\ftk{\vr}{\vq}{}}{\epsilon_0} \\
    \ftk{\vr}{\vq}{q^2\Phi_{tot}(\vq)}&=\frac{e}{\epsilon_0}{\ftk{\vr}{\vq}{(e\rho(E_F)\Phi_{tot}(\vr)-1)}}\text{.}
 \end{align*} 
 Obe strany integrujeme cez celý q-priestor, teda hodnoty oboch pod integrálnych funkcií sa musia rovnať. Po vykrátení a jednoduchých algebraických operáciach viem vyjadriť
 výsledný pretransformovaný potenciál:
 
 \begin{equation}
  \label{eq:screenpot_FT}
  \Phi_{tot}(\vq)=\frac{-e}{\epsilon_0(q^2+k_s^2)}\text{,}
 \end{equation} 
kde $k_s^2=\frac{e^2 \rho(Ef)}{\epsilon_0}$ je tienenie, ktoré v typických prípadoch má hodnotu $k_s\doteq k_F$.

Tu by sme mohli náš výpočet ukončiť, lebo pri výpočte cez Fockove rovnice \eqref{eq:fock_final} robíme tiež Fourierovu transformáciu,
ale pre úplnosť riešenia môžme spätne transformovať, a dostať tak skutočný tienený potenciál. 
Prejdením do sférických súradníc a následnou substitúciou $\cos{\theta}=z$ s dostaneme:
 \begin{align*}
 \Phi_{tot}(\vr)=\ftk{\vr}{\vq}{\frac{e}{\epsilon_0(q^2+k_s^2)}} &=\\ 
 \frac{-e}{(2\pi)^3\epsilon_0} \int_0^{2\pi}d\phi \int_0^{\pi}d\theta \sin(\theta) \int_0^{\infty} dq\ q^2 e^{iqr} \frac{1}{(q^2+k_s^2)} &= \\ \notag
 \frac{-2\pi e }{(2\pi)^3\epsilon_0} \int_0^{\pi}d\theta \sin(\theta) \int_0^{\infty} dq\ q^2 e^{iqr\cos{\theta}} \frac{1}{(q^2+k_s^2)} &= \\ \notag
\frac{-e}{(2\pi)^2\epsilon_0} \int_0^{\infty} dq\ \frac{q^2}{q^2+k_s^2} \frac{e^{iqr}-e^{-iqr}}{iqr} \text{.}
 \end{align*} 
 Posledný integrál vypočítame prechodom do komplexnej roviny cez reziduovú vetu. Dostaneme vzorec pre exponenciálne zanikajúci Yukkavov potenciál.
 \begin{equation}
  \label{eq:yukav_pot}
  \phi_{tot}(\vr)=\frac{-e}{4\pi\epsilon_0 |\vr| }e^{-k_s r} \text{.}
 \end{equation} 
