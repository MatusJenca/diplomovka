\thispagestyle{empty}
\section*{Abstrakt}

Jenča, Matúš: Vplyv elektrón-elektrónovej interakcie a disorderu na hustotu elektrónových stavov v kove: Teoretické skúmanie [Bakalárska práca], Univerzita Komenského v Bratislave, Fakulta matematiky, fyziky a informatiky, Katedra experimentálnej fyziky; školiteľ: Doc. RNDr. Martin Moško, DrSc., Bratislava, 2019,  38 s.

Elektrón-elektrónová interakcia v kombinácii s elektrónovými zrážkami s disorderom (s náhodnou priestorovou distribúciou prímesí) spôsobuje
v kovoch potlačenie elektrónovej hustoty stavov v blízkom okolí Fermiho
energie. Toto lokálne zmenšenie hustoty stavov v okolí Fermiho energie (pod okolím sa rozumie energetický interval daný ako frekvencia zrážok s disorderom násobená Planckovou konštantou)
prvý krát teoreticky predpovedali Altshuler a Aronov. Jav, nazývaný jav Altshulera-Aronova, bol experimentálne pozorovaný metódami tunelovej
spektroskopie a fotoelektrónovej spektroskopie. 
%Pomocou týchto metód sa v posledných rokoch podarilo pozorovať aj zmeny hustoty stavov mimo
%blízkeho okolia Fermiho energie, teda za hranicami platnosti teórie Altshulera-Aronova. 
Hlavným cieľom tejto bakalárskej práce je
opísať odvodenie hustoty stavov blízko Fermiho energie a zreprodukovať výsledok Altshulera-Aronova. Na rozdiel od Altshulera a Aronova ktorí použili metódu Greenových funkcií a Feynmannových diagramov,
v tejto práci budeme prezentovať mikroskopické odvodenie založené na elementárnej kvantovej mechanike, zvládnuteľné na bakalárskej úrovni. 
%Druhým cieľom našej práce bude opísať istú aproximáciu, vďaka ktorej sa
%mikroskopická teória stane použiteľná aj mimo blízke okolie Fermiho energie. Úspechom bude, keď takto rozšírená teória vystihne experimentálne pozorovania mimo blízke okolie Fermiho energie aspoň kvalitatívne. 
Stručne zhrnuté, v našej práci ide o kvantovomechanický výpočet hustoty
stavov v elektrónovom plyne, v ktorom elektróny interagujú vzájomne cez slabú
(tienenú) Coulombovskú interakciu vo Fockovej aproximácii a zároveň sú vystavené pôsobeniu slabého
náhodného potenciálu prímesí.

\begin{flushleft}
\textbf{Kľúčové slová:} elektrón-elektrónová interakcia, disorder, hustota stavov, jav Altshulera-Aronova
\end{flushleft} 