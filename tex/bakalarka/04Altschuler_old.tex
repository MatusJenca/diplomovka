\section{Elektróny v kovovej mriežke s disorderom : Altshuler-Aronovova aproximácia}
 Doteraz sme sa zaoberali elektrónmi v dokonale homogénnej kovovej mriežke, kde sme kladný náboj mohli aproximovať homogénnym.
 Vďaka termodynamickým javom, ióny netvoria dokonalú mriežku, pretože jednotlivé ióny sa vychýlia zo svojej polohy.
 \subsection{Schr\"odingerova rovnica pre disorderovaný potenciál}
 V mriežke dochádza k porušeniu štruktúry - disorderu. V hamiltoniáne sa to prejaví tak, že miesto potenciálu $U^{ion}$ z rovnice \eqref{eq:ion} budeme mať
 náhodný potenciál $V_{dis}$. Bez elektrón-elektrónovej interakcie budeme mať:
 %disorder bez e-e interakcie
\begin{equation}
\label{eq:schr_dis}
\bigl(-\frac{\hbar^2}{2m}\laplace + V_{dis}{\vr}\bigr)\phi_m(\vr)=\E_m\phi_m(\vr) \text{.}
\end{equation}
Aj keď zanedbáme elektron-elektrónovú interakciu, čo podľa predchádzajúcich výsledkov nemôžeme spraviť, dostaneme vo všeobecnosti exaktne neriešiteľný problém.
Disorder a teda aj $V_{dis}$ je náhodný, preto treba použiť nástroje štatistickej fyziky.

Ideme riešiť problém uvažujúc elektron-elektrónovú interakciu a disorder:

%disorder s ee interakciou
\begin{equation}
 \label{eq:fock_dis}
 \bigl(-\frac{\hbar^2}{2m}\laplace + V_{dis}(\vr)\bigr)\phi_m(\vr)-\sum_{\forall m'} \int d\vrp \phi^*_{m'}(\vr)\phi_{m}(\vrp)V(|\vr-\vrp|)\phi_m'(\vr)=E_m\phi_m(\vr) \text{,}
\end{equation}
kde sme zanedbali Hartreeho člen, ktorý sa narozdiel od \eqref{eq:fock_final} nevypadne, pretože už nemôžme použiť pudingový model.
Neskôr ukážeme že ho môžme zanedbať.

V tomto prípade nemôžme problém riešiť jednoduchým dosadením rovinných vĺn. Namiesto toho požijeme poruchovú teóriu.

Nech \eqref{eq:schr_dis} je neporušený hamiltonián $\hat{H^{0}}$:

\begin{equation}
\label{eq:aa_h0}
\hat{H^{0}}=(-\frac{\hbar^2}{2m}\laplace + V_{dis}{\vr}) \text{.}
\end{equation}
V našom prípade bude porucha Fockov člen:
\begin{equation}
 \label{eq:porucha}
 \hat{H'}=\sum_{\forall m'} \int d\vrp \phi^*_{m'}(\vr)\phi_{m}(\vrp)V(|\vr-\vrp|)\phi_m'(\vr) \text{.}
\end{equation}
Ideme riešiť nultý rád poruchovej teórie, teda do $\hat{H^0}+\hat{H'}$ dosadíme riešenia \eqref{eq:schr_dis} $\phi^{(0)}(\vr)$.
Pre väčšiu prehľadnosť výrazov už indexy (0) nepíšeme:
\begin{equation}
 \label{eq:fock_dis_erg}
 E_m=\E_m-\sum_{\forall m'} \int d\vrp \int d\vr \phi^*_{m'}(\vrp) \phi_{m}(\vrp)\phi^*_{m'}(\vr)\phi_{m'}(\vr)V(|\vr-\vrp|) \text{.}
\end{equation}
Kde $\E_m$ je vlastná energia neporušeného hamiltoniánu \eqref{eq:aa_h0}
Potenciál e-e interakcie Fourierovsky pretransformujeme:
\begin{equation}
 \label{eq:V_ft}
 V(|\vr-\vrp|)=\ftk{|\vr-\vrp|}{\vq}{V(|\vq|)}\text{,}
\end{equation}
a dosadíme do \eqref{eq:fock_dis_erg}:
\begin{equation}
\label{eq:erg_V_ft}
 E_m=\E_m-\sum_{\forall m'} \int d\vq V(|\vq|) |\bra{\phi_m}e^{i\vq\vr}\ket{\phi_{m'}}|^2 \text{.}
\end{equation}
Teraz máme rovnicu s neznámymi vlnovými funkciami $\phi_m$ a $\phi_m'$ a neznámou energiu $\E_m$. Tieto energie máme pre jeden konkrétny disorder.
Disorder je náhodná mikroskopická veličina, teda môžme cez ňu vystredovať:
\begin{equation}
\label{eq:erg_meandis}
 E_m=\overline{\E_m}-\sum_{\forall m'} \int d\vq V(|\vq|) \overline{|\bra{\phi_m}e^{i\vq\vr}\ket{\phi_{m'}}|^2} \text{.}
\end{equation}
Ak je disorder slabý, teda disorderovaný potenciál sa len veľmi málo líši od neporušeného potenciálu $U^{ion}$,
bude platiť parabolický disperzný zákon $\E(\vk_m)=\frac{\hbar \vk_m^2}{2m}$. Jednej neznámej sme sa zbavili,
ďalších sa tiež vieme zbaviť,ak si uvedomíme, že na výpočet $E_m$ potrebujeme len maticový element:
%hladame teda maticovy element
\begin{equation}
\label{eq:aa_matrix_element}
 M_{mm'}=\overline{|\bra{\phi_m}e^{i\vq\vr}\ket{\phi_{m'}}|^2} \text{.}
\end{equation}
\subsection{semiklasická aproximácia pravdepodobnosti výskytu častice v disorderovanom potenciáli}
Vráťme sa na chvíľu do klasickej fyziky, a pozrime sa na elektrón v kove ako na klasickú časticu,
ktorá má presne určenú polohu a hybnosť. Pri Fermiho energii sa častica voľne pohybuje rýchlosťou $v_F=\sqrt{\frac{2E_F}{m}}$
Častici teraz do cesty náhodne rozhádžeme body na ktoré bude narážať. V našom prípade body predstavujú ióny.
Predpokladáme, že elektrón pri interakcii iónom nestratí svoju energiu. To znamená že len zmení svoj smer.

Dôsledok našej úvahy je taký, že elektrón sa chová ako Brownovská častica, náhodne sa pohybujúca
po krokoch dĺžky $l=v_F \tau$  kde $\tau$ je stredný rozptylový čas elektrónu na disordri.

Brownoská častica spĺňa difúznu rovnicu:

%pravdepodobnost vyskytu brownovskej castice or r,t
\begin{equation}
 \label{eq:diffusion}
 P(\vr,t)=\frac{1}{(4\pi Dt)^{3/2}}e^{-\frac{|\vr-\vr_0|^2}{4Dt}}
\end{equation}
, kde difúzny koeficient $D$ je úmerný Fermiho energii:
\begin{equation}
\label{eq:diff_coef}
 D=\frac{1}{3}v_Fl
\end{equation}
Máme pravdepodobnosť výskytu častice ktorá priamo súvisí s jej vlnovou funkciou. Preto môžem postulovať:
%Postulat: elektron sa v blizkosti EF sprava ako brownovska castica
\begin{equation}
 \label{eq:aa_postulate}
 P(\vr,t)=|\psi^*(\vr,t)\psi(\vr,t)|
\end{equation}
Stacionárne stavy elekrónu, riešenia \eqref{eq:fock_dis} $\phi_m(\vr)$ tvoria ortogonálnu bázu.
. Nestacionárny stav $\psi(\vr)$ sa dá napísať ako  superpozícia $\phi_m(\vr)$ s pridanou časovou časťou $e^{-i\frac{E_m}{\hbar}t}$:
\begin{equation}
 \label{eq:aa_psi_sum}
 \psi(\vr,t)=\frac{1}{\sqrt{n}}\sum_m \phi_m^*(\vr_0)\phi_m(\vr)e^{-i\frac{E_m}{\hbar}t}
\end{equation}

Rozvinutú funkciu $\psi(\vr,t)$ \eqref{eq:aa_psi_sum} dosadím do postulátu \eqref{eq:aa_postulate}.
Tu však nastáva problém : difúzna rovnica je klasická, a v koeficiente $D$ obsahuje ostrú hodnotu energie. My ju však porovnávame s kvadrátom
vlnovej funkcie $\psi(\vr,t)$ ktorá popisuje kvantový objekt, vlnový balík, napríklad Gausovský. Vlnový balík nemá ostrú hodnotu energie,
a vo všeobecnosti nemusí zodpovedať energii v $D$. Preto sa treba obmedziť na veľmi malý interval okolo $E_F$. Teda treba brať len také
$m$, že energia  $E_m=E_F\pm \Delta E$. Tento predpoklad platí pre všetky ďalšie sumovania cez $m$.

Po dosadení do \eqref{eq:aa_postulate} porovnáme s difúznou rovnicou:
%dosadenim dostanem rovnicu
\begin{equation}
 \label{eq:aa_matrix_element_eq}
 \frac{1}{n}\sum_m \sum_{m'} \phi_m^*(\vr_0)\phi^*_{m'}(\vr)\phi_m(\vr)\phi_{m'}(\vr_0)e^{-i\frac{E_m-E_{m'}}{\hbar}t}=\frac{1}{(4\pi Dt)^{3/2}}e^{-\frac{|\vr-\vr_0|^2}{4Dt}}
\end{equation}

Kde $n$ je počet členov sumy cez $m$. Táto rovnica sa dá upraviť a vyjadriť z nej maticový element \eqref{eq:aa_matrix_element}.
Úpravy sú však netriviálne a neintuitívne, preto ich uvedieme detailnejšie. Pre väčšiu prehľadnosť textu budeme riešiť ľavú a
pravú stranu rovnice osobitne.
\subsection{Výpočet maticového elementu}
Obe strany rovnice prenásobím $e^{-i\vq(\vr-\vr_0)}$. Potom integrujeme cez $\int d\vr$ a $\int d\vr_0$ a nakoniec prenásobím $\frac{1}{V}$
%upravujem lavu stranu
\begin{align*}
&\frac{1}{nV}\sum_m \sum_{m'} \int d\vr_0  \phi_m^*(\vr_0)\phi_{m'}(\vr_0) e^{i\vq\vr_0} \int d\vr e^{-i\vq\vr}\phi_m(\vr)\phi_{m'}(\vr)e^{-i\frac{E_m-E_{m'}}{\hbar}t} \\
&\frac{1}{nV}\sum_m \sum_{m'}|\int d\vr e^{-i\vq\vr}\phi_m(\vr)\phi_{m'}(\vr)|^2 e^{-i\frac{E_m-E_{m'}}{\hbar}t}\\
&\frac{1}{nV}\sum_m \sum_{m'} M_{mm'} e^{-i\frac{E_m-E_{m'}}{\hbar}t}\\
&\frac{1}{nV}\sum_m \sum_{m'} M_{mm'} \int_0^{\infty} dt e^{-i\frac{E_m-E_{m'}}{\hbar}t} e^{i\omega t})
\end{align*}
Vidíme, že na ľavej strane sme dostali maticový element $M_{mm'}$ z \eqref{eq:aa_matrix_element}. Teraz sa musíme zbaviť časovej súradnice.
Preto ju Fourierovsky pretransformujeme ale vieme, že len kladné časy majú zmysel, preto výraz prenásobený $e^{(i\omega t)}$ integrujem cez čas od $0$ po $\infty$.
Urobíme pomocný výpočet:
 Výraz $Re(\int_0^{\infty} dt e^{-i\frac{E_m-E_{m'}}{\hbar}t} e^{i\omega t}))$ viem upraviť nasledovne:
\begin{align*}
 Re(\int_0^{\infty} dt e^{-i\omega_{mm'}t} e^{i\omega t})=&\\
 \frac{1}{2} (\int_0^{\infty} dt e^{-i(\omega_{mm'}t-\omega)t}+\int_0^{\infty} dt e^{i(\omega_{mm'}t-\omega)t})&=
  \frac{1}{2} (\int_0^{\infty} dt e^{-i(\omega_{mm'}t-\omega)t}+\int_{-\infty}^{0} dt e^{-i(\omega_{mm'}t-\omega)t})&=\\
  \frac{1}{2} \int_{-\infty}^{\infty} dt e^{-i(\omega_{mm'}t-\omega)t}&=\pi \delta(\omega_{mm'}-\omega)
\end{align*}
Kde $\omega_{mm'}=\frac{E_m-E_{m'}}{\hbar}$. Takto dostaneme delta funkciu, ktorú vieme dosadiť do ľavej strany rovnice.
Po dosadení dostanem:
\begin{equation}
 \label{eq:aa_matrix_LHS}
 \frac{1}{nV}\sum_m \sum_{m'} M_{mm'} \pi \delta(\omega_{mm'}-\omega)
\end{equation}
Teraz máme delta funkciu obsahujúcu $\omega_{mm'}$ resp. $E_m'$. Preto by sme mali integrovať cez $E_m'$. Po prejdení od sumy k integrálu a do energetických
súradníc dostávame.
%to viem po prechode od sumy k integralu a do energetickych suradnic
\begin{equation}
 \frac{\pi \hbar}{n}\sum_m \int dE_{m'} \rho(m') \delta(E_m-E_{m'}+\hbar \omega) M_{mm'}=\frac{\pi \hbar}{n}\sum_m \rho(E_{m'}+\hbar\omega)M_{(E_m)(E_m+\hbar\omega)}
\end{equation}
kde $\rho(E)$ je hustota stavov. Takto sme si upravili ľavú stranu rovnice. Presne tie isté úpravy musíme vykonať na pravej strane.

Teraz ideme upravovať pravú stranu rovnice. Musíme robiť tie isté kroky ako na ľavej strane. Najskôr prenásobíme $e^{iq(\vr-\vr_0)}$ a integrujeme cez $\vr$ a $\vr_0$
a normujeme objemom $\frac{1}{V}$
\begin{align*}
 \frac{1}{(4\pi Dt)^{3/2}}e^{-\frac{|\vr-\vr_0|^2}{4Dt}}&=\\
 \frac{1}{(4\pi Dt)^{3/2}}\frac{1}{V} \int d\vr \int d\vr_0 e^{-\frac{|\vr-\vr_0|^2}{4Dt}}e^(-i\vq(\vr-\vr_0))&=\\
 \frac{1}{(4\pi Dt)^{3/2}}\int d\vrp e^{-\frac{|\vrp|^2}{4Dt}}e^(-i\vq \vrp)
\end{align*}
Bez zmeny hraníc integrálu cez $\vr$ vieme substituovať $\vrp=\vr-\vr_0$. Integrál cez $\vr_0$ je normovaný, teda sa z neho stane $1$.
Výraz postupne integrujeme v premenných $x$,$y$,$z$. Pre každú premennú upravíme exponent na úplný štvorec a substitúciou prejdeme na Laplaceov integrál.
$\int_{-\infty}^{\infty} dx e^{-x^2}=\sqrt{\pi}$. Uvedieme len pre súradnicu $x$, ostatné sa integrujú analogicky:
\begin{align*}
  \frac{1}{\sqrt{4\pi Dt}}\int dx e^{-\frac{x^2}{4Dt}}e^(-iq_x x)&=\\
  \frac{1}{\sqrt{4\pi Dt}}\int dx e^{-\frac{(x-(2iq_xt))^2}{4Dt}-q_x^2Dt}&=\\
  \frac{1}{\pi}\int ds e^{s^2} e^{-q_x^2Dt}&= e^{-q_x^2Dt}
\end{align*}
To isté dostaneme pre $y$ a $z$. Výsledný tvar pravej strany bude $e^{-q^2Dt}$
Teraz musíme pravú stranu prenásobiť $e^{i\omega t}$, integrovať cez čas, a nakoniec zobrať reálnu časť. Tu je však integrál jednoduchý:
\begin{equation}
\label{eq:aa_matrix_RHS}
 Re{\int_0^{\infty} e^{i\omega t}e^{-q^2Dt}}=Re(\frac{1}{-i\omega+q^2D})=\frac{q^2D}{\omega^2+q^4D^2}
\end{equation}
Z ľavej \eqref{eq:aa_matrix_LHS} a pravej \eqref{eq:aa_matrix_RHS} strany rovnice vieme vyjadriť maticový element.
\begin{equation}
 \label{eq:aa_matrix_element_final}
 M_{mm'}=\frac{\hbar D q^2}{\rho(E_m')(E_m-E_{m'})^2+(\hbar Dq^2)^2}
\end{equation}


\subsection{Výpočet hustoty stavov disorderovaného potenciálu}
Pomocou maticového elementu \eqref{eq:aa_matrix_element_final} vieme zistiť strednú energiu elektrónu v slabo disorderovanom potenciáli
dosadením do rovnice \eqref{eq:erg_meandis}, a potom numericky určiť hustotu stavov podľa \eqref{eq:rho}.
To však robiť nebudeme, namiesto toho určíme analyticky priamo hustotu stavov.Výhodou bude, že dostaneme výraz,
ktorý sa pohodlnejšie porovnáva s konkrétnym experimentom.
Naša úvaha bude zavádzať ďalšie aproximácie, ale tie budú malé oproti už zavedeným.

Napíšme si vzťah pre energiu \eqref{eq:erg_meandis} ako
\begin{equation}
 \label{eq:aa_energy}
 \tilde E(E)=\overline\E-E_{self}(E)
\end{equation}
kde $\overline\E$ je rovné energii voľnej častice podľa kapitoly \ref{sec:free_electrons}, a $E_{self}$
je energia e-e interakcie.
Hustotu stavov vyjadríme z \eqref{eq:aa_energy}. Celú rovnicu pre energiu derivujeme podľa počtu stavov $n$.
Dostaneme inverznú hustotu stavov, hustota stavov je derivácia počtu stavov podľa energie.
\begin{align}
  \frac{d\tilde E(E)}{dn}&=\frac{d\E}{dn}-\frac{dE_{self}(E)}{dn}\\ \notag
  \frac{d\tilde E(E)}{dn}&=\frac{d\E}{dn}-\frac{dE_{self}(E)}{dE}\frac{dE}{dn}\\ \notag
  \label{eq:aa_dos_invert}
  \frac{d\tilde E(E)}{dn}&=\frac{d\E}{dn}(1-\frac{dE_{self}(E)}{dE})
\end{align}
Predpokladáme, že všetky funkcie sú dostatočne hladké, a teda s deriváciami môžme pracovať ,,ako so zlomkami''.
Preto pre hustotu stavov dostaneme:
\begin{equation}\frac{dE_{self}(E)}{dE}
 \label{eq:aa_dos1}
 \rho(E)=\rho_0(E)\frac{1}{1+\frac{dE_{self}(E)}{dE}}
 \end{equation}
 Kde $\rho_0(E)$ je hustota stavov pre voľný elektrón podľa \eqref{eq:rho_par}.
 Pre malé $\frac{dE_{self}(E)}{dE}$ môžme aproximovať do prvého rádu Taylorovho rozvoja.
\begin{equation}
 \label{eq:aa_dos2}
 \rho(E)\doteq\rho_(E_F)[1-\frac{dE_{self}(E)}{dE}]
\end{equation}

Stále však musíme počítať energiu $E_{self}$.
Vo vzťahu \eqref{eq:erg_meandis} prejdem od sumy cez $m'$ k integrálu.
Po integrovaní v energetických súradniciach ($dm'=\rho(E')dE'$)  maticového elementu \eqref{eq:aa_matrix_element_final} dostanem:
\begin{equation}
 \label{eq:aa_self_energy}
 E_{self}=-\int_{0}^{E_F}dE' \int \frac{d\vq}{8\pi^3}V(q)\frac{\rho(E)\hbar D q^2}{(\hbar D q^2)+(E-E')}
\end{equation}


Zavedením jednoduchých substitúcii $\epsilon=E-E_F$ $\epsilon'=E'-E_F$ integrál prejde na
\begin{equation}
\label{aa_selfenergy_subst1}
E_{self}=-\int_{E_F}^{0}d\epsilon \int \frac{d\vq}{8\pi^4}V(\vq)\frac{\hbar D q^2}{(\hbar Dq^2)^2+(\epsilon-\epsilon')^2}
\end{equation}
Po substitúcii člena v zátvorke:
$\epsilon''=\epsilon-\epsilon'$ dostanem
\begin{equation}
\label{eq:aa_selfenergy_subst_2}
E_{self}=\int_{0}^{\epsilon}d\epsilon \int \frac{d\vq}{8\pi^4}V(\vq)\frac{\hbar D q^2}{(\hbar Dq^2)^2+(\epsilon'')^2}
\end{equation}

Pretože naše doterajšie úvahy platia v okolí Fermiho Energie, môžme rádovo odhadnúť energiu na  $E \sim E_F=10$eV
Elektrón elektrónová interakcia bude rádovo Coulombovská interakcia 2 elektrónov $E_{ee}\sim \frac{e^2}{4\pi\epsilon_0 k_F^{-1}}=10^{-2}E_F$. Preto môžem urobiť takzvanú aproximáciu nekonečného pásu,
čo znamená integrovať do nekonečna. Po substitúciách sa aproximácia prejaví ako:
\begin{equation}
\label{eq:aa_selfenergy_infinite}
E_{self}=\int_{\epsilon}^{\infty}d\epsilon \int \frac{d\vq}{8\pi^4}V(\vq)\frac{\hbar D q^2}{(\hbar Dq^2)^2+(\epsilon'')^2}
\end{equation}


Z definície derivácie $\lim_{x \to 0}\frac{df(x)}{dx}=\frac{f(x+\Delta x)-f(x)}{\Delta x}$ vieme vyjadriť deriváciu self energie ako:
\begin{equation}
 \label{eq:aa_selfenergy_der}
 \frac{dE_{self}(\epsilon)}{d\epsilon}=\int \frac{d\vq}{8\pi^3}V(\vq)\frac{\hbar D q^2}{(\hbar Dq^2)^2+(\epsilon)^2}
\end{equation}
Týmto sme vyriešili jeden integrál, ostáva nám integrovať cez $d\vq$. Za potenciál $V(q)$ dosadíme tienený potenciál
\eqref{eq:screenpot_FT}. Dostaneme integrál, ktorý závisí už len na veľkosti $q$


\begin{equation}
 \frac{dE_{self}(\epsilon)}{d\epsilon}=4\pi \int_0^\infty \frac{dq}{8\pi^3}\frac{e^2}{\epsilon_0(q^2+k_s^2)}\frac{\hbar D q^2}{(\hbar Dq^2)^2+(\epsilon)^2}
\end{equation}
 Tento integrál vyriešime substitúciou $x=\frac{q}{k_s}$ a $a=\sqrt{\frac{|\epsilon|}{\hbar D k_s^2}}$.

 Vzniknutý výraz vieme rozložiť na parciálne zlomky:
\begin{equation}
\label{eq:aa_selfenergy_der_subst1}
\frac{e^2}{4\pi^2 \epsilon_0 \hbar D k_s^{-1}}[1+\frac{|\epsilon|^2}{\hbar^2D^2k_s^4}]\frac{2}{\pi}\int dx(\frac{1}{1+x^2}-
\frac{1}{1+(\frac{x}{a})^4}+\frac{x^2}{1+(\frac{x}{a})^4})
\end{equation}
Jednotlivé integrály vypočítame najskôr s nekonečnou hranicou:
Prvý integrál
\begin{equation}
 \label{eq:aa_int1}
 \frac{2}{\pi}\int_0^{\infty}\frac{1}{1+x^2}=1
\end{equation}
dostaneme jednoducho, druhý ako aj tretí  integrál  vieme rátať napr reziduovú vetou.
\begin{equation}
 \label{eq:aa_int2}
 \frac{2}{\pi}\int_0^{\infty}\frac{1}{1+(\frac{x}{a})^4}=a\frac{\pi}{2\sqrt{2}}=\sqrt{\frac{|\epsilon|}{\hbar D k_s^2}} \frac{\pi}{2\sqrt{2}}
\end{equation}

\begin{equation}
 \label{eq:aa_int3}
 \frac{2}{\pi}\int_0^{\infty}\frac{x^2}{1+(\frac{x}{a})^4}=a^3\frac{\pi}{2\sqrt{2}}
\end{equation}

Integrály však nemôžme rátať s nekonečnou hranicou. Dôvodom je difúzna aproximácia pre pravdepodobnosť \eqref{eq:aa_postulate}
použitá na výpočet maticového elementu. Tu sme predpokladali, že elektrón sa na disordri správa ako klasická častica,
teda je v kvantovej mechanike popísaný vlnovým balíkom s neurčitosťou energie $\Delta E=\frac{\hbar}{t}$
kde $t$ je doba života balíka. Tá však musí byť väčšia nanajvýš rovná difúznemu kroku $\tau$, lebo ak by balík zanikol skôr,
nemohli by sme hovoriť o difúzii. Z toho dostávame  $\Delta E<\frac{\hbar}{\tau}$.

Z princípu neurčitosti \eqref{eq:neurc} dostávame $\Delta q \Delta x=1$. Keďže doba života je minimálne $\tau$ , tak neurčitosť
polohy častice je $\Delta x=v_F \tau=l$. Potom platí $\Delta q=\frac{1}{k_s l}$ a teda hranicu integrovania cez $q$ musím obmedziť
na $q_{max}=\frac{1}{l}$.

Po substitúcii  na $x=\frac{q}{k_s}$ hranica prejde na  $x_{max}=\frac{1}{k_sl}$. Prvý integrál nám prejde na:
\begin{equation}
 \label{eq:aa_int1_capped}
 \frac{2}{\pi}\int_0^{(l k_s)^{-1}}\frac{1}{1+x^2}= \frac{2}{\pi}\tan^{-1}{lk_s^{-1}}\doteq \frac{2}{\pi k_s l}
\end{equation}
kde sme v poslednom kroku sme využili Taylorov rozvoj $\tan^{-1}{x}\doteq x$ pre  $x<<1$.
Teda  obmedzenie hraníc integrovania prenásobí prvý člen v zátvorke \eqref{eq:aa_selfenergy_der_subst1}  faktorom $\frac{2}{\pi k_s l}$

Pri počítaní druhého integrálu je vhodné urobiť substitúciu $y=\frac{x}{a}$, hranica prejde na $y_{max}=\frac{1}{k_sla}$
Pripomeňme vzťahy pre difúzny koeficient $D=\frac{1}{3}v_F^2 \tau$ a pre strednú voľnú dráhu  $l=v_F \tau$
Hornú hranicu potom viem prepísať na $y_{max}=\sqrt\frac{\hbar}{3\tau \epsilon}$.

\begin{equation}
 \label{eq:aa_int2_capped}
 \frac{2}{\pi}\int_0^{\sqrt\frac{\hbar}{3\tau \epsilon}} dy a\frac{1}{1+y^4}=aF(\sqrt\frac{\hbar}{3\tau \epsilon})=aF(y_{max})
\end{equation}

Kde primitívna funkcia $F(y)$ je:
\begin{equation}
 \label{eq:aa_primitive_func}
 F(y)=[\frac{1}{4\sqrt 2} \ln{y^2+\sqrt 2 y+1}-\ln{y^2-\sqrt 2 y+1} 2(\arctan{1+\sqrt 2 y  } - \arctan{1-\sqrt 2 y})]
\end{equation}
Funkciu $F(y)$ rozvinieme do Taylorovho radu. Pre jednotlivé časti dostaneme
\begin{align*}
 \arctan{\sqrt 2 y+1}&=\frac{\pi}{2}-\frac{1}{\sqrt 2 y}+\frac{1}{2 y^2}-\frac{1}{3 \sqrt 2 y^3} ... \\
 \arctan{\sqrt 2 y -1}&= -\frac{\pi}{2}+\frac{1}{\sqrt 2 y}+\frac{1}{2 y^2}+\frac{1}{3 \sqrt 2 y^3} ...\\
 \ln{y^2+\sqrt 2 y+1}&=  2 \ln y + \frac{\sqrt 2} {y}-\frac{\sqrt 2}{3y^3}\\
 \ln{y^2-\sqrt 2 y+1}&= 2 \ln y - \frac{\sqrt 2} {y}+\frac{\sqrt 2}{3y^3}
\end{align*}
Pre rozvoj $F(x)$ dostávame:
\begin{equation}
 \label{eq:aa_primitive_func_taylor}
F(y)\doteq \frac{1}{2\sqrt2\pi} - \frac{1}{3y^3}
\end{equation}
Za $y$ dosadíme $y_{max}=\sqrt\frac{\hbar}{3\tau \epsilon}$ v našom prípade je $\epsilon$ dostatočne malé,
teda možno  člen $\frac{1}{3y^3}$ zanedbať.

Dostali sme rovnaký výsledok pre \eqref{eq:aa_int2_capped} ako pre \eqref{eq:aa_int2}.
\begin{equation}
 \label{eq:aa_int2_capped_final}
  \frac{2}{\pi}\int_0^{\sqrt\frac{\hbar}{3\tau \epsilon}} dy a\frac{1}{1+y^4}=\sqrt{\frac{|\epsilon|}{\hbar D k_s^2}} \frac{\pi}{2\sqrt{2}}
\end{equation}

 Pre tretí integrál dostaneme:
 \begin{equation}
  \label{eq:aa_int3_capped}
  \frac{2a^3}{\pi}\int_0^{\infty}\frac{y^2}{1+y^4}=a^3G(x)
 \end{equation}

Primitívnu funkciu vieme vypočítať ale ak my ju máme násobenú faktorom $a^3$, teda ju môžme zanedbať.
Po zavedení ďalších substitúcii pre $U_{co}=2\hbar D k_s^2$ a $U_i=\frac{e^2}{4\pi \epsilon_0 k_s^{-1}}$ dostaneme s
\eqref{eq:aa_selfenergy_der_subst1}
\begin{equation}
\label{eq:aa_selfenergy_der_final}
\frac{dE_{self}}{d\epsilon}=\frac{2 U_i}{\pi U_{co}}[1+\frac{4 \epsilon^2}{U_{co}^2}]\frac{2}{\pi l k_s}[1-l k_s \frac{\pi\sqrt\epsilon}{2\sqrt{2\hbar D k_s^2}}]
\end{equation}

Výraz \eqref{eq:aa_selfenergy_der_final} dosadíme do rovnice pre hustotu stavov \eqref{eq:aa_dos1} s tým, že
pre malé $\epsilon$ platí $1+\frac{4 \epsilon^2}{U_{co}^2}\doteq 1$
\begin{equation}
 \label{eq:aa_dos3}
 \rho(E)=\rho_0(E_F)[1-\frac{4 U_i}{\pi^2 U_{co} lk_s}+\frac{2U_i }{\pi U_{co} \sqrt{\hbar Dk_s^2}}\sqrt\epsilon ]
\end{equation}
Keďže sme substituovali $\epsilon=E-E_F$, vieme že na Fermiho energii bude $\epsilon=0$, teda hustota stavov bude:
\begin{equation}
 \label{eq:aa_dos_fermi}
 \rho(E_F)=\rho_0(E_F)[1-\frac{4 U_i}{\pi^2 U_{co} lk_s}]
\end{equation}
Hustotu stavov potom možno skrátene písať ako:
\begin{equation}
 \label{eq:aa_dos4}
 \rho(E)=\rho(E_F)+\rho_0(E_F)\frac{4 U_i}{\pi^2 U_{co} lk_s}
\end{equation}
Zostáva nám už len vyjadriť si substituované členy. Po dosadení za substituvané premenné a za
$k_s^2=\sqrt{e^2 2\rho_0(E_F)\epsilon_0}$ dostaneme finálny Altshuler-Aronovovov vzťah
\begin{equation}
 \label{eq:aa_dos_final}
 \rho(E)=\rho(E_F)+\frac{\sqrt{|E_F-E|}}{4\sqrt \pi^2 2 (\hbar D)^{3/2}}
\end{equation}

