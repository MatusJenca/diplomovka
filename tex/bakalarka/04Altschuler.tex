\section
{Interagujúce elektróny v kove s disorderom : Altshuler-Aronovova aproximácia}
 
 Doteraz sme sa zaoberali elektrónmi v ideálnej kryštalickej mriežke v rámci modelu {\it želé}, v rámci ktorého boli náboje iónov mriežky aproximované
 priestorovo homogénnym nábojom. V reálnom kove však existujú aj rôzne odchýľky od ideálnej kryštalickej mriežky, ktoré sa zvyknú nazývať disorder. Vo veľkých 3D vzorkách
 ide najmä o náhodne rozmiestnené atómy prímesí. Tie vytvárajú náhodný potenciál $V_{dis}(\vr)$, ktorý elektróny rozptyľuje. V zhode s AA aproximáciou budeme predpokladať tzv. slabý disorder, pre ktorý
 platí, že $k_F l \gg 1$, kde $l$ je elektrónová stredná voľná dráha, spôsobená elektrónovými zrážkami s disorderom. 
Ak neuvažujeme e-e interakciu, elektrón interagujúci s disorderom je v rámci modelu {\it želé} popísaný Schrodingerovou rovnicou
\begin{equation}
\label{eq:schr_dis}
\bigl(-\frac{\hbar^2}{2m}\laplace + V_{dis}(\vr)\bigr)\phi_m^{(0)}(\vr)=\E_m\phi_m^{(0)}(\vr) \text{,}
\end{equation}
kde index $(0)$ na vlnovej funkcii $\phi_m^{(0)}(\vr)$ zdôrazňuje absenciu e-e interakcie. Posledná rovnica je exaktne riešiteľná len numericky, aj to len pre jeden špecifický náhodný potenciál  $V_{dis}(\vr)$.
Keď do rovnice \eqref{eq:schr_dis} zahrnieme e-e interakciu v Hartree-Fockovej aproximácii a Hartreeho interakciu vynecháme, dostaneme sústavu Fockovych rovníc v tvare
\begin{equation}
 \label{eq:fock_dis}
 \bigl(-\frac{\hbar^2}{2m}\laplace + V_{dis}(\vr) \bigr)\phi_m(\vr)-\sum_{\forall m'} \int d\vrp \phi^*_{m'}(\vr)\phi_{m}(\vrp)V(\vr-\vrp)\phi_m'(\vr)=E_m\phi_m(\vr) \text{.}
\end{equation}
v ktorej $\phi_m(\vr)$ a $E_m$ sú vlnové funkcie a vlastné energie elektrónov vo Fockovej aproximácii a $V(\vr-\vrp)$ je potenciálna energia e-e interakcie. Poznamenajme, že na rozdiel od modelu {\it želé} bez disorderu, teraz sa Hartreeho člen
nenuluje presne, takže jeho zanedbanie je aproximácia. V článkoch Alshulera a Aronova je ukázané, že Hartreeho príspevok k interakcii je v porovnaní s Fockovym príspevkom naozaj malý.
Opäť, presné riešenie Fockových rovníc \eqref{eq:fock_dis} je možné len numericky. V zhode s AA aproximáciou ich tu budeme riešiť v prvom ráde poruchovej teórie za predpokladu, že 
interakcia $V(\vr-\vrp)$ je slabá. 

Najprv vynásobíme Fockovu rovnicu \eqref{eq:fock_dis} zľava vlnovou funkciou $\phi^*_{m'}(\vr)$, potom na obe strany rovnice aplikujeme integrál $\int d \vr$ a zintegrujeme cez celý priestor.
Takto získaná rovnica (nepíšeme ju explicitne) je ešte stále presná vo Fockovej aproximácii. Keď v tejto rovnici nahradíme všetky vlnové funkcie v prvom ráde poruchovej teórie aproximáciou 
$\phi_m(\vr) \simeq \phi_m^{(0)}(\vr)$, rovnica po jednoduchej úprave nadobudne tvar 
\begin{equation}
 \label{eq:fock_dis_erg}
 E_m=\E_m-\sum_{\forall m'} \int d\vrp \int d\vr\ \phi^*_{m'}(\vrp) \phi_{m}(\vrp)\phi^*_{m'}(\vr)\phi_{m'}(\vr)V(\vr-\vrp) \text{,}
\end{equation}
ktorý vyjadruje Fockovu energiu $E_m$ ako energiu $\E_m$ neinteragujúceho problému \eqref{eq:schr_dis} plus Fockova oprava v prvom ráde poruchovej teórie. Zdôraznime, že funkcie 
$\phi_m^{(0)}(\vr)$ sú v rovnici $\eqref{eq:fock_dis_erg}$  kvôli jednoduchosti preznačené na $\phi_m(\vr)$. Odteraz už teda symbol $\phi_m(\vr)$
označuje presné riešenie neinteragujúceho problému \eqref{eq:schr_dis}.
Keď do rovnicu \eqref{eq:fock_dis_erg} dosadíme Fourierovu transformáciu
\begin{equation}
 \label{eq:V_ft}
 V(\vr-\vrp)=\ftkvec{(\vr-\vrp)}{\vq}{V(\vq)}\text{,}
\end{equation}
dostaneme rovnicu
\begin{equation}
\label{eq:erg_V_ft}
 E_m=\E_m-\sum_{\forall m'} \int d\vq\ V(\vq) \ |\bra{\phi_m}e^{i\vq\cdot\vr}\ket{\phi_{m'}}|^2 \text{.}
\end{equation}
Posledná rovnica platí pre jednu vzorku s jednou konkrétnou konfiguráciou disorderu. Vystredujeme túto rovnicu cez štatistický súbor vzoriek, z ktorých
každá má makroskopický rovnaký ale mikroskopický rôzny disorder. Dostaneme
\begin{equation}
\label{eq:erg_meandis}
 \overline{E_m}=\overline{\E_m}-\sum_{\forall m'} \int d\vq\ V(|\vq|) \overline{|\bra{\phi_m}e^{i\vq\cdot\vr}\ket{\phi_{m'}}|^2} \text{.}
\end{equation}
kde $\overline{X_m}$ označuje strednú hodnotu veličiny $X_m$, získanú vyššie spomenutým vystredovaním. Pre slabý disorder je rozumné predpokladať, že približne platí $\overline{\E_m}= \hbar^2 \vk_m^2/2m$.
Vzťah \eqref{eq:erg_meandis} obsahuje však aj vlnové funkcie
$\phi_m(\vr)$, ktoré nepoznáme. Našťastie, 
ani ich poznať nemusíme, pretože nám stačí vypočítať strednú hodnotu štvorca maticového elementu $M_{mm'}$,
\begin{equation}
\label{eq:aa_matrix_element}
\overline{| M_{mm'}|^2} =\overline{|\bra{\phi_m}e^{i\vq\cdot\vr}\ket{\phi_{m'}}|^2} \text{.}
\end{equation}
Výpočet urobíme v semiklasickej difúznej aproximácii, na ktorú sa spolieha aj AA teória.

Analýza vodivosti kovov so slabým disorderom ukazuje, že k vodivosti kovu prispievajú najmä elektróny z Fermiho hladiny a jej blízkeho okolia veľkosti $k_BT$, pričom tieto elektróny sa pohybujú podobne ako difundujúce klasické častice. Konkrétne, elektrón sa pohybuje rýchlosťou blízkou Fermiho rýchlosti $v_F=\sqrt{\frac{2\E_F}{m}}$ a v priemere raz za čas $\tau$ sa elasticky rozptýli v náhodnom smere. Taký elektrón má strednú voľnú dráhu
$l=v_F\tau$ a na jeho pohyb sa dá nazerať ako na difúziu klasickej Brownovskej časti, teda náhodné kráčanie s dĺžkou kroku $l$. 
Ak sa taká Brownovská častica v čase $t=0$ nachádza v polohe $\vec r = \vec r_0$, potom pravdepodobnosť, že časticu nájdeme v čase $t$ v polohe $\vec r$, je daná známym vzťahom
\begin{equation}
 \label{eq:diffusion}
 P(\vr,t)=\frac{1}{(4\pi Dt)^{3/2}}e^{-\frac{|\vr-\vr_0|^2}{4Dt}} \text{,}
\end{equation}
kde $D =\frac{1}{3}v_Fl $ je difúzny koeficient častice.
Nech $\psi(\vr,t)$ je nestacionárna vlnová funkcia častice, ktorá difunduje v jednom špecifickom disorderi. Ako sme uviedli v kapitole 1, kvantovomechanická pravdepodobnosť výskytu kvantovomechanickej častice v čase 
$t$ v bode $\vr$, je
\begin{equation}
 \label{eq:aa_pravd}
 P(\vr,t)=\psi^*(\vr,t)\psi(\vr,t) \text{.}
\end{equation}
Semiklasická difúzna aproximácia spočíva v postulovaní rovnice 
\begin{equation}
 \label{eq:aa_postulate}
 \overline{\psi^*(\vr,t)\psi(\vr,t)}=\frac{1}{(4\pi Dt)^{3/2}}e^{-\frac{|\vr-\vr_0|^2}{4Dt}} \text{,}
\end{equation}
kde na ľavej strane je $\psi^*(\vr,t)\psi(\vr,t)$ vystredované cez disorder. Ako ešte upresníme, aproximácia \eqref{eq:aa_postulate} platí rozumne pre dostatočne dlhý čas $t$.
Nestacionárny stav $\psi(\vr,t)$ sa dá rozvinúť do stacionárnych stavov $\phi_m(\vr)$ ako
\begin{equation}
 \label{eq:aa_psi_sum}
 \psi(\vr,t)=\frac{1}{\sqrt{N}}\sum_m \phi_m^*(\vr_0)\phi_m(\vr)e^{-i\frac{\E_m}{\hbar}t} \text{,}
\end{equation}
kde $N$ je počet stavov cez ktoré sa sumuje a sumovanie beží iba cez stavy $m$, ktorých energie $\E_m$ sa nachádzajú v intervale $\Delta \E$ okolo energie, ktorú má zodpovedajúca klasická častica.
Keďže častica začína náhodné kráčanie počnúc prvou zrážkou, vlnový balík \eqref{eq:aa_psi_sum} môže popisovať difúziu iba ak $t > \tau$. Energia častice popísanej 
vlnovým balíkom s dobou života $t$ má neurčitosť $\hbar/t$, takže maximálna neurčitosť počas difúzie je $\Delta \E=\hbar/\tau$.


Rozvoj  \eqref{eq:aa_psi_sum} dosadíme do postulátu \eqref{eq:aa_postulate}. Dostaneme
\begin{equation}
 \label{eq:aa_matrix_element_eq}
 \frac{1}{N}\sum_m \sum_{m'} \overline{\phi_m^*(\vr_0)\phi^*_{m'}(\vr)\phi_m(\vr)\phi_{m'}(\vr_0)e^{-i\frac{\E_m-\E_{m'}}{\hbar}t}}=\frac{1}{(4\pi Dt)^{3/2}}e^{-\frac{|\vr-\vr_0|^2}{4Dt}}\text{.}
\end{equation}

Vezmime najprv ľavú stranu rovnice \eqref{eq:aa_matrix_element_eq}. Násobime ju výrazom $e^{-i\vq(\vr-\vr_0)}$, integrujeme cez $\int d\vr$ a $\int d\vr_0$, a ešte násobíme $\frac{1}{V}$, kde V je integračný objem. 
Stredovaciu čiaru na chvíľu vynecháme a upravujeme.
\begin{align*}
&\frac{1}{NV}\sum_m \sum_{m'} \int d\vr_0  \phi_m^*(\vr_0)\phi_{m'}(\vr_0) e^{i\vq\vr_0} \int d\vr e^{-i\vq\vr}\phi_m(\vr)\phi_{m'}(\vr)e^{-i\frac{\E_m-\E_{m'}}{\hbar}t} \\
=&\frac{1}{NV}\sum_m \sum_{m'}|\int d\vr e^{-i\vq\vr}\phi_m(\vr)\phi_{m'}(\vr)|^2 e^{-i\frac{\E_m-\E_{m'}}{\hbar}t}\\
=&\frac{1}{NV}\sum_m \sum_{m'} |M_{mm'}|^2 e^{-i\frac{\E_m-\E_{m'}}{\hbar}t} \text{.}
\end{align*}
kde nám už vznikol štvorec maticového elementu $|M_{mm'}|^2$, ktorý chceme vypočítať. Teraz ešte na posledný riadok aplikujme Fourierovu transformáciu 
v tvare $Re (\int_0^{\infty} dt e^{i\omega t})$. Dostaneme
\begin{equation}
 \label{eq:aa_matrix_LHS semifinal}
\frac{1}{NV}\sum_m \sum_{m'} |M_{mm'}|^2 Re (\int_0^{\infty} dt e^{-i\frac{\E_m-\E_{m'}}{\hbar}t} e^{i\omega t}) \text{.}
\end{equation}
Upravíme si výraz $Re(\int_0^{\infty} dt e^{-i\frac{\E_m-\E_{m'}}{\hbar}t} e^{i\omega t})$:
\begin{align*}
 Re(\int_0^{\infty} dt e^{-i\omega_{mm'}t} e^{i\omega t})=&\\
 \frac{1}{2} \bigl(\int_0^{\infty} dt e^{-i(\omega_{mm'}-\omega)t}+\int_0^{\infty} dt e^{i(\omega_{mm'}-\omega)t}\bigr)&=
  \frac{1}{2} \bigl(\int_0^{\infty} dt e^{-i(\omega_{mm'}-\omega)t}+\int_{-\infty}^{0} dt e^{-i(\omega_{mm'}-\omega)t}\bigr)&=\\
  \frac{1}{2} \int_{-\infty}^{\infty} dt e^{-i(\omega_{mm'}-\omega)t}&=\pi \delta(\omega_{mm'}-\omega)\text{,}
\end{align*}
kde $\omega_{mm'}=\frac{\E_m-\E_{m'}}{\hbar}$. Výraz  \eqref{eq:aa_matrix_LHS semifinal} tak nadobudne tvar
\begin{equation}
 \label{eq:aa_matrix_LHS}
 \frac{1}{NV}\sum_m \sum_{m'} \overline{|M_{mm'}|^2 \pi \delta(\omega_{mm'}-\omega)} \text{,}
\end{equation}
kde sme už vrátili stredovanie cez disorder. Tento výraz môžeme ľahko integrovať vďaka prítomnosti delta funkcie. Integrujeme cez $\E_{m'}$ tak že prejdeme od sumy k integrálu. Dostaneme
\begin{equation}
 \frac{\pi \hbar}{N}\sum_m  \overline{\int d\E_{m'} \rho(\E_{m'}) \delta(\E_m-\E_{m'}+\hbar \omega) |M_{mm'}|^2} =\frac{\pi \hbar}{N}\sum_m \rho(\E_{m}+\hbar\omega)\overline{|M_{(\E_m)(\E_m+\hbar\omega)}|^2} \text{,}
\end{equation}
kde $\rho(\E)$ je hustota stavov, ktorú pre slabý disorder môžeme približne považovať za hustotu stavov voľných elektrónov a vyňať ju zo stredovania. Konečne, sumu $N^{-1}\sum_m$ môžeme chápať ako stredovanie cez stavy $m$ a 
dostávame záverečný výsledok
\begin{equation}
 \label{eq:aa_matrix_LHS vysledok}
\pi \hbar \rho(\E_{m}+\hbar\omega)  \overline{|M_{(\E_m)(\E_m+\hbar\omega)}|^2} \text{,}
\end{equation}
ktorý chápeme ako vystredovaný cez $m$.


Teraz tým istým spôsobom upravíme pravú stranu rovnice \eqref{eq:aa_matrix_element_eq}. Násobime ju výrazom $e^{-i\vq(\vr-\vr_0)}$, integrujeme cez $\int d\vr$ a $\int d\vr_0$, a násobíme $\frac{1}{V}$. Dostaneme 
\begin{equation}
\label{eq:aa_matrix_RHS begin}
 \frac{1}{(4\pi Dt)^{3/2}}\frac{1}{V} \int d\vr \int d\vr_0 e^{-\frac{|\vr-\vr_0|^2}{4Dt}}e^{-i\vq(\vr-\vr_0)} =
 \frac{1}{(4\pi Dt)^{3/2}}\int d\vrp e^{-\frac{|\vrp|^2}{4Dt}}e^{-i\vq \vrp} \text{,}
\end{equation}
pravú stranu môžeme faktorizovať na súčin troch rovnakých integrálov v premenných $x$,$y$,$z$ a každý vypočítať. Napr. integrál cez $x$ dá
\begin{align*}
  \frac{1}{\sqrt{4\pi Dt}}\int_{-\infty}^{\infty} dx e^{-\frac{x^2}{4Dt}}e^{-iq_x x} &=\\
  \frac{1}{\sqrt{4\pi Dt}}\int_{-\infty}^{\infty} dx e^{-\frac{(x-2iq_xt)^2}{4Dt}-q_x^2Dt}&=\\
  \frac{1}{\pi}\int_{-\infty}^{\infty} ds e^{-s^2} e^{-q_x^2Dt}&= e^{-q_x^2Dt}\text{.}
\end{align*}
a analogicky pre $y$ a $z$.  Týmto sa pravá strana rovnice \eqref{eq:aa_matrix_element_eq} pretransformovala na tvar $e^{-q^2Dt}$,
ktorý ešte stransformujeme Fourierovou transformáciou cez čas:
\begin{equation}
\label{eq:aa_matrix_RHS}
 Re{\int_0^{\infty}dt\ e^{i\omega t}e^{-q^2Dt}}=Re(\frac{1}{-i\omega+q^2D})=\frac{q^2D}{\omega^2+q^4D^2}\text{.}
\end{equation}
Posledný výsledok je rovný výrazu \eqref{eq:aa_matrix_LHS vysledok}, odkiaľ nachádzame hľadaný výsledok
\begin{equation}
 \label{eq:aa_matrix_element_final}
 \overline{|M_{mm'}|^2}=\frac{\hbar D q^2}{\rho(E_m')(E_m-E_{m'})^2+(\hbar Dq^2)^2}\text{.}
\end{equation}

Vezmime vzťah \eqref{eq:erg_meandis} a vystredujme ho cez všetky energie $\E_m = \E$. Dostaneme
\begin{equation}
 \label{eq:aa_energy}
 \tilde E(E)=\overline\E+E_{self}(E)\text{,}
\end{equation}
kde $\overline\E$ je rovné energii voľnej častice podľa kapitoly \ref{sec:free_electrons} a self=energia má tvar
\begin{equation}
 \label{eq:aa_self_energy}
 E_{self}=-\int_{0}^{E_F}dE' \int \frac{d\vq}{8\pi^3}V(q)\frac{\rho(E)\hbar D q^2}{(\hbar D q^2)+(E-E')}\text{,}
\end{equation}
v ktorom sme prešli od sumy cez $m'$ k integrálu cez energiu ako $dm'=\rho(E')dE'$.

Hustotu stavov vyjadríme z \eqref{eq:aa_energy}. Celú rovnicu pre energiu derivujeme podľa počtu stavov $n$.
\begin{align}
  \frac{d\tilde E(E)}{dn}&=\frac{d\E}{dn}+\frac{dE_{self}(E)}{dn}\\ \notag
  \frac{d\tilde E(E)}{dn}&=\frac{d\E}{dn}+\frac{dE_{self}(E)}{dE}\frac{dE}{dn}\\ \notag
  \label{eq:aa_dos_invert}
  \frac{d\tilde E(E)}{dn}&=\frac{d\E}{dn}(1+\frac{dE_{self}(E)}{dE}) \text{.}
\end{align}
Keďže hustota stavov je derivácia počtu stavov podľa energie, pre hustotu stavov dostávame
\begin{equation}
 \label{eq:aa_dos1}
 \rho(E)=\rho_0(E)\frac{1}{1+\frac{dE_{self}(E)}{dE}} \text{,}
 \end{equation}
 kde $\rho_0(E)$ je hustota stavov pre voľný elektrón \eqref{eq:rho_par}.
 Pre malé $\frac{dE_{self}(E)}{dE}$ urobíme Taylorov rozvoj:
\begin{equation}
 \label{eq:aa_dos2}
 \rho(E)\doteq\rho_0(E_F)[1-\frac{dE_{self}(E)}{dE}]\text{,}
\end{equation}



Zavedením jednoduchých substitúcii integrál \eqref{eq:aa_selfenergy_subst_2} prejde na

\begin{equation}
\label{eq:aa_selfenergy_subst_2}
E_{self}=\int_{0}^{\epsilon}d\epsilon' \int \frac{d\vq}{8\pi^4}V(\vq)\frac{\hbar D q^2}{(\hbar Dq^2)^2+(\epsilon')^2}\text{.}
\end{equation}
Teraz urobíme takzvanú aproximáciu nekonečného pásu, čiže dno energetického pásu presunieme do $-\infty$.
Po ďalších substitúciách sa táto aproximácia prejaví ako
\begin{equation}
\label{eq:aa_selfenergy_infinite}
E_{self}=\int_{\epsilon}^{\infty}d\epsilon' \int \frac{d\vq}{8\pi^4}V(\vq)\frac{\hbar D q^2}{(\hbar Dq^2)^2+(\epsilon')^2}\text{.}
\end{equation}


Z definície derivácie potom vieme vyjadriť deriváciu self energie ako
\begin{equation}
 \label{eq:aa_selfenergy_der}
 \frac{dE_{self}(\epsilon)}{d\epsilon}=\int \frac{d\vq}{8\pi^3}V(\vq)\frac{\hbar D q^2}{(\hbar Dq^2)^2+(\epsilon)^2}\text{.}
\end{equation}
Týmto sme vyriešili jeden integrál, ostáva nám integrovať cez $d\vq$. Za potenciál $V(q)$ dosadíme tienený Coulombov potenciál
z kapitoly 3 a prejdeme do sférických súradnic:
\begin{equation}
 \frac{dE_{self}(\epsilon)}{d\epsilon}= \frac{4\pi}{8\pi^3} \int_0^\infty dq q^2 \frac{e^2}{\epsilon_0(q^2+k_s^2)}\frac{\hbar D q^2}{(\hbar Dq^2)^2+(\epsilon)^2} \text{.}
\end{equation}
Zavedieme substitúcie substitúciou $x=\frac{q}{k_s}$ a $a=\sqrt{\frac{|\epsilon|}{\hbar D k_s^2}}$, a pravú stranu poslednej rovnice rozložíme na zlomky:
\begin{equation}
\label{eq:aa_selfenergy_der_subst1}
\frac{e^2}{4\pi^2 \epsilon_0 \hbar D k_s^{-1}}[1+\frac{|\epsilon|^2}{\hbar^2D^2k_s^4}]\frac{2}{\pi}\int dx(\frac{1}{1+x^2}-
\frac{1}{1+(\frac{x}{a})^4}+\frac{x^2}{1+(\frac{x}{a})^4})\text{.}
\end{equation}
Jednotlivé integrály vieme vypočítať napríklad prechodom do komplexnej roviny.
Pre prvý integrál dostaneme
\begin{equation}
 \label{eq:aa_int1}
 \frac{2}{\pi}\int_0^{\infty}\frac{dx}{1+x^2}=1\text{,}
\end{equation}
pre druhý
\begin{equation}
 \label{eq:aa_int2}
 \frac{2}{\pi}\int_0^{\infty}\frac{dx}{1+(\frac{x}{a})^4}=\frac{a}{\sqrt{2}}=\sqrt{\frac{|\epsilon|}{\hbar D k_s^2}} \frac{1}{\sqrt{2}}\text{.}
\end{equation}
a napokon pre tretí
\begin{equation}
 \label{eq:aa_int3}
 \frac{2}{\pi}\int_0^{\infty}dx\frac{x^2}{1+(\frac{x}{a})^4}=\frac{a^3}{\sqrt{2}}\text{.}
\end{equation}

Integrály však nemôžme rátať s nekonečnou hornou hranicou. Dôvodom je semiklasický postulát \eqref{eq:aa_postulate}
použitý na výpočet maticového elementu \eqref{eq:aa_matrix_element_final}, vďaka ktorému je maticový element \eqref{eq:aa_matrix_element_final} platný len pre $q < 1/l$. 
Hornú hranicu integrovania cez $q$ preto musíme obmedziť
na $q_{max}=\frac{1}{l}$.

Po substitúcii $x=\frac{q}{k_s}$ hranica prejde na  $x_{max}=\frac{1}{k_sl}$ a teda prvý integrál prejde na:
\begin{equation}
 \label{eq:aa_int1_capped}
 \frac{2}{\pi}\int_0^{(l k_s)^{-1}}dx \ \frac{1}{1+x^2}= \frac{2}{\pi}\arctan\bigl((lk_s)^{-1}\bigr)\doteq \frac{2}{\pi k_s l}\text{,}
\end{equation}
kde sme v poslednom kroku sme využili Taylorov rozvoj $\arctan{x}\doteq x$ pre  $x<<1$.


Pri počítaní druhého integrálu je vhodné urobiť substitúciu $y=\frac{x}{a}$, hranica prejde na $y_{max}=\frac{1}{k_sla}$.
Pripomeňme vzťahy pre difúzny koeficient, $D=\frac{1}{3}v_F^2 \tau$, a pre strednú voľnú dráhu , $l=v_F \tau$.
Hornú hranicu potom vieme prepísať na $y_{max}=\sqrt\frac{\hbar}{3\tau \epsilon}$. Druhý integrál je teda:

\begin{equation}
 \label{eq:aa_int2_capped}
 \frac{2}{\pi}a\int_0^{\sqrt\frac{\hbar}{3\tau \epsilon}} dy \frac{1}{1+y^4}=aF(\sqrt\frac{\hbar}{3\tau \epsilon})=aF(y_{max})\text{,}
\end{equation}

kde $F(y)$ je primitívna funkcia:
\begin{equation}
 \label{eq:aa_primitive_func}
 F(y)=\frac{1}{4\sqrt 2}[ \ln(y^2+\sqrt 2 y+1)-\ln(y^2-\sqrt 2 y+1) + 2\arctan(1+\sqrt 2 y ) - 2\arctan(1-\sqrt 2 y)]\text{.}
\end{equation}
Funkciu $F(y)$ rozvinieme do Taylorovho radu v nekonečne. Pre jednotlivé členy dostaneme:
\begin{align*}
 \arctan(\sqrt 2 y+1)&=\frac{\pi}{2}-\frac{1}{\sqrt 2 y}+\frac{1}{2 y^2}-\frac{1}{3 \sqrt 2 y^3} ... \\
 \arctan(\sqrt 2 y -1)&= -\frac{\pi}{2}+\frac{1}{\sqrt 2 y}+\frac{1}{2 y^2}+\frac{1}{3 \sqrt 2 y^3} ...\\
 \ln(y^2+\sqrt 2 y+1)&=  2 \ln y + \frac{\sqrt 2} {y}-\frac{\sqrt 2}{3y^3} ...\\
 \ln(y^2-\sqrt 2 y+1)&= 2 \ln y - \frac{\sqrt 2} {y}+\frac{\sqrt 2}{3y^3}...\ \text{.}
\end{align*}
Pre celý rozvoj $F(y)$ dostávame:
\begin{equation}
 \label{eq:aa_primitive_func_taylor}
F(y)\doteq \frac{1}{2\sqrt2\pi} - \frac{1}{3y^3}\text{.}
\end{equation}
Za $y$ dosadíme $y_{max}=\sqrt\frac{\hbar}{3\tau \epsilon}$ v našom prípade uvažujeme $\epsilon$ len do prvého rádu,
teda členy $\frac{1}{3y^3}$ a vyššie zanedbáme.

Pre \eqref{eq:aa_int2_capped} sme dostali rovnaký výsledok  ako pre \eqref{eq:aa_int2}:
\begin{equation}
 \label{eq:aa_int2_capped_final}
  \frac{2}{\pi}a\int_0^{\sqrt\frac{\hbar}{3\tau \epsilon}} dy \frac{1}{1+y^4}=\sqrt{\frac{|\epsilon|}{\hbar D k_s^2}} \frac{1}{\sqrt{2}}\text{.}
\end{equation}

 Pre tretí integrál dostaneme:
 \begin{equation}
  \label{eq:aa_int3_capped}
  \frac{2a^3}{\pi}\int_0^{(k_sl)^{-1}}\frac{y^2}{1+y^4}=a^3G(x)\text{.}
 \end{equation}

Primitívnu funkciu $G(x)$ vieme vypočítať podobne ako \eqref{eq:aa_primitive_func}. Prenásobením $a^3$ však dostaneme všetky členy rádu $\epsilon^{\frac{3}{2}}$ a vyššie, teda celý tretí integrál zanedbáme.

Po zavedení ďalších substitúcii pre $U_{co}=2\hbar D k_s^2$ a $U_i=\frac{e^2}{4\pi \epsilon_0 k_s^{-1}}$ vzťah
\eqref{eq:aa_selfenergy_der_subst1} prejde na:
\begin{equation}
\label{eq:aa_selfenergy_der_final}
\frac{dE_{self}}{d\epsilon}=\frac{2 U_i}{\pi U_{co}}[1+\frac{4 \epsilon^2}{U_{co}^2}]^{-1}\frac{2}{\pi l k_s}[1-l k_s \frac{\pi\sqrt\epsilon}{2\sqrt{2\hbar D k_s^2}}]\text{,}
\end{equation}
kde členy vyššieho rádu ako $\epsilon^{\frac{1}{2}}$ zanedbáme, teda platí
\begin{equation}
\label{eq:aa_selfenergy_der_final}
\frac{dE_{self}}{d\epsilon}=\frac{2 U_i}{\pi U_{co}}\frac{2}{\pi l k_s}[1-l k_s \frac{\pi\sqrt\epsilon}{2\sqrt{2\hbar D k_s^2}}]\text{.}
\end{equation}
Výraz  \eqref{eq:aa_selfenergy_der_final} dosadíme do rovnice pre hustotu stavov \eqref{eq:aa_dos2}.
\begin{equation}
 \label{eq:aa_dos3}
 \rho(E)=\rho_0(E_F)[1-\frac{4 U_i}{\pi^2 U_{co} lk_s}+\frac{2U_i }{\pi U_{co} \sqrt{2\hbar Dk_s^2}}\sqrt\epsilon ]\text{.}
\end{equation}
Keďže sme substituovali $\epsilon=E-E_F$, vieme že na Fermiho energii bude $\epsilon=0$, teda hustota stavov bude:
\begin{equation}
 \label{eq:aa_dos_fermi}
 \rho(E_F)=\rho_0(E_F)[1-\frac{4 U_i}{\pi^2 U_{co} lk_s}]\text{.}
\end{equation}
Hustotu stavov potom možno skrátene písať ako:
\begin{equation}
 \label{eq:aa_dos4}
 \rho(E)=\rho(E_F)+\rho_0(E_F)\frac{2U_i }{\pi U_{co} \sqrt{2\hbar Dk_s^2}}\sqrt\epsilon\text{.}
\end{equation}
Zostáva nám už len vyjadriť si substituované členy. Po dosadení za substituvané premenné a za
$k_s=\sqrt{\frac{e^2 \rho_0(E_F)}{\epsilon_0}}$ dostaneme finálny Altshuler-Aronovovov vzťah pre hustotu stavov
\begin{equation}
 \label{eq:aa_dos_final}
 \rho(E)=\rho(E_F)+\frac{\sqrt{|E-E_F|}}{4\sqrt 2 \pi^2 (\hbar D)^{3/2}}\text{.}
\end{equation}

