\section{Názov kapitoly}
\label{Nazovkapitoly}

Jadro je hlavná časť práce a jeho členenie je určené typom práce. Vo vedeckých a odborných prácach  má jadro spravidla tieto hlavné časti:
súčasný stav riešenej problematiky doma a v zahraničí,
\begin{itemize}
	\item cieľ práce,
	\item metodika práce a metódy skúmania,
	\item výsledky práce,
	\item diskusia. 
\end{itemize}

 
V časti Súčasný stav riešenej problematiky autor uvádza dostupné informácie a poznatky týkajúce sa danej témy. Zdrojom pre spracovanie sú aktuálne publikované práce domácich a zahraničných autorov. Podiel tejto časti práce má tvoriť približne 30 % práce.
Časť Cieľ práce jasne, výstižne a presne charakterizuje predmet riešenia. Súčasťou sú aj rozpracované čiastkové ciele, ktoré podmieňujú dosiahnutie cieľa hlavného. 

Časť Metodika práce a metódy skúmania spravidla obsahuje:
\begin{itemize}
	\item charakteristiku objektu skúmania,
	\item pracovné postupy, 
	\item spôsob získavania údajov a ich zdroje,
	\item použité metódy vyhodnotenia a interpretácie výsledkov,
	\item štatistické metódy. 
\end{itemize}

Výsledky práce a diskusia sú najvýznamnejšími časťami záverečnej práce. Výsledky (vlastné postoje alebo vlastné riešenie vecných problémov), ku ktorým autor dospel, sa musia logicky usporiadať a pri popisovaní sa musia dostatočne zhodnotiť. Zároveň sa komentujú všetky skutočnosti a poznatky v konfrontácii s výsledkami iných autorov. Ak je to vhodné, výsledky práce a diskusia môžu tvoriť aj jednu samostatnú časť a spoločne tvoria spravidla 30 až 40 \% záverečnej práce.

{\it V prípade čisto teoretických matematických prác je členenie jadra práce určené povahou problematiky. Zvyčajne prvé kapitoly oboznamujú s pojmami a výsledkami nevyhnutnými na pochopenie problematiky, nasleduje súčasný stav problematiky, ktorý logicky vyúsťuje do podrobného formulovania cieľov práce. Ďalšie kapitoly obsahujú vlastné výsledky práce.  Tieto majú byť formulované, popísané a odôvodnené tak, aby bolo možné ľahko overiť ich pravdivosť.}

\subsection{Názov podkapitoly}
\label{Nazovpodkapitoly}

Podkapitoly diplomovej práce slúžia na členenie textu bakalárskej práce s cieľom čo najväčšej prehľadnosti.


\subsubsection{Názov Tretia úroveň} 
Editujte svoju prácu v kapitolách a podkapitolách. Čísla kapitol a podkapitol (druhej a tretej úrovne) sa citujú v texte práce takto: 

... V kapitole \ref{Nazovkapitoly} sme už uviedli, že ...; ... pozri \ref{Ilustracie} ... atď. ...

Odporúčaný rozsah bakalárskej práce je 30 až 40 strán (54 000 až 72 000 znakov vrátane medzier). Do tohto rozsahu sa počíta len hlavný text, t. j. úvod, kapitoly, záver a zoznam použitej literatúry. Dôležitejší ako rozsah práce je kvalita práce a úroveň jej spracovania. Pri písaní je dôležité dbať na vyváženosť (proporcionálnosť) jednotlivých častí práce.):
\begin{itemize}
	\item úvod má spravidla  1 - 2 strany,
	\item teoreticko-metodologická časť tvorí spravidla jednu tretinu práce,
	\item ostatné kapitoly tvoria približne dve tretiny práce,
	\item záver má zvyčajne 1 - 2 strany.
\end{itemize}


