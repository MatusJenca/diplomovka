
\section
{Elektrón-elektrónová interakcia v kove s disorderom : Jav Altshulera - Aronovova}

 V predchádzajúcej kapitole sme sa zaoberali interagujúcimi elektrónmi v kove v modeli {\it želé}. Neuvažovali sme vplyv disorderu, ktorý je zvyčajne tvorený náhodne rozmiestneými prímesnými atomami. Tie vytvárajú náhodný potenciál $V_{dis}(\vr)$, s ktorým elektróny interagujú. V zhode s Altshulerom a Aronovom budeme uvažovať slabý disorder, pre ktorý
 platí, že $k_F l \gg 1$, kde $l$ je stredná voľná dráha determinovaná elektrónovými zrážkami s disorderom. Odvodíme Altshuler-Aronovov efekt, to znamená, ukážeme, že interakcia s disorderom kombinovaná s tienenou coulombovskou e-e interakciou
 spôsobuje potlačenie hustoty stavov v energetickom intervale $|E-E_F| \lesssim \hbar/\tau$, a že potlačená hustota stavov v okolí $E_F$ závisí od energie elektrónu $E$ ako $\sqrt{|E-E_F|}$.
 Odvodenie, ktoré tu prezentujeme, je spracované podľa prednášok školiteľa a konzultantky. Od prác Altshulera a Aronova \cite{Altshuler1},\cite{Altshuler3},\cite{Altshuler4}
 sa líši tým, že je založené len na elementárnej kvantovej mechanike, jeho fyzikálny obsah a získané výsledky sú však rovnaké.



Ak neuvažujeme e-e interakciu, elektrón interagujúci s disorderom je v rámci modelu {\it želé} popísaný Schrodingerovou rovnicou
\begin{equation}
\label{eq:schr_dis}
\bigl(-\frac{\hbar^2}{2m}\laplace + V_{dis}(\vr)\bigr)\phi_m^{(0)}(\vr)=\E_m\phi_m^{(0)}(\vr) \text{,}
\end{equation}
kde index $(0)$ na vlnovej funkcii označuje absenciu e-e interakcie.
Keď do rovnice \eqref{eq:schr_dis} pridáme e-e interakciu v Hartree-Fockovej aproximácii a Hartreeho príspevok k e-e interakcii zanedbáme, dostaneme Fockove rovnice
\begin{equation}
 \label{eq:fock_dis}
 \bigl(-\frac{\hbar^2}{2m}\laplace + V_{dis}(\vr) \bigr)\phi_m(\vr)-\sum_{\forall m'} \int d\vrp \phi^*_{m'}(\vr)\phi_{m}(\vrp)V(\vr-\vrp)\phi_{m'}(\vr)=E_m\phi_m(\vr) \text{.}
\end{equation}
 kde $\phi_m(\vr)$ a $E_m$ sú vlastné funkcie a vlastné energie elektrónov vo Fockovej aproximácii a $V(\vr-\vrp)$ je potenciálna energia e-e interakcie. Poznamenávame, že na rozdiel od modelu {\it želé} bez disorderu, Hartreeho interakčný príspevok sa v prítomnosti disorderu
nevynuluje exaktne, takže jeho zanedbanie je aproximácia.
Fockove rovnice \eqref{eq:fock_dis} budeme riešiť v prvom ráde poruchovej teórie za predpokladu, že
interakcia $V(\vr-\vrp)$ je slabá.

Najprv rovnicu \eqref{eq:fock_dis} prenásobíme zľava funkciou $\phi^*_{m}(\vr)$, potom na jej obe strany aplikujeme integrál $\int d \vr$ a integrujeme cez objem $V$.
Získaná rovnica (nepíšeme ju) je ešte stále presná v rámci Fockovho priblíženia. Keď v nej nahradíme všetky vlnové funkcie v prvom ráde poruchovej teórie aproximáciou
$\phi_m(\vr) \simeq \phi_m^{(0)}(\vr)$, po malých úpravách dostaneme vzťah
\begin{equation}
 \label{eq:fock_dis_erg}
 E_m=\E_m-\sum_{\forall m'} \int d\vrp \int d\vr\ \phi^*_{m'}(\vrp) \phi_{m}(\vrp)\phi^*_{m}(\vr)\phi_{m'}(\vr)V(\vr-\vrp) \text{,}
\end{equation}
  kde funkcie
$\phi_m^{(0)}(\vr)$ sú pre jednoduchosť preznačené naspäť na $\phi_m(\vr)$.
avšak teraz už $\phi_m(\vr)$ označuje riešenie neinteragujúceho problému \eqref{eq:schr_dis}.
 Rovnica \eqref{eq:fock_dis_erg}
  vyjadruje Fockovu energiu $E_m$ ako súčet energie $\E_m$ neinteragujúceho problému \eqref{eq:schr_dis}
 a Fockovej selfenergie v prvom ráde poruchovej teórie.
Keď do rovnice \eqref{eq:fock_dis_erg} dosadíme Fourierovu transformáciu
\begin{equation}
 \label{eq:V_ft}
 V(\vr-\vrp)=\ftkvec{(\vr-\vrp)}{\vq}{V(q)}\text{,}
\end{equation}
dostaneme rovnicu
\begin{equation}
\label{eq:erg_V_ft}
 E_m=\E_m-\sum_{\forall m'}\frac{1}{(2\pi)^{3}} \int d\vq\ V(q) \ |\bra{\phi_m}e^{i\vq\cdot\vr}\ket{\phi_{m'}}|^2 \text{.}
\end{equation}

Na tomto mieste musíme poruchový výsledok \eqref{eq:erg_V_ft} skorigovať. Poruchové riešenie
$\phi_m(\vr) \simeq \phi_m^{(0)}(\vr)$ môžeme na rovnicu  \eqref{eq:fock_dis} aplikovať len vtedy,
keď je interakcia $V(\vr-\vrp)$ slabá v porovnaní s interakcou s disorderom.
To je pravda pre medzielektrónové vzdialenosti $|\vr-\vrp| \gtrsim l$, pretože na vzdialenostiach $>l$ sa už interakcia s disorderom uplatňuje,
zatiaľčo interakcia $V(\vr-\vrp)$ je exponenciálne zatienená už na vzdialenosti $|\vr-\vrp| \simeq k_F^{-1} \ll l$.
Keď však dva elektróny navzájom interagujú vo vzdialenosti $|\vr-\vrp| \lesssim l$, interakcia s disorderom sa nestihne uplatniť.
To znamená že interakcia $V(\vr-\vrp)$ na vzdialenostich $|\vr-\vrp| \lesssim l$ nie je v porovnaní s disorderom slabá porucha.
Intuitívne je jasné, že správny poruchový ansatz v tomto prípade musí byť rovinná vlna
$\phi_m(\vr) \simeq \frac{1}{\sqrt V}e^{i\vk_m\cdot\vr}$, ktorá je správnym riešením Fockovej rovnice v modeli želé bez disorderu.

To vedie k nasledovnej korekcii výsledku \eqref{eq:erg_V_ft}. Integračný interval premennej $q$ rozdelíme na interval $(0, q_{max})$ a interval $q > q_{max}$, kde $q_{max}=A/l$ a $A$ je číslo blízke jednotke.
Pre $q < q_{max}$ ostáva pravá strana rovnice \eqref{eq:erg_V_ft} bez zmeny, pretože $q \lesssim 1/l$ zodpovedá medzielektrónovým vzdialenostiam $|\vr-\vrp| \gtrsim l$.
Pre $q > q_{max}$  však do pravej strany rovnice \eqref{eq:erg_V_ft} vstúpia vlnové funkcie $\phi_m(\vr) \simeq \frac{1}{\sqrt V}e^{i\vk_m\cdot\vr}$, pretože $q \gtrsim 1/l$
zodpovedá v reálnom prestore vzdialenostiam $|\vr-\vrp| \lesssim l$. Túto korekciu teraz explicitne nedosadíme, ale budeme si ju pamätať a použijeme ju neskôr.

Rovnica \eqref{eq:erg_V_ft} popisuje jednu kovovú vzorku so špecifickým náhodným potenciálom disorderu. Vystredujeme ju cez mnoho vzoriek s makroskopicky rovnakým ale mikroskopicky rôznym disorderom a dostaneme
\begin{equation}
\label{eq:erg_meandis}
 \overline{E_m}=\overline{\E_m}-\sum_{\forall m'} \frac{1}{(2\pi)^{3}} \int d\vq\ V(q) \overline{|\bra{\phi_m}e^{i\vq\cdot\vr}\ket{\phi_{m'}}|^2} \text{.}
\end{equation}
kde $\overline{X_m}$ označuje strednú hodnotu veličiny $X_m$. Pre slabý disorder je rozumné predpokladať približne $\overline{\E_m} = \hbar^2 \vk_m^2/2m$.
Vo vzťahu \eqref{eq:erg_meandis} však vystupujú aj vlnové funkcie
$\phi_m(\vr)$, ktoré pre $q < q_{max}$ nepoznáme. Našťastie, my nepotrebujeme explicitne poznať $\phi_m(\vr)$, potrebujeme vypočítať strednú hodnotu štvorca maticového elementu
\begin{equation}
\label{eq:aa_matrix_element}
\overline{| M_{mm'}|^2} =\overline{|\bra{\phi_m}e^{i\vq\cdot\vr}\ket{\phi_{m'}}|^2} \text{.}
\end{equation}
Výpočet $\overline{| M_{mm'}|^2}$ pre $q < q_{max}$ urobíme v difúznej aproximácii tak ako Altshuler a Aronov.

 Vodivostný elektrón z blízkeho okolia Fermiho nergie sa v slabom prímesovom disorderi pohybuje rýchlosťou blízkou Fermiho rýchlosti $v_F$, pričom sa raz za čas $\tau$ elasticky zrazí s prímesou a rozptýli do náhodného smeru. Elektrón má teda elastickú strednú voľnú dráhu $l=v_F\tau$ a jeho pohyb je vlastne náhodným kráčaním s dĺžkou kroku $l$. To znamená, že jeho pohyb je difúzny podobne ako pohyb klasickej Brownovskej častice.
Ak sa Brownovská častica v čase $t=0$ nachádza v mieste $\vec r = \vec r_0$, potom pravdepodobnosť nájsť túto časticu v čase $t$  v mieste $\vec r$ je
\begin{equation}
 \label{eq:diffusion}
 P(\vr,t)=\frac{1}{(4\pi Dt)^{3/2}}e^{-\frac{|\vr-\vr_0|^2}{4Dt}} \text{,}
\end{equation}
kde $D =\frac{1}{3}v_Fl $ je difúzny koeficient.
Nech $\psi(\vr,t)$ je nestacionárna vlnová funkcia elektrónu, ktorý sa pohybuje v danom disorderi.
Difúzna aproximácia spočíva v postulovaní rovnice
\begin{equation}
 \label{eq:aa_postulate}
 \overline{\psi^*(\vr,t)\psi(\vr,t)}=\frac{1}{(4\pi Dt)^{3/2}}e^{-\frac{|\vr-\vr_0|^2}{4Dt}} \text{,}
\end{equation}
kde na ľavej strane je kvantovomechanická hustota pravdepodobnosti $\psi^*(\vr,t)\psi(\vr,t)$ vystredovaná cez disorder.

Nech častica, ktorej difúziu rovnica \eqref{eq:aa_postulate} popisuje, má energiu $\E$.
Časovo závisú vlnovú funkciu $\psi(\vr,t)$ môžeme rozvinúť do stacionárnych stavov $\phi_m(\vr)$ s použitím rozvoja
\begin{equation}
 \label{eq:aa_psi_sum}
 \psi(\vr,t)=\frac{\sqrt{V}}{\sqrt{N}}\sum_m \phi_m^*(\vr_0)\phi_m(\vr)e^{-i\frac{\E_m}{\hbar}t} \text{,}
\end{equation}
kde $N$ je počet stavov $m$ cez ktoré sa sumuje a sumuje sa iba cez tie stavy $m$, ktorých energie $\E_m$ sa nachádzajú vo veľmi úzkom intervale $\Delta \E$ okolo energie $\E$ difundujúcej častice.

Interval $\Delta \E$ predstavuje neurčitosť energie $\E$. Pojem difúzie má zmysel, ak difúzny čas $t$ je aspoň niekoľko krát dlhší ako zrážkový čas $\tau$. Keďže z kvantovomechanického hladiska čas $t$ predstavuje
dobu života stavu, platí že $\Delta \E \simeq \hbar/t$. Počas difúzie bude teda $\Delta \E$ menšie ako maximálna hodnota na začiatku difúzie, $\Delta \E \simeq \hbar/\tau$. Súčasne, aby rovnica \eqref{eq:aa_postulate} dostatočne dobre popisovala
klasickú difúziu častice s energiou $\E$, musí platiť, že $\Delta \E \ll \E $. Prichádzame k obmedzeniam
\begin{equation}
 \label{eq:obmedzenia}
 \Delta \E \lesssim \hbar/\tau \ \ , \ \  \hbar/\tau \ll E_F
\end{equation}
Obmedzenie $\Delta \E \lesssim \hbar/\tau$ znamená, že difúzna aproximácia \eqref{eq:aa_postulate} je spoľahlivá len pre časy $t >> \tau$. Na tieto obmedzenia ešte v našom texte
veľa krát narazíme.  Obmedzenie $\hbar/\tau \ll E_F$ je ekvivalentné podmienke slabého disorderu $k_Fl \gg 1$.

Keď rozvoj  \eqref{eq:aa_psi_sum} dosadíme do rovnice \eqref{eq:aa_postulate}, dostaneme
\begin{equation}
 \label{eq:aa_matrix_element_eq}
 \frac{V}{N}\sum_m \sum_{m'} \overline{\phi_m^*(\vr_0)\phi^*_{m'}(\vr)\phi_m(\vr)\phi_{m'}(\vr_0)e^{-i\frac{\E_m-\E_{m'}}{\hbar}t}}=\frac{1}{(4\pi Dt)^{3/2}}e^{-\frac{|\vr-\vr_0|^2}{4Dt}}\text{.}
\end{equation}
Upravujeme ľavú stranu rovnice \eqref{eq:aa_matrix_element_eq}. Násobme ju výrazom $e^{-i\vq(\vr-\vr_0)}$, integrujme cez $\int d\vr$ a $\int d\vr_0$, a ešte prenásobme $\frac{1}{V}$.
Pre jednoduchosť na chvíľu vynecháme stredovaciu čiaru a dostávame
\begin{align*}
&\frac{1}{N}\sum_m \sum_{m'} \int d\vr_0  \phi_m^*(\vr_0)\phi_{m'}(\vr_0) e^{i\vq\vr_0} \int d\vr e^{-i\vq\vr}\phi_m(\vr)\phi_{m'}(\vr)e^{-i\frac{\E_m-\E_{m'}}{\hbar}t} \\
=&\frac{1}{N}\sum_m \sum_{m'} |M_{mm'}|^2 e^{-i\frac{\E_m-\E_{m'}}{\hbar}t} \text{.}
\end{align*}
Vo výraze na pravej strane poslednej rovnice už vidíme štvorec maticového elementu $|M_{mm'}|^2$, ktorý potrebujeme vypočítať. Keď na tento výraz aplikujeme Fourierovu transformáciu
$Re (\int_0^{\infty} dt e^{i\omega t})$, dostaneme
\begin{equation}
 \label{eq:aa_matrix_LHS semifinal}
\frac{1}{N}\sum_m \sum_{m'} |M_{mm'}|^2 Re (\int_0^{\infty} dt e^{-i\frac{\E_m-\E_{m'}}{\hbar}t} e^{i\omega t}) \text{.}
\end{equation}
Využijeme vzťah $Re(\int_0^{\infty} dt e^{-i\frac{\E_m-\E_{m'}}{\hbar}t} e^{i\omega t})=\pi \delta(\omega_{mm'}-\omega)$, kde $\omega_{mm'}=\frac{\E_m-\E_{m'}}{\hbar}$ a
výraz  \eqref{eq:aa_matrix_LHS semifinal} upravíme na tvar
\begin{equation}
 \label{eq:aa_matrix_LHS}
 \frac{1}{N}\sum_m \sum_{m'} \overline{|M_{mm'}|^2 \pi \delta(\omega_{mm'}-\omega)} \text{,}
\end{equation}
kde sme už vrátili aj stredovanie cez disorder. Sumu cez $m'$ zameníme za integrál $\int_{\E-\Delta \E/2}^{\E+\Delta \E/2} d\E_{m'} \rho(\E_{m'})$, kde hranice integrálu zohľadňujú,
že suma ide iba cez stavy  $m'$ , ktorých energie $\E_{m'}$ sa nachádzajú v intervale $\Delta \E$ okolo energie $\E$. Dostaneme
\begin{equation}
\label{eq:aa_vysl}
 \frac{\pi \hbar}{N}\sum_m  \overline{\int_{\E-\Delta \E/2}^{\E+\Delta \E/2}  d\E_{m'} \rho(\E_{m'}) \delta(\E_m-\E_{m'}+\hbar \omega) |M_{\E_{m}, \E_{m'}}|^2} =\frac{\pi \hbar}{N}\sum_m \overline{\rho(\E_{m}+\hbar\omega)|M_{\E_{m}, \E_{m}+\hbar\omega}|^2} \text{,}
\end{equation}
kde integrál $\int_{\E-\Delta \E/2}^{\E+\Delta \E/2} d\E_{m'}$ sme vypočitali s využitím $\delta$-funkcie $\delta(\E_m-\E_{m'}+\hbar \omega)$ predpokladajúc, že
$\delta$-funkcia $\delta(\E_m-\E_{m'}+\hbar \omega)$ má v intervale $(\E-\Delta \E/2,\E+\Delta \E/2)$ nulový bod.
To znamená, že rovnica \eqref{eq:aa_vysl} platí
len pre  $\hbar \omega < \Delta \E$ resp. $|\E_m-\E_{m'}| < \Delta \E$. Ak predpokladáme $\Delta \E \lesssim  \hbar/ \tau$ , tak rovnica \eqref{eq:aa_vysl} platí
pre
\begin{equation}
 \label{eq:restrikcia na energ}
|\E_m-\E_{m'}| \lesssim  \hbar/ \tau  \text{.}
\end{equation}
Spoľahlivé sú však obmedzenia $\Delta \E \ll  \hbar/ \tau$ a teda $|\E_m-\E_{m'}| \ll  \hbar/ \tau$.



Podobne, aj suma cez $m$ na pravej strane rovnice \eqref{eq:aa_vysl} ide len cez energie $\E_{m}$ v úzkom intervale $\Delta \E$ okolo energie $\E$. To znamená, že
sumu $N^{-1}\sum_m$ môžeme jednoducho vynechať s tým, že energie $\E_{m}$ nahradíme hodnotou $\E$. Dostaneme výsledok
\begin{equation}
 \label{eq:aa_matrix_LHS vysledok}
\pi \hbar \rho(\E+\hbar\omega) \overline{ |M_{\E, \E+\hbar\omega}|^2}  \text{,}
\end{equation}
ktorý sme ešte upravili vyňatím hustoty stavov $\rho(\E+\hbar\omega)$ zo stredovania.


Rovnakým spôsobom teraz pretransformujeme pravú stranu rovnice \eqref{eq:aa_matrix_element_eq}. Prenásobime ju funkciou $e^{-i\vq(\vr-\vr_0)}$, integrujeme cez $\int d\vr$ a $\int d\vr_0$, a prenásobíme $\frac{1}{V}$. Dostaneme
\begin{equation}
\label{eq:aa_matrix_RHS begin}
 \frac{1}{(4\pi Dt)^{3/2}}\frac{1}{V} \int d\vr \int d\vr_0 e^{-\frac{|\vr-\vr_0|^2}{4Dt}}e^{-i\vq(\vr-\vr_0)} =
 \frac{1}{(4\pi Dt)^{3/2}}\int d\vrp e^{-\frac{|\vrp|^2}{4Dt}}e^{-i\vq \vrp} = e^{-q^2Dt}\text{,}
\end{equation}
Výsledok $e^{-q^2Dt}$  pretransformujeme Fourierovou transformáciou cez čas:
\begin{equation}
\label{eq:aa_matrix_RHS}
 Re{\int_0^{\infty}dt\ e^{i\omega t}e^{-q^2Dt}}=Re(\frac{1}{-i\omega+q^2D})=\frac{q^2D}{\omega^2+q^4D^2}\text{.}
\end{equation}
Výsledok $\frac{q^2D}{\omega^2+q^4D^2}$ dáme do rovnosti s výrazom \eqref{eq:aa_matrix_LHS vysledok} a dostávame výsledok
pre štvorec maticového elementu,
\begin{equation}
 \label{eq:aa_matrix_element_final}
\overline{|\bra{\phi_{\E}}e^{i\vq\cdot\vr}\ket{\phi_{\E'}}|^2} \equiv \overline{|M_{\E, \E'}|^2}=\frac{1}{\pi \rho(\E')} \frac{\hbar D q^2}{(\E-\E')^2+(\hbar Dq^2)^2}\text{,}
\end{equation}
platný pre $q \gtrsim 1/l$ a pre $|\E-\E'| \ll \hbar/ \tau$.

Vezmime vzťah \eqref{eq:erg_meandis} a aplikujme naň operáciu $\frac{1}{\rho(\E)}\sum_{m} \delta(\E-\E_{m})$, kde suma ide cez všetky možné $m$. Táto operácia
zmení $\E_{m}$ na $\E$. Potom vo vťahu \eqref{eq:erg_meandis} prejdime od premennej $m'$ k premennej $\E_{m'}$, sumu $\sum_m'$ nahraďme integrálom
$\int_{0}^{\E_F} d\E_{m'} \rho(\E_{m'})$ a nakoniec preznačme $\E_{m'}$ na $\E'$. Vzťah \eqref{eq:erg_meandis} tak nadobudne tvar
\begin{equation}
\label{eq:erg_meandis modif}
 \overline{E(\E})=\overline{\E}-\int_{0}^{\E_F} d\E' \rho(\E') \frac{1}{(2\pi)^{3}} \int d\vq\ V(q) \overline{|\bra{\phi_{\E}}e^{i\vq\cdot\vr}\ket{\phi_{\E'}}|^2} \text{.}
\end{equation}