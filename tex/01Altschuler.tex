\section{Elektróny v kovovej mriežke s disorderom : Altshuler-Aronovova aproximácia}
V tejto kapitole si zavedieme základné pojmy týkajúce sa problému elektrónov v kove s  
disorderom. 

Elektrón v kovovej mriežke popisujeme vlnovou funkciou $\phi(\vr,t) $.
 Hamiltonián problému rozdelíme na tri časti:
\begin{equation}
\label{eq:01Hamiltonian}
\hat H=\hat H_0+\hat H_{int}+\hat H_{dis} \text{,}
\end{equation}
kde $H_0=\frac{-\hbar}{2m}\laplace$ je hamiltonián voľnej častice,
$H_{int}$ popisuje interakciu z iónmi v mriežke, ako aj elektrón elektrónovú (ee) interakciu
a $H_{dis}$ popisuje disorder. 

Ak uvažujeme elektrón ako voľnú časticu $H_0$, dostaneme najhrubšiu aproximáciu, kde vieme nájsť bázové funkcie $\phi(\vr)$. 
Použitím Born von Karmanovej periodickej okrajovej podmienky (PBC) dostávame  
\begin{equation}
\phi(\vr)=\frac{1}{\sqrt{V}}e^{i(\vec k\cdot\vec r-\frac{\E(k)t}{\hbar})} \text{,}
\end{equation}
kde $V$ je Born von Karmanov objem. Energia takejto vlnovej funkcie je
\begin{equation}
\E(k)=\frac{\hbar k^2}{2m}
\end{equation}

Použitím PBC dostávame diskrétne hodnoty vektora $\vk$ (vektory recipročnej mriežky). 
Elektróny sú fermióny, a v recipročnom priestore tvoria guľu. Polomer tejto gule nazývame Fermiho polomerom. Ak poznáme elektrónovú koncentráciu, vieme určiť Fermiho Polomer
\begin{equation}
 \label{eq:kf}
 k_F=(3\pi^2 n_e)^{\frac{1}{3}}\text{.}
\end{equation}
 
Energia elektrónov na Fermiho polomeri nazývame Fermiho energiou
\begin{equation}
\E_{F}=\frac{\hbar k_F^2}{2m}
\end{equation}

Pri integrovaní cez Fermiho guľu vieme zameniť súradnicu polomeru $k$ za energiu.
Jakobián takejto zámeny súradníc dáva dôležitú veličinu: hustotu stavov.

Vzťah pre hustotu stavov všeobecného systému budeme v ďalšom písať ako
\begin{equation}
\label{eq:01rho}
 \rho(E)=\frac{1}{\pi^2} \frac{dk}{dE} k^2  \ \text{.}
\end{equation}  

Druhý člen v \eqref{eq:01Hamiltonian}  popisuje Hartree-Fockova teória, ktorá je štandartne pokrytá v učebniciach, preto ju v tejto práci nebudeme rozoberať.
Uvedieme len toľko, že výpočet hustoty stavov nám pre interakciu cez Coulombovský potenciál $V(\vr_1, \vr_2)=\frac{e^2}{4\pi \epsilon_{\infty}|\vr-\vrp|}$ nám na fermiho hladine dá nulovú hustotu stavov, čo je v rozpore s realitou. Preto musíme použiť Yukkavov tienený potenciál.  
\begin{equation}
\label{eq:01yukk_pot}
V(\vr_1, \vr_2)=\frac{e^2}{4\pi \epsilon_{\infty}|\vr-\vrp|}exp^{-k_sr}\text{,}
\end{equation}
kde $k_s$ je recipročná tieniaca dĺžka. Pre náš výpočet je dôležitá jeho Fourierova transformácia
\begin{equation}
\label{eq:01pot3D}
V_3(\vq)=\frac{1}{4\pi\epsilon_0}\frac{4\pi e^2}{q^2+k_{s}^{2} }
\end{equation}

Posledný člen popisuje disorder, náhodný potenciál od prímesí. Práve jemu sa 
budeme podrobne venovať.  Bez ee interakcie máme Schr\"odingerovu rovnicu 
\begin{equation}
\label{eq:01schr_dis}
\bigl(-\frac{\hbar^2}{2m}\laplace + V_{dis}(\vr)\bigr)\phi_m(\vr)=\E_m\phi_m(\vr) \text{.}
\end{equation} 

Altschuler a Aronov (AA) riešili problém pomocou Greenových funkcii, ktoré sú nad rámec 
magisterského štúdia. Preto použijeme štandardný formalizmus vlnových funkcii.

Problém AA budeme riešiť poruchovou metódou, kde \eqref{eq:01schr_dis}bude 
neporušený problém a porucha zapríčinená interakciou bude 
\begin{equation}
 \label{eq:01porucha}
 \hat{H'}=\sum_{\forall m'} \int d\vrp \ \phi^*_{m'}(\vr)\phi_{m}(\vrp)V(|\vr-\vrp|)
 \phi_m'(\vr) \text{.}
\end{equation} 
Riešime teda $\hat H=\hat H_0+\hat H'$ v nultom ráde poruchovej teórie, takže dosádzame rovno vlnové funkcie neporušeného problému $\phi_m(r)$, $\phi_m'(r)$ a ich vlastné hodnoty $\E_m$ .  V poruchovo člene \eqref{eq:01porucha} je vhodné urobiť Fourierovu transformáciu a zaviesť $\vq=\vk-\vkp$. Po algebraických úpravách dostávame
\begin{equation}
\label{eq:01erg_V_ft} 
 E_m=\E_m-\sum_{\forall m'} \int d\vq\ V(|\vq|) |\bra{\phi_m}e^{i\vq\cdot\vr}\ket{\phi_{m'}}|^2 \text{.}
\end{equation} 
Samotný charakter $V_{dis}(\vr)$ naznačuje že úloha nie je analyticky riešiteľná. Ide
náhodný potenciál závisiaci od porúch v kryštalickej mriežke. Pre každú mriežku môžme 
mať  iné $V_{dis}(\vr)$ 
, preto aj iné vlastné vektory $\phi_m(r)$, $\phi_m'(r)$ a vlastné hodnoty $\E_m$. Je teda 
vhodné zaviesť 
štatistický súbor disorderov a hodnôt kvantového čísla $m$.

\begin{equation}
\label{eq:01erg_meandis} 
 \overline{E_m}=\overline{\E_m}-\sum_{\forall m'} \int d\vq\ V(|\vq|) \overline{|\bra{\phi_m}e^{i\vq\cdot\vr}\ket{\phi_{m'}}|^2} \text{.}
\end{equation} 
Maticový element vieme vypočítať aj bez znalosti vlnových funkcii pomocou aproximácie 
elektrónu vlnovým balíkom.

Elektrón považujeme za klasickú časticu. Disorder vieme aproximovať náhodnými bodmi 
na ktoré bude častica elasticky narážať. Ide teda o Brownov pohyb, a pravdepodobnosť výskytu častice popisuje difúzna rovnica
\begin{equation}
 \label{eq:01diffusion}
 P(\vr,t)=\frac{1}{(4\pi Dt)^{3/2}}e^{-\frac{|\vr-\vr_0|^2}{4Dt}} \text{,}
\end{equation} 
kde $D$ je difúzny koeficient úmerný Fermiho energii. 

Z vyššie uvedenej úvahy vyplýva, že môžme postulovať 
\begin{equation}
 \label{eq:01postulate}
\psi^*(\vr,t)\psi(\vr,t)\simeq\frac{1}{(4\pi Dt)^{3/2}}e^{-\frac{|\vr-\vr_0|^2}{4Dt}} \text{,}
\end{equation} 
kde $\psi(\vr,t)$  je vlnový balík častice, ktorý vieme vyjadriť v báze vlastných funkcii nášho 
skúmaného systému
\begin{equation}
 \label{eq:01psi_sum}
 \psi(\vr,t)=\frac{1}{\sqrt{N}}\sum_m \phi_m^*(\vr_0)\phi_m(\vr)e^{-i\frac{E_m}{\hbar}t} \text{.}
\end{equation} 

Postulát \eqref{eq:01postulate} platí len za určitých predpokladov.
Prísne vzaté, rovnosť \eqref{eq:01postulate} nemôže platiť, pretože 
porovnáva klasickú časticu s ostrou hodnotou energie s vlnovým 
balíkom (preto je v rovnici znak $\simeq$).  Preto sa v sume
 \eqref{eq:01psi_sum} obmedzíme na také stavt $\phi_m$ pre ktoré 
 platí $E=\E_F +\Delta E$ kde $\Delta E$ je rozumne malé. 

Dosadíme teda rozklad \eqref{eq:01psi_sum} do \eqref{eq:01postulate} 
\begin{equation}
 \label{eq:01matrix_element_eq}
 \frac{1}{N}\sum_m \sum_{m'} \phi_m^*(\vr_0)\phi^*_{m'}(\vr)\phi_m(\vr)\phi_{m'}(\vr_0)e^{-i\frac{E_m-E_{m'}}{\hbar}t}=\frac{1}{(4\pi Dt)^{3/2}}e^{-\frac{|\vr-\vr_0|^2}{4Dt}}\text{,}
\end{equation} 
kde vydelením počtom členov sumy $N$ sme dostali stednú hodnotu cez $m$.  Z rovnice \eqref{eq:01matrix_element_eq} vyjadriť z nej maticový element .

Úpravy sú však netriviálne, preto ich uvedieme detailnejšie. Pre väčšiu prehľadnosť textu budeme riešiť ľavú a
pravú stranu rovnice osobitne. 
 
 Vezmime najprv ľavú stranu rovnice \eqref{eq:01matrix_element_eq}. Násobime ju výrazom $e^{-i\vq(\vr-\vr_0)}$, integrujeme cez $\int d\vr$ a $\int d\vr_0$, a ešte násobíme $\frac{1}{V}$, kde V je integračný objem. 
Stredovaciu čiaru na chvíľu vynecháme a upravujeme.
\begin{align*}
&\frac{1}{NV}\sum_m \sum_{m'} \int d\vr_0  \phi_m^*(\vr_0)\phi_{m'}(\vr_0) e^{i\vq\vr_0} \int d\vr e^{-i\vq\vr}\phi_m(\vr)\phi_{m'}(\vr)e^{-i\frac{\E_m-\E_{m'}}{\hbar}t} \\
=&\frac{1}{NV}\sum_m \sum_{m'}|\int d\vr e^{-i\vq\vr}\phi_m(\vr)\phi_{m'}(\vr)|^2 e^{-i\frac{\E_m-\E_{m'}}{\hbar}t}\\
=&\frac{1}{NV}\sum_m \sum_{m'} |M_{mm'}|^2 e^{-i\frac{\E_m-\E_{m'}}{\hbar}t} \text{.}
\end{align*}
kde nám už vznikol štvorec maticového elementu $|M_{mm'}|^2$, ktorý chceme vypočítať. Teraz ešte na posledný riadok aplikujme Fourierovu transformáciu 
v tvare $Re (\int_0^{\infty} dt e^{i\omega t})$. Dostaneme
\begin{equation}
 \label{eq:aa_matrix_LHS semifinal}
\frac{1}{NV}\sum_m \sum_{m'} |M_{mm'}|^2 Re (\int_0^{\infty} dt e^{-i\frac{\E_m-\E_{m'}}{\hbar}t} e^{i\omega t}) \text{.}
\end{equation}
Upravíme si výraz $Re(\int_0^{\infty} dt e^{-i\frac{\E_m-\E_{m'}}{\hbar}t} e^{i\omega t})$:
\begin{align*}
 Re(\int_0^{\infty} dt e^{-i\omega_{mm'}t} e^{i\omega t})=&\\
 \frac{1}{2} \bigl(\int_0^{\infty} dt e^{-i(\omega_{mm'}-\omega)t}+\int_0^{\infty} dt e^{i(\omega_{mm'}-\omega)t}\bigr)&=
  \frac{1}{2} \bigl(\int_0^{\infty} dt e^{-i(\omega_{mm'}-\omega)t}+\int_{-\infty}^{0} dt e^{-i(\omega_{mm'}-\omega)t}\bigr)&=\\
  \frac{1}{2} \int_{-\infty}^{\infty} dt e^{-i(\omega_{mm'}-\omega)t}&=\pi \delta(\omega_{mm'}-\omega)\text{,}
\end{align*}
kde $\omega_{mm'}=\frac{\E_m-\E_{m'}}{\hbar}$. Výraz  \eqref{eq:aa_matrix_LHS semifinal} tak nadobudne tvar
\begin{equation}
 \label{eq:aa_matrix_LHS}
 \frac{1}{NV}\sum_m \sum_{m'} \overline{|M_{mm'}|^2 \pi \delta(\omega_{mm'}-\omega)} \text{,}
\end{equation}
kde sme už vrátili stredovanie cez disorder. Tento výraz môžeme ľahko integrovať vďaka prítomnosti delta funkcie. Integrujeme cez $\E_{m'}$ tak že prejdeme od sumy k integrálu. Dostaneme
\begin{equation}
 \frac{\pi \hbar}{N}\sum_m  \overline{\int d\E_{m'} \rho(\E_{m'}) \delta(\E_m-\E_{m'}+\hbar \omega) |M_{mm'}|^2} =\frac{\pi \hbar}{N}\sum_m \rho(\E_{m}+\hbar\omega)\overline{|M_{(\E_m)(\E_m+\hbar\omega)}|^2} \text{,}
\end{equation}
kde $\rho(\E)$ je hustota stavov, ktorú pre slabý disorder môžeme približne považovať za hustotu stavov voľných elektrónov a vyňať ju zo stredovania. Konečne, sumu $N^{-1}\sum_m$ môžeme chápať ako stredovanie cez stavy $m$ a 
dostávame záverečný výsledok
\begin{equation}
 \label{eq:aa_matrix_LHS vysledok}
\pi \hbar \rho(\E_{m}+\hbar\omega)  \overline{|M_{(\E_m)(\E_m+\hbar\omega)}|^2} \text{,}
\end{equation}
ktorý chápeme ako vystredovaný cez $m$.


Teraz tým istým spôsobom upravíme pravú stranu rovnice \eqref{eq:01matrix_element_eq}. Násobime ju výrazom $e^{-i\vq(\vr-\vr_0)}$, integrujeme cez $\int d\vr$ a $\int d\vr_0$, a násobíme $\frac{1}{V}$. Dostaneme 
\begin{equation}
\label{eq:aa_matrix_RHS begin}
 \frac{1}{(4\pi Dt)^{3/2}}\frac{1}{V} \int d\vr \int d\vr_0 e^{-\frac{|\vr-\vr_0|^2}{4Dt}}e^{-i\vq(\vr-\vr_0)} =
 \frac{1}{(4\pi Dt)^{3/2}}\int d\vrp e^{-\frac{|\vrp|^2}{4Dt}}e^{-i\vq \vrp} \text{,}
\end{equation}
pravú stranu môžeme faktorizovať na súčin troch rovnakých integrálov v premenných $x$,$y$,$z$ a každý vypočítať. Napr. integrál cez $x$ dá
\begin{align*}
  \frac{1}{\sqrt{4\pi Dt}}\int_{-\infty}^{\infty} dx e^{-\frac{x^2}{4Dt}}e^{-iq_x x} &=\\
  \frac{1}{\sqrt{4\pi Dt}}\int_{-\infty}^{\infty} dx e^{-\frac{(x-2iq_xt)^2}{4Dt}-q_x^2Dt}&=\\
  \frac{1}{\pi}\int_{-\infty}^{\infty} ds e^{-s^2} e^{-q_x^2Dt}&= e^{-q_x^2Dt}\text{.}
\end{align*}
a analogicky pre $y$ a $z$.  Týmto sa pravá strana rovnice \eqref{eq:01matrix_element_eq} pretransformovala na tvar $e^{-q^2Dt}$,
ktorý ešte stransformujeme Fourierovou transformáciou cez čas:
\begin{equation}
\label{eq:aa_matrix_RHS}
 Re{\int_0^{\infty}dt\ e^{i\omega t}e^{-q^2Dt}}=Re(\frac{1}{-i\omega+q^2D})=\frac{q^2D}{\omega^2+q^4D^2}\text{.}
\end{equation}
Posledný výsledok je rovný výrazu \eqref{eq:aa_matrix_LHS vysledok}, odkiaľ nachádzame hľadaný výsledok
\begin{equation}
 \label{eq:aa_matrix_element_final}
 \overline{|M_{mm'}|^2}=\frac{\hbar D q^2}{\rho(E_m')(E_m-E_{m'})^2+(\hbar Dq^2)^2}\text{.}
\end{equation}

Vezmime vzťah \eqref{eq:01erg_meandis} a vystredujme ho cez všetky energie $\E_m = \E$. Dostaneme
\begin{equation}
 \label{eq:aa_energy}
 \tilde E(E)=\overline\E+E_{self}(E)\text{,}
\end{equation}
kde $\overline\E$ je rovné energii voľnej častice selfenergia má tvar
\begin{equation}
 \label{eq:aa_self_energy}
 E_{self}=-\int_{0}^{E_F}dE' \int \frac{d\vq}{8\pi^3}V(q)\frac{\rho(E)\hbar D q^2}{(\hbar D q^2)+(E-E')}\text{,}
\end{equation}
v ktorom sme prešli od sumy cez $m'$ k integrálu cez energiu ako $dm'=\rho(E')dE'$.

Hustotu stavov vyjadríme z \eqref{eq:aa_energy}. Celú rovnicu pre energiu derivujeme podľa počtu stavov $n$.
\begin{align}
  \frac{d\tilde E(E)}{dn}&=\frac{d\E}{dn}+\frac{dE_{self}(E)}{dn}\\ \notag
  \frac{d\tilde E(E)}{dn}&=\frac{d\E}{dn}+\frac{dE_{self}(E)}{dE}\frac{dE}{dn}\\ \notag
  \label{eq:aa_dos_invert}
  \frac{d\tilde E(E)}{dn}&=\frac{d\E}{dn}(1+\frac{dE_{self}(E)}{dE}) \text{.}
\end{align}
Keďže hustota stavov je derivácia počtu stavov podľa energie, pre hustotu stavov dostávame
\begin{equation}
 \label{eq:aa_dos1}
 \rho(E)=\rho_0(E)\frac{1}{1+\frac{dE_{self}(E)}{dE}} \text{,}
 \end{equation}
 kde $\rho_0(E)$ je hustota stavov pre voľný elektrón \eqref{eq:01rho}.
 Pre malé $\frac{dE_{self}(E)}{dE}$ urobíme Taylorov rozvoj:
\begin{equation}
 \label{eq:aa_dos2}
 \rho(E)\doteq\rho_0(E_F)[1-\frac{dE_{self}(E)}{dE}]\text{,}
\end{equation}



Zavedením jednoduchých substitúcii integrál \eqref{eq:aa_selfenergy_subst_2} prejde na

\begin{equation}
\label{eq:aa_selfenergy_subst_2}
E_{self}=\int_{0}^{\epsilon}d\epsilon' \int \frac{d\vq}{8\pi^4}V(\vq)\frac{\hbar D q^2}{(\hbar Dq^2)^2+(\epsilon')^2}\text{.}
\end{equation}
Teraz urobíme takzvanú aproximáciu nekonečného pásu, čiže dno energetického pásu presunieme do $-\infty$.
Po ďalších substitúciách sa táto aproximácia prejaví ako
\begin{equation}
\label{eq:aa_selfenergy_infinite}
E_{self}=\int_{\epsilon}^{\infty}d\epsilon' \int \frac{d\vq}{8\pi^4}V(\vq)\frac{\hbar D q^2}{(\hbar Dq^2)^2+(\epsilon')^2}\text{.}
\end{equation}


Z definície derivácie potom vieme vyjadriť deriváciu self energie ako
\begin{equation}
 \label{eq:aa_selfenergy_der}
 \frac{dE_{self}(\epsilon)}{d\epsilon}=\int \frac{d\vq}{8\pi^3}V(\vq)\frac{\hbar D q^2}{(\hbar Dq^2)^2+(\epsilon)^2}\text{.}
\end{equation}
Týmto sme vyriešili jeden integrál, ostáva nám integrovať cez $d\vq$.  Prejdeme do sférických súradnic:
\begin{equation}
 \frac{dE_{self}(\epsilon)}{d\epsilon}= \frac{4\pi}{8\pi^3} \int_0^\infty dq q^2V(q)\frac{\hbar D q^2}{(\hbar Dq^2)^2+(\epsilon)^2} \text{.}
\end{equation}
%%%%%%%%

%%%%%%%%%
\begin{equation}
 \label{eq:aa_dos3}
 \rho(E)=\rho_0(E_F)[1-\frac{4 U_i}{\pi^2 U_{co} lk_s}+\frac{2U_i }{\pi U_{co} \sqrt{2\hbar Dk_s^2}}\sqrt\epsilon ]\text{.}
\end{equation}
Keďže sme substituovali $\epsilon=E-E_F$, vieme že na Fermiho energii bude $\epsilon=0$, teda hustota stavov bude:
\begin{equation}
 \label{eq:aa_dos_fermi}
 \rho(E_F)=\rho_0(E_F)[1-\frac{4 U_i}{\pi^2 U_{co} lk_s}]\text{.}
\end{equation}
Hustotu stavov potom možno skrátene písať ako:
\begin{equation}
 \label{eq:aa_dos4}
 \rho(E)=\rho(E_F)+\rho_0(E_F)\frac{2U_i }{\pi U_{co} \sqrt{2\hbar Dk_s^2}}\sqrt\epsilon\text{.}
\end{equation}
Zostáva nám už len vyjadriť si substituované členy. Po dosadení za substituvané premenné a za
$k_s=\sqrt{\frac{e^2 \rho_0(E_F)}{\epsilon_0}}$ dostaneme finálny Altshuler-Aronovovov vzťah pre hustotu stavov
\begin{equation}
 \label{eq:aa_dos_final}
 \rho_3(E)=\rho(E_F)+\frac{\sqrt{|E-E_F|}}{4\sqrt 2 \pi^2 (\hbar D)^{3/2}}\text{.}
\end{equation}


%%%%%% POTENCIALY %%%%%%%%
Vypočítali sme hustotu stavov v disorderovanom systéme v 3D priestore. V nasledujúcich 
kapitolách budeme potrebovať aj 2D a 1D prípady. 

Aproximácia cez difúznu rovnicu a maticový element, na základe ktorého sme vypočítali 3D  
prípad platí aj  v iných dimenziách. Jeden rozdiel vo výpočtoch hustoty stavov 2D a 1D však bude v integračných 
súradniciach a jakobiánoch. Druhý rozdiel bude vo fourierových transformáciach 
Yukkavovského potenciálu. Tieto transformácie, vzhľadom na zdĺhavosť výpočtov nebudeme odvodzovať, iba ich uvedieme: \cite{Altshuler4}


\begin{equation}
\label{eq:01pot2D}
V_2(\vq)=\frac{1}{2\epsilon_0}\frac{e^2}{|q|+|k_2|}
\end{equation}
\begin{equation}
\label{eq:01pot1D}
V_1(q)=\frac{1}{4\pi\epsilon_0}\frac{e^2}{e^2\rho_0+ln^{-1}(\frac{1}{q^2a^2})}
\end{equation}
Kde  $k_2=2\pi e^2 \rho_0$  je recipro4n8 tieniaca dĺžka v 2D.  

V oboch prípadoch môžme pri dostatočne malom ohraničení intehrálu cez $d\vq$ 
uvažovať limitu $q \to 0$, čo dáva vo všetkých prípadochako konštantný potenciál 
\begin{equation}
\label{eq:01potAppr}
V(q)=\frac{1}{\rho_0} \text{.}
\end{equation}

Teraz nám už zostáva iba riešiť rovnicu \eqref{eq:aa_selfenergy_der} dosadením potenciálu
\eqref{eq:01potAppr} za $V(\vq)$. Najskôr počítame pre 2D
\begin{equation}
 \frac{dE_{self}(\epsilon)}{d\epsilon}= \frac{\hbar D}{2\pi^2}\frac{1}{\rho_0} \int_0^{\frac{\sqrt{2}}{l}} dq \frac{ q^3}{(\hbar Dq^2)^2+(\epsilon)^2} \text{.}
\end{equation}
Tento integrál je ľahko vyčísliteľný , stačí použiť substitúciu $x=\hbar^2D^2q^4$. Dostaneme výsledok
\begin{equation}
\label{eq:01dE2D_1}
\frac{dE_{self}(\epsilon)}{d\epsilon}= \frac{1}{8\pi^2 \rho_0 \hbar D} ln (\frac{1+(\frac{\epsilon}{\epsilon_0})^2}{(\frac{\epsilon}
 {\epsilon_0})^2})\text{,}
\end{equation}
kde $\epsilon_0=\hbar D \frac{\sqrt 2}{l}$. Pre malé $\epsilon$, teda pre energie v okolí 
Fermiho energie, môžme člen $(\frac{\epsilon}{\epsilon_0})^2$ v čitateli  
\eqref{eq:01dE2D_1} zanedbať. Potom vieme exponent vyňať pred logaritmu a dostaneme
\begin{equation}
\label{eq:01dE2D_2}
\frac{dE_{self}(\epsilon)}{d\epsilon}= -\frac{1}{4\pi^2 \rho_0 \hbar D} ln(\frac{\epsilon}
{\epsilon_0})\text{.}
\end{equation}
Teraz nám stačí len dosadiť za $\epsilon_0$ a následne upraviť \eqref{eq:01dE2D_2}
 pomocou vzťahov $l=v_F\tau$ a $D=\frac{1}{2}v_F l$. Dostaneme finálny vzťah
 \begin{equation}
 \label{eq:01dE2Dfinal}
 \frac{dE_{self}(\epsilon)}{d\epsilon}= -\frac{1}{4\pi^2 \rho_0 \hbar D} ln(\frac{\tau\epsilon}
{\hbar})\text{.}
 \end{equation}
 
 Hustota stavov podľa rovnice \eqref{eq:aa_dos2} bude
 \begin{equation}
 \label{eq:01dos__2D} 
 \rho_2(E)=\rho_{02}(E_F)[1-\frac{1}{4\pi^2 \rho_0 \hbar D} ln(\frac{\tau(|E-E_F|)}
{\hbar})]
 \end{equation}
 Na rozdiel od 3D prípadu \eqref{eq:aa_dos_final}, kde sme dostali odmocninovú závislosť, 
v 2D prípade  \eqref{eq:01dE2Dfinal} máme logaritmickú závislosť.

V 1D prípade nemusíme zamieňať súradnice, integrujeme rovno cez jednorozmernú 
premennú $q$.

\begin{equation}
 \label{eq:aa_selfenergy_der}
 \frac{dE_{self}(\epsilon)}{d\epsilon}=\frac{\hbar D}{2\pi}\int_0^\infty dq\frac{ q^2}{(\hbar Dq^2)^2+(\epsilon)^2}\text{.}
\end{equation}

Tento integrál nieje až tak triviálny ako v 2D prípade. Vieme ho upraviť do tvaru:

\begin{equation}
\frac{1}{2\pi^2\epsilon_0 q_0 \rho_0}\int_0^{1} dx \frac{x^2}{x^4+\alpha^2}
\end{equation}
kde $\epsilon_0=\sqrt{\frac{\hbar D}{\epsilon}}$ a $\alpha=\frac{\epsilon}{\epsilon_0}$. Určíme primitívnu funkciu
\begin{equation}
\label{eq:01prim}
P(x)=\frac{1}{2\pi^2\epsilon_0 q_0 \rho_0}\frac{1}{4\sqrt {2 \alpha} }[ F(x)-G_1(x)+G_2(x)] \text{,}
\end{equation}
kde
\begin{align}
F(x)&=\ln(\frac { x^2+\alpha-\sqrt{2\alpha}x }{ x^2+\alpha+\sqrt{2\alpha}x })\\
G_1(x)&=2\arctan(1+\sqrt{\frac{2}{ \alpha}}x)\\
G_1(x)&=2\arctan(1-\sqrt{\frac{2}{ \alpha}}x) \text{.}
\end{align}
 Teraz vyhodnotíme primitívnu funkciu \eqref{eq:01prim}.
 
  Vidíme, že $P(0)=0$ pretože funkcia  $F$ má nulovú hodnotu  a $G_1$ a $G_2$ sa odčítajú na nulu. 
  
  Pre $P(1)$ môžme urobiť nasledovnú aproximáciu. Stále sme v oblasti v okolí Fermiho 
  energie, teda $\epsilon << 1$ a taktiež $\alpha<<1$. Môžme preto predpokladať $F(1)=0$.
Zostáva teda vyčísliť funkcie $G_1$ a $G_2$. Tie môžme pre $\alpha$ idúce do nuly napísať ako $G_1=2\pi$ a $G_2=-2\pi$. Z toho dostávame 

\begin{equation}
\label{eq:01primApprox}
P(1)=\frac{1}{2\pi^2\epsilon_0 q_0 \rho_0}\frac{1}{4\sqrt {2 \alpha} } 4\pi
\end{equation}    

Po aritmetických úpravách a dosadeniach dostávame.
\begin{equation}
\frac{dE_{self}(\epsilon)}{d\epsilon}=\frac{1}{2\sqrt{2}\pi \sqrt{\hbar D\epsilon}}
\end{equation}
Znova dosadíme do \eqref{eq:aa_dos2} a dostaneme
\begin{equation}
\label{eq:01dos2D}
\rho_1(E)=\rho_{01}[1-\frac{1}{2\sqrt{2}\pi \sqrt{\hbar D|E-E_F|}}]
\end{equation}
 