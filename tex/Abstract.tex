
\thispagestyle{empty}
\section*{Abstract}

Jenča, Matúš: Density of the electron states in a weakly disordered metal with weak electron-electron interaction: Altshuler-Aronov effect [Master
Thesis], Comenius University in Bratislava, Faculty of Mathematics, Physics and Informatics, Department of Experimental Physics; Supervisor: Doc. RNDr. Martin Moško, DrSc., Bratislava, 2022, 59 p.

The conduction electrons in a disordered metal at low temperatures interact predominantly with impurity disorder and with each other via the screened electron-electron interaction. Combination of these 
two interactions suppresses the density of the electron states in the vicinity of the Fermi energy in comparison with its value in a clean metal.
This local suppression of the density of states (here the vicinity means the energy interval given by the product of the Planck constant and frequency of the electron collisions with disorder) was first predicted by theoretical works of Altshuler and Aronov and later confirmed experimentally by spectroscopic methods. In the last two decades the spectroscopic methods also succeeded to detect the density of states at electron energies outside the vicinity of the Fermi level, where the Altshuler-Aronov theory is inapplicable and the density of states is usually approximated by a constant value attributed to the density of states in the clean metal.
In fact, recent experimental data indicate that the density of the states outside the vicinity of the Fermi level first increases well above the clean metal value and then decays to the clean metal value at energies far from the Fermi level. In this work we study theoretically how the e-e interaction and interaction with disorder affect the density of states outside the vicinity of the Fermi level, where
the Altshuler-Aronov theory is inapplicable and the theorists approximate the density of states by a constant without considering the recent experiments.
In the Altshuler-Aronov theory the electron wave function in the disordered metal is designed to mimic the semiclassical electron diffusion on the time scales much larger than the electron-impurity collision time.
This is the reason why the A-A theory works only at small excitation energies in the vicinity of the Fermi level. In our calculation the effect of the disorder on the electron wave function is described in the
self-consistent Born approximation which on the contrary works for the time scales shorter than the electron-impurity collision time, and thus for the excitation energies outside the vicinity of the Fermi level.
In addition, in our work the unperturbed density of states is explicitly calculated by considering the screened Fock e-e interaction, that means, it is not approximated by a constant. Finally, the obtained results are compared
with recent experiments.

\begin{flushleft}
  \textbf{Keywords:} density of the electron states, electron-electron interaction, disorder, Altshuler-Aronov effect
\end{flushleft}

