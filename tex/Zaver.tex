
Keď do čistého kovu dáme prímesný disorder, elektrón-elektrónová interakcia medzi vodivostnými elektrónmi nevyhnutne koexistuje s elektrónovou interakciou s prímesným disorderom. Altshuler a Aronov \cite{Altshuler1},\cite{Altshuler3},\cite{Altshuler4} ukázali teoreticky, že koexistencia týchto
dvoch interakcií potláča hustotu elektrónových stavov v porovnaní s hustotou stavov v čistom kove, a to takým spôsobom, že v intervale energií $|E-E_F| \lesssim \hbar/\tau$ potlačená hustota stavov vykazuje závislosť $\propto \sqrt{|E-E_F|}$. Táto teoretická predpoveď bola experimentálne mnoho krát potvrdená metódami tunelovej
spektroskopie. V posledných dvoch desaťročiach sa podarilo experimentálne pozorovať aj zmeny hustoty stavov mimo intervalu $|E-E_F| \lesssim \hbar/\tau$, kde už teória Altshulera-Aronova neplatí a kde teoretici zvyknú hustotu stavov aproximovať od energie nezávislou konštantou zodpovedajúcou hustote stavov čistého kovu.

 Experiment \cite{Mazur}, \cite{Schmitz1}, \cite{Schmitz2}, \cite{Escudero}, \cite{Moskova} však ukazuje, že mimo intervalu $|E-E_F| \lesssim \hbar/\tau$ hustota stavov s rastúcou hodnotou $|E-E_F|$ narastie až natoľko, že pre $|E-E_F| $ trochu väčšie ako $\sim \hbar/\tau$ výrazne prevýši hodnotu hustoty stavov v čistom kove. Až následne začne hustota stavov postupne klesať k hodnote zodpovedajúcej čistému kovu, ktorú dosiahne pre $|E-E_F|$ niekoľkonásobne väčšie ako $\hbar/\tau$.

V tejto práci sme preskúmali teoreticky vplyv e-e interakcie a disorderu na
hustotu elektrónových stavov v kove práve pre stavy mimo intervalu $|E-E_F| \lesssim \hbar/\tau$. Altshuler a Aronov vo svojej teórii použili elektrónovú vlnovú funkciu, ktorá pohyb elektrónu v disorderi popisuje ako semiklasickú difúziu na časoch oveľa dlhších ako zrážkový čas $\tau$. To obmedzilo platnosť ich teórie na interval $|E-E_F| \lesssim \hbar/\tau$. V našom výpočte hustoty stavov bola vlnová funkcia elektrónu v disorderi popísaná v selfkonzistentnej Bornovej aproximácii, ktorá naopak platí pre časy kratšie ako elektrónový zrážkový čas $\tau$ a teda platí pre $|E-E_F| \gtrsim \hbar/\tau$. Naviac, príspevok od stavov pre $q > 1/l$  sme započítali explicitne ako príspevok od stavov v čistom kove s Fockovou tienenou e-e interakciou.



  Na záver sme náš výpočet hustoty stavov pre energie $|E-E_F|  \gtrsim \hbar/\tau$ spojili s Altshuler-Aronovovou teóriou pre $|E-E_F|  \lesssim \hbar/\tau$. Dostali sme výsledky, ktoré sú v rozumnej a systematickej zhode so stavy zachovávajúcou hustotou stavou pozorovanou experimentálne \cite{Mazur}, \cite{Schmitz1}, \cite{Schmitz2}, \cite{Escudero}, \cite{Moskova}. Experimentálny fakt \cite{Mazur}, že stavy vypudené AA efektom z oblasti $|E-E_F| \lesssim U_{co}$ majú tendenciu sa nakopiť hneď nad energiou $U_{co}$ v oblasti veľkosti dva a až tri krát $U_{co}$, sa ukázal byť generickou vlastnosťou teórie.

