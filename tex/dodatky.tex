%%%%%%%%%%%%%%%%%%%%%%%%%%%%%%%%%%%%%%%%%%%%%%%%%%%%%%%%%%%%%%%%%%%%%%%%%%%%%%%%%%%%%%%%
%%%%%%%%%%%%%%%%%%%%%%%%%%%%%%%%%%%%%%%%%%%%%%%%%%%%%%%%%%%%%%%%%%%%%%%%%%%%%%%%%%%%%%%%
%dodatky
%%%%%%%%%%%%%%%%%%%%%%%%%%%%%%%%%%%%%%%
%%%%%%%%%%%%%%%%%%%%%%%%%%%%%%%


\newpage
  \appendix
  \section*{Dodatok A}
  \addcontentsline{toc}{section}{Dodatok A}
  \markboth{Dodatok A}{Dadatok A}
  %  \renewcommand{\thesection}{A.\arabic{section}}
\renewcommand{\theequation}{A.\arabic{equation}}
\renewcommand{\thefigure}{A.\arabic{figure}}
\setcounter{equation}{0}
\subsection*{Odvodenie Hartree-Fockovej rovnice z variačného princípu}
Potenciál $U(r)$ si znova rozdelíme na elektrónovú a iónovú časť. Tak isto si pre elektróny zavedieme nové súradnice $\tau$ ktoré v sebe zahŕňajú aj spin.
Keď budeme integrovať cez $d\tau$ myslíme tým integrál cez $dr$ a sumu cez jednotlivé spiny.

Najskôr si určíme strednú hodnotu Hamiltoniánu pre dva elektróny. Hamiltonián sa dá potom napísať ako
\begin{equation}
\label{eq:hm}
\hat H =\hat h_1+ \hat h_2 + V_{12} \text{,}
\end{equation}

kde $\hat h_i$ popisujú kinetickú energiu a potenciál od iónov pre jednotlivé elektróny a $V_{12}$ ich vzájomnú interakciu:
\begin{align}
 \hat h_i&=-\frac{\hbar}{2m}\Delta_{i}+U^{ion}(r_i) \\ \notag
 V_{12} &= \frac {e^2}{4\pi \epsilon_0|\vr-\vrp|} \text{,}
\end{align}

Teraz ideme minimalizovať strednú hodnotu hamiltoniánu:
\begin{equation}
 \label{eq:mvh}
 <H>=\int{d\tau_1d\tau_2\Psi^*(\tau_1,\tau_2)\hat H\Psi(\tau_1,\tau_2)} \text{.}
\end{equation}



Do \eqref{eq:mvh} dosadíme Slatterov determinant pre dve funkcie:
\begin{equation}
 \label{eq:Psi2}
 \Psi(\tau_1,\tau_2)=\frac{1}{\sqrt{2}}({\phi_1(\tau_1)\phi_2(\tau_2)-\phi_2(\tau_1)\phi_1(\tau_2)}) \text{,}
\end{equation}
\begin{equation}
 \label{eq:mvh2}
 <H>=\int{d\tau_1d\tau_2\bigl( \phi_1(\tau_1)\phi_2(\tau_2)-\phi_2(\tau_1)\phi_1(\tau_2)\bigr)^*\hat H(\phi_1(\tau_1)\phi_2(\tau_2)-\phi_2(\tau_1)\phi_1(\tau_2))} \text{.}
\end{equation}
Teraz dosadíme dosadíme \eqref{eq:hm} za $\hat H$ a roznásobíme. Dostaneme 12 členov.
\begin{align}
<H>=&\\ \notag
 &\frac{1}{2}\int{d\tau_1 d\tau_2 \phi^*_1(1)\phi^*_2(2)\hat h_1\phi_1(1)\phi_2(2)}+\\ \notag
 &\frac{1}{2}\int{d\tau_1 d\tau_2 \phi^*_1(1)\phi^*_2(2)\hat h_2\phi_1(1)\phi_2(2)}- \notag
 &\frac{1}{2}\int{d\tau_1 d\tau_2 \phi^*_1(2)\phi^*_2(1)\hat h_1\phi_1(1)\phi_2(2)}-\\ \notag
 &...\\ \notag
  &\frac{1}{2}\int{d\tau_1 d\tau_2 \phi^*_1(1)\phi^*_2(2)V_{12}\phi_1(1)\phi_2(2)}- \notag
  &\frac{1}{2}\int{d\tau_1 d\tau_2 \phi^*_1(2)\phi^*_2(1)V_{12}\phi_1(1)\phi_2(2)}-\\ \notag
  &\frac{1}{2}\int{d\tau_1 d\tau_2 \phi^*_1(1)\phi^*_2(2)V_{12}\phi_1(2)\phi_2(1)}+ \notag
   &\frac{1}{2}\int{d\tau_1 d\tau_2 \phi^*_1(2)\phi^*_2(1)V_{12}\phi_1(2)\phi_2(1)} \notag \text{.}
\end{align}
Integrály obsahujúce $\hat h_1$ a $\hat h_2$ budú rovnaké, pretože sa líšia len zámenou premenných. Členy z $V_{12}$ sa dajú prepísať na sumy
\begin{align}
 \label{eq:sumij}
 &\frac{1}{2} \sum_{i=1}^2{\sum_{i\neq j}^2 \int d\tau_a d\tau_b\phi_i(a)\phi_j(b)V_{ab}\phi_i(a)\phi_j(b)}-\\
 &\frac{1}{2} \sum_{i=1}^2{\sum_{i\neq j}^2 \int d\tau_a d\tau_b\phi_i(a)\phi_j(b)V_{ab}\phi_i(b)\phi_j(a)} \notag \text{.}
\end{align}

V tvare sumy sa potom dá napísať celé $<H>$:
\begin{align}
  \label{eq:hsum}
 &\sum_i^2 \int d\tau_a \phi^*_i(a)\hat h_a \phi_i(a) +\\
 &\frac{1}{2} \sum_{i=1}^2{\sum_{i\neq j}^2 \int d\tau_a d\tau_b\phi_i^*(a)\phi_j^*(b)V_{ab}\phi_i(a)\phi_j(b)}- \notag \\
 &\frac{1}{2} \sum_{i=1}^2{\sum_{i\neq j}^2 \int d\tau_a d\tau_b\phi_i^*(a)\phi_j^*(b)V_{ab}\phi_i(b)\phi_j(a)} \notag \text{.}
\end{align}

Teraz je vhodné sa zbaviť spinovej súradnice. V prvom a v druhom člene \eqref{eq:hsum} dostaneme integrovaním cez spiny 1, lebo tvoria ortogonálnu bázu.
V poslednom člene nám s rovnakých dôvodov   prežijú len paralelné spiny.

\begin{align}
 \label{eq:hsum}
 &\sum_i^2 \int d\vr_a \phi^*_i(a)\hat h_a \phi_i(a) +\\ \notag
 &\frac{1}{2} \sum_{i=1}^2{\sum_{i\neq j}^2 \int d\vr_a d\vr_b\phi_i^*(a)\phi_j^*(b)V_{ab}\phi_i(a)\phi_j(b)}-\\ \notag
 &\frac{1}{2} \sum_{i=1}^2{\sum_{i\neq j;cez || spiny}^2 \int d\vr_a d\vr_b\phi_i^*(a)\phi_j^*(b)V_{ab}\phi_i(b)\phi_j(a)} \text{,} \notag
\end{align}
Tento výsledok možno zovšeobecniť na viac elektrónov.

Teraz ideme nájsť funkcie $\phi_i^*$ ktoré minimalizujú funkcionál $E[\phi^*_i]=<H>$, v Diracovej notácí:
\begin{equation}
 \label{eq:e_func}
 E[\Psi^*]=\expval{H}{\Psi} \text{.}
\end{equation}

V minime funkcionálu musí byť variácia nulová $\delta E[\Psi^*]=0$ , kde
\begin{equation}
 \label{eq:e_var}
 \delta E=\bra{\Psi+\delta \Psi}H\ket{\Psi}-\expval{H}{\Psi} \text{.}
\end{equation}

\newpage
Toto však nieje jediná podmienka pre minimum. Funkcia $\Psi$ musí navyše spĺňať väzobnú podmienku ortogonality:

\begin{equation}
 \label{eq:ortho}
 \bra{\Psi}\ket{\Psi}=1 \text{.}
\end{equation}

Táto podmienka platí aj pre jednotlivé $\phi_i$ a dá sa prepísať do integrálneho tvaru.
\begin{equation}
 \label{eq:ortho_2}
 \int{ d\vr\ \phi^*_j\phi_i }-\delta_{ij}=0 \text{.}
 \end{equation}
Rovnicu \eqref{eq:ortho_2} vynásobíme Lagrangeovým multiplikátorom $\lambda_{ij}$. Potom odpočítame od \ref{eq:e_func} a dostaneme
\begin{equation}
 \label{eq:lagr}
 L[\phi]=\expval{H}{\Psi}-\lambda_{ij}( \int{ d\vr\ \phi^*_j\phi_i }-\delta_{ij})\text{.}
\end{equation}

Lagrangián \eqref{eq:lagr} bude rovný funkcionálu \eqref{eq:e_func} pretože sme od neho odčítali nulový člen. Ak položím variáciu $\delta L$ rovnú nule,
dostanem minimum funkcionálu \eqref{eq:e_func} ktoré navyše spĺňa väzobnú podmienku \eqref{eq:ortho}.

Teraz za $H$ dosadíme hamiltonián \eqref{eq:hm} a za $\Psi$ Slatterov determinant \eqref{eq:Psi2} pre dve funkcie.

Všimnime si, že varírovaním prvého člena sme dostali
\begin{equation}
 \label{eq:var1}
 \sum_i^2 \int d\vr_a \delta\phi^*_i(a)\hat h_a \phi_i(a)\text{,}
\end{equation}
pretože od $L[\Psi+\delta\Psi]$ sa členy s $\phi^*_i+\delta\phi^*_i$ sa nám odčítajú s pôvodnými členmi $L[\Psi]$.

Pri druhom a treťom člene máme roznásobiť $(\phi^*_i+\delta\phi^*_i)(\phi^*_j+\delta\phi^*_j)$. Tu nám prežijú len dva členy $\phi^*_i\delta\phi^*_j + \phi^*_i\delta\phi^*_j$
pretože člen s dvoma deltami je druhého rádu, čo pri variácii neuvažujeme, a člen bez delty sa znova odčíta s $L[\Psi]$. Teda suma bude vyzerať nasledovne
\begin{equation}
 \label{eq:part_sum}
 \frac{1}{2} \sum_{i=1}^2{\sum_{i\neq j}^2 \int d\vr_a d\vr_b\delta\phi^*_i(a)\phi^*_j(b)Vab\phi_i(a)\phi_j(b)+ \int d\vr_a d\vr_b\phi^*_i(a)\delta\phi^*_j(b)Vab\phi_i(a)\phi_j(b)} \text{.}
\end{equation}


Vidíme však, že tieto dva členy sa líšia len integračnými premennými, preto ich môžem napísať ako jeden, čím sa faktor $\frac{1}{2}$ stratí. Veľmi podobne viem zredukovať tretí člen.
Posledne po odčítaní $L[\Psi]$ časti nám člen s Lagrangeovým multiplikátorom prejde na
\begin{equation}
 \label{eq:part_lagr}
 \sum_j{\lambda_{ij}\phi_j(a)} \text{.}
\end{equation}

Celková variácia lagrangiánu $L[\Psi]$ bude
\begin{align}
 \label{eq:var_final}
 &\delta L=\int d\vr_a\delta\phi_i^*(a) \\ \notag
 &\bigl(\sum_i\hat h_a \phi_i(a) +
 \sum_{j\neq i} |\phi_j(a)|^2V_{ab}\phi_i(a)+
 \sum_{j \neq i || sp}\int d\vr_b \phi_i(b)^*V_{ab}\phi_i(b)\phi_j(a)-
 \sum_j \lambda_{ij} \phi_j(a)\bigr)=0 \text{.}
\end{align}

%\subsection{Hartree-Fockove rovnice}
Z nulovosti integrálu vyplýva nulovosť podinitegrálnej funkcie
Rovnosť \eqref{eq:var_final} musí platiť pre ľubovoľné malé $\delta \phi_i^*(a)$, teda
členy v zátvorke musia byť nulové.

\begin{equation}
 \label{eq:fock1}
 \hat{h_a}\phi_i{a}+\sum_{j=1}^2\int{d\vec{r_b}}|\phi_j(b)|^2V_{ab}|\phi_i(a)^2|-\sum_{j=1}^2\int{d\vec{r_b}}\phi_i(b)^*V_{ab}\phi_i(b)\phi_j(a)=\sum_j{\lambda_{ij}\phi_j(a)}\text{.}
\end{equation}

Ľavá strana rovnice sa dá napísať ako operátor pôsobiaci na vlnovú funkciu (Fockov operátor) $\hat{\mathcal{F}}\phi_i$ . Pravá strana je vlastne násobenie maticou $\Lambda$. Keby sme vedeli maticu
diagonalizovať, úloha prejde na hľadanie vlastných hodnôt Fockovho operátora.

Nech $C$ je unitárna transformácia ($CC^\dagger=I$). Taká, že $C\Lambda C^\dagger=\diag(E_1,E_2)$. Fockovu rovnicu vieme jednoducho upraviť na
\begin{equation}
 \label{eq:fock2}
\hat{\mathcal{F}}\phi_i'=E_i\phi_i' \text{,}
\end{equation}
kde $\phi'_i$ je nová báza.

Po rozpísaní $\hat{\mathcal {F}}$ dostaneme Hartree-Fockovu rovnicu \eqref{eq:fock3}, v ktorej už píšeme iba $\phi_i$.




\section{Kubova-Greenwoodova formula}
\label{sec:kubo A}

V tomto dodatku odvodíme Kubovu Greenwoodovu formulu pre kvantovú vodivosť kovu s disorderom. Uvažujme kovovú vzorku. Nech na elektróny vo vzorke pôsobí časovo závislá porucha
\begin{align}
\label{eq:03potential}
V(t)=V\cos \omega t = -eEx \cos \omega t \mathrm{,}
\end{align}
kde $E$ je elektrické pole pôsobiace v smere $x$. Predpokladajme, že porucha začala pôsobiť v čase $t=t_0$.
V neprítomnosti poruchy sú elektrónové stavy opísané stacionárnou Schrodingerovou rovnicou
\begin{align}
\label{eq:03schr}
\hat{H} \Phi_j(\vr)= \epsilon_j\Phi_j(\vr) \mathrm{,}
\end{align}
ktorej Hamiltonián
\begin{align}
\label{eq:03Hamiltonian}
\hat{H}= - \frac{\hbar^2}{2m}\laplace \vr + V_{dis}(\vr) \mathrm{}
\end{align}
obsahuje náhodný potenciál disorderu $V_{dis}(\vr)$. Po zapnutí poruchy \eqref{eq:03potential}
bude vlnová funkcia elektrónu $\Psi(\vr,t)$ v časoch $t>t_0$ opísaná časovo závislou Schrodingerovou rovnicou
\begin{align}
\label{eq:03timeschr}
i\hbar\ \frac{\partial}{\partial t} \Psi(\vr,t) = (\hat{H} + V(t))\Psi(\vr,t)\mathrm{.}
\end{align}
Rovnicu \eqref{eq:03timeschr} riešime pomocou časovo závislej poruchovej teórie. Funkciu $\Psi(\vr,t)$ rozvinieme do stacionárnych stavov $\phi_j(\vr)$:
\begin{align}
\label{eq:03stac}
\Psi(\vr,t)=\sum_j c_{ji}(t)\Phi_j(\vr)e^{-\frac{i}{\hbar}\epsilon_jt} \mathrm{,}
\end{align}
kde $c_{ji}(t)$ je amplitúda pravdepodobnosti, že elektrón, ktoý bol v čase $t=t_0$ v stacionárnom stave $\phi_i(\vr)$, sa v čase $t=t_0$ bude nachádzať v stave $\phi_j(\vr)$. V čase $t=t_0$ teda platí
\begin{align}
\label{eq:03cji0}
c_{ji}(t_0)=\delta_{ji}
\end{align}
Rozvoj \eqref{eq:03stac} dosadíme do rovnice \eqref{eq:03timeschr}, obe strany rovnice prenásobíme $\Phi^{\ast}_f(\vr)e^{\frac{i}{\hbar}\epsilon_f t}$, kde $\Phi_f(\vr)$ a $\epsilon_f$ sú vlnová funkcia a energia finálneho stavu $f$, a integrujeme cez normalizačný objem $\Omega$.
Dostaneme
\begin{align}
\label{eq:03timeschr_expanded}
i\hbar\frac{\partial}{\partial t}c_{fi}(t)=\sum_j c_{ji}(t) V_{fj}(t)e^{-\frac{i }{\hbar}\epsilon_{fj} t} \mathrm{,}
\end{align}
kde $\epsilon_{fj} \equiv \epsilon_f - \epsilon_j$ a
\begin{equation}
V_{fj}(t) \equiv \int_{\Omega}d\vr \Phi_f^{\ast}(\vr)V(t)\Phi_j(\vr)  \text{.}
\end{equation}
V prvom ráde poruchovej teórie robíme na pravej strane rovnice  \eqref{eq:03timeschr_expanded} aproximáciu $c_{ji}(t) \simeq c_{ji}(t_0) = \delta_{ji}$,
takže rovnica \eqref{eq:03timeschr_expanded} nadobudne tvar
\begin{align}
\label{eq:03born_appr}
i\hbar\frac{\partial}{\partial t}c_{fi}(t)=V_{fi}(t)e^{-\frac{i}{\hbar}\epsilon_{fi} t}\mathrm{.}
\end{align}
Integrovaním cez čas dostaneme
\begin{align}
\label{eq:03cfi}
c_{fi}(t) = \frac{1}{i\hbar} \int_0^t dt' V_{fi}(t')e^{\frac{-i}{\hbar}\epsilon_{fi} t'} \text{.}
\end{align}.
Maticový element $V_{fi}(t)$ prepíšeme ako $V_{fi}(t)=V_{fi}(e^{i\omega t}+e^{-i\omega t})$, kde
\begin{equation}
\label{eq:03Vfi}
V_{fi} \equiv \int_{\Omega} d\vr \Phi_f^{*}(\vr)(\frac{-eEx}{2})\Phi_i(\vr) \text{.}
\end{equation}
Po integrácii cez čas dostaneme
\begin{align}
\label{eq:03cfi_final}
c_{fi}(t)=\frac{1}{i\hbar}V_{fi}\left[\frac{e^{\frac{i}{\hbar}(\epsilon_{fi} - \hbar\omega)t}-1}{\frac{i}{\hbar}(\epsilon_{fi}-\hbar\omega)}+\frac{e^{\frac{i}{\hbar}(\epsilon_{fi} + \hbar\omega)t}-1}{\frac{i}{\hbar}(\epsilon_{fi}+\hbar\omega)}\right] \text{.}
\end{align}
Prenásobením \eqref{eq:03cfi_final} komplexne združeným $c_{fi}(t)$  dostaneme
\begin{align}
\label{eq:03prob}
|c_{fi}(t)|^2=c_{fi}(t)c_{fi}^*(t)
&
=\mid V_{fi}\mid^{2}\ \frac{t^{2}}{\hbar^{2}}
\left[
\sinc^{2}\left\lbrace\frac{(\epsilon_{fi}-\hbar\omega)t}{2\hbar}\right\rbrace
+\sinc^{2}\left\lbrace\frac{(\epsilon_{fi}+\hbar\omega)t}{2\hbar}\right\rbrace
\right.
\\
&
\left.
+2
\cos(\omega t)
 \sinc\left\lbrace\frac{(\epsilon_{fi}-\hbar\omega)t}{2\hbar}\right\rbrace
 \sinc\left\lbrace\frac{(\epsilon_{fi}-\hbar\omega)t}{2\hbar}\right\rbrace
\right]
\text{,} \notag
\end{align}
kde sme zaviedli označenie
\begin{align}
\sinc(x)=\frac{\sin(x)}{x}\text{.}
\end{align}
Hľadáme pravdepodobnosť prechodu za jednotku času z $f$ do $i$, $\frac{|c_{fi}(t)|^2}{t}$, v limite dlhých časov,
\begin{align}
\label{eq:03wfi}
W_{fi}=\lim_{t\to \infty} \frac{|c_{fi}(t)|^2}{t} \text{.}
\end{align}
Keď dosadíme \eqref{eq:03prob} do  \eqref{eq:03wfi}, získame rovnicu, ktorá má na prave strane tri členy.
Tretí člen je v limite $t\to \infty$ nulový, prvé dva členy upravíme pomocou vzťahu
\begin{align}
t^{2}\ \sinc^{2}\left\lbrace\frac{(\epsilon_{fi}\pm \hbar\omega)t}{2\hbar}\right\rbrace
\to
2\pi\hbar t \ \delta(\epsilon_{fi}\pm \hbar\omega)
\text{.}
\end{align}
Dostávame výsledok
\begin{equation}
W_{fi}=
\frac{2\pi}{\hbar}|V_{fi}|^2[\delta(\epsilon_{i}-\epsilon_{i}-\hbar\omega)
+\delta(\epsilon_{f}-\epsilon_{i}+\hbar\omega)]\\
\end{equation}
kde
\begin{equation}
W_{fi}^{\mathrm{ABS}} =\frac{2\pi}{\hbar}|V_{fi}|^2\delta(\epsilon_{f}-\epsilon_{i}-\hbar\omega)
\end{equation}
je pravdepodobnosť prechodu z $f$ do $i$ absorpciou kvanta $\hbar\omega$ a
\begin{equation}
W_{fi}^{\mathrm{EMIS}}=\frac{2\pi}{\hbar}|V_{fi}|^2\delta(\epsilon_{f}-\epsilon_{fi}+\hbar\omega)\text{}
\end{equation}
je pravdepodobnosť prechodu z $f$ do $i$ emisiou kvanta $\hbar\omega$.

Celkový výkon $A$, ktorý od striedavého elektrického poľa získajú za jednotku času všetky  elektróny vo vzorke s objemom $\Omega$,
môžeme vyjadriť kvantovomechanicky vzťahom
\begin{align}
\label{eq:03pw_quant}
A=2\left[
\sum_{f,i}\hbar\omega W_{fi}^{\mathrm{ABS}}f(\epsilon_i)(1-f(\epsilon_f))-\sum_{f,i}\hbar\omega W_{fi}^{\mathrm{EMIS}}f(\epsilon_i)(1-f(\epsilon_f))
\right] \
\end{align}
kde $f(\epsilon_i)$ sú Fermi-Diracove distribúcie a faktor 2 je kvôli spinu. Ten istý výkon vyjadrený
klasicky je $\Omega\frac{1}{T}\int_0^T\sigma(\omega)E^2\cos^2(\omega t)dt=\Omega\frac{1}{2}\sigma(\omega)E^2$, kde
$\sigma(\omega)$ je merná elektrónová vodivosť, ktorú chceme vypočítať kvantovomechanicky. Môžme tak urobiť s použitím rovnice
\begin{equation}
\label{rovnostvykonov}
\Omega\frac{1}{2}\sigma(\omega)E^2 = A \text{.}
\end{equation}
Urobíme úpravy
\begin{align}
\label{eq:03pw_quant_final}
\notag
A&=\frac{4\pi}{\hbar}
\left[
\sum_{f,i}\hbar\omega|V_{fi}|^2\delta(\epsilon_{fi}-\hbar\omega)f(\epsilon_i)(1-f(\epsilon_f))-\sum_{f,i}\hbar\omega|V_{fi}|^2\delta(\epsilon_{fi}+\hbar\omega) f(\epsilon_i)(1-f(\epsilon_f))
\right]\\ \notag
&=\frac{4\pi}{\hbar}
\left[\sum_{f,i}\hbar\omega|V_{fi}|^2\delta(\epsilon_f-\epsilon_i-\hbar\omega)f(\epsilon_i)(1-f(\epsilon_f))-\sum_{i,f}\hbar\omega|V_{if}|^2\delta(\epsilon_i-\epsilon_f+\hbar\omega) f(\epsilon_f)(1-f(\epsilon_i))
\right]\\
&=\frac{4\pi}{\hbar}\sum_{f,i}\hbar\omega|V_{fi}|^2\delta(\epsilon_f-\epsilon_i-\hbar\omega)(f(\epsilon_i)-f(\epsilon_f))
\end{align}
kde v druhom riadku sme vymenili sčítacie indexy v druhej sume a v treťom riadku sme využili symetriu $|V_{fi}|=|V_{if}|$, a párnosť delta funkcie $\delta(x)=\delta(-x)$.
Upravený vzťah \eqref{eq:03pw_quant_final} dosadíme za $A$ do rovnice \eqref{rovnostvykonov}. Keď ešte dosadíme za maticový element $V_{fi}$  vzťah \eqref{eq:03Vfi}, po malých úpravách dostaneme vzťah pre vodivosť,
\begin{equation}
\label{eq:03sigma}
\sigma(\omega)=\frac{2\pi}{\hbar \Omega}\sum_{f,i}\hbar\omega|v_{fi}|^2\delta(\epsilon_f-\epsilon_i-h\omega)(f(\epsilon_i)-f(\epsilon_f)) \text{,}
\end{equation}
kde
\begin{equation}
\label{eq:03sigma matic}
v_{fi} \equiv \int_{\Omega}d\vr \Phi_f^*(\vr)x\Phi_i(\vr) \text{.}
\end{equation}
Maticový element \eqref{eq:03sigma matic} sa dá prepísať využitím komutačného vzťahu $[x,H]=\frac{i\hbar}{m}\hat{p_x}$ ako
\begin{equation}
v_{fi}=-\frac{\hbar^{2}}{m(\epsilon_f-\epsilon_i)}D_{fi} \text{,}
\end{equation}
kde
\begin{equation}
\label{eq:03matrix_element}
D_{fi} \equiv \int_{\Omega}d\vr\Phi_f^*(\vr)\frac{d}{dx}\Phi_i(\vr) \text{.}
\end{equation}
Vzťah \eqref{eq:03sigma} sa tým upraví na finálny tvar známy ako Kubova-Greenwoodova formula:
\begin{align}
\label{eq:03sigma2}
\sigma(\omega)=
\frac{2\pi}{\hbar}\frac{\hbar^{4}}{m^2}\frac{e^2}{\Omega}
\sum_{f,i}\frac{\hbar\omega}{(\epsilon_f-\epsilon_i)^2}|D_{fi}|^2\delta(\epsilon_f-\epsilon_i-\hbar\omega)(f(\epsilon_i)-f(\epsilon_f))\text{.}
\end{align}



  \section*{Dodatok B}
  \addcontentsline{toc}{section}{Dodatok B}
  \markboth{Dodatok B}{Dadatok B}
  %  \renewcommand{\thesection}{A.\arabic{section}}
\renewcommand{\theequation}{A.\arabic{equation}}
\renewcommand{\thefigure}{A.\arabic{figure}}
\setcounter{equation}{0}
\subsection*{Odvodenie Hartree-Fockovej rovnice z variačného princípu}
Potenciál $U(r)$ si znova rozdelíme na elektrónovú a iónovú časť. Tak isto si pre elektróny zavedieme nové súradnice $\tau$ ktoré v sebe zahŕňajú aj spin.
Keď budeme integrovať cez $d\tau$ myslíme tým integrál cez $dr$ a sumu cez jednotlivé spiny.

Najskôr si určíme strednú hodnotu Hamiltoniánu pre dva elektróny. Hamiltonián sa dá potom napísať ako
\begin{equation}
\label{eq:hm}
\hat H =\hat h_1+ \hat h_2 + V_{12} \text{,}
\end{equation}

kde $\hat h_i$ popisujú kinetickú energiu a potenciál od iónov pre jednotlivé elektróny a $V_{12}$ ich vzájomnú interakciu:
\begin{align}
 \hat h_i&=-\frac{\hbar}{2m}\Delta_{i}+U^{ion}(r_i) \\ \notag
 V_{12} &= \frac {e^2}{4\pi \epsilon_0|\vr-\vrp|} \text{,}
\end{align}

Teraz ideme minimalizovať strednú hodnotu hamiltoniánu:
\begin{equation}
 \label{eq:mvh}
 <H>=\int{d\tau_1d\tau_2\Psi^*(\tau_1,\tau_2)\hat H\Psi(\tau_1,\tau_2)} \text{.}
\end{equation}



Do \eqref{eq:mvh} dosadíme Slatterov determinant pre dve funkcie:
\begin{equation}
 \label{eq:Psi2}
 \Psi(\tau_1,\tau_2)=\frac{1}{\sqrt{2}}({\phi_1(\tau_1)\phi_2(\tau_2)-\phi_2(\tau_1)\phi_1(\tau_2)}) \text{,}
\end{equation}
\begin{equation}
 \label{eq:mvh2}
 <H>=\int{d\tau_1d\tau_2\bigl( \phi_1(\tau_1)\phi_2(\tau_2)-\phi_2(\tau_1)\phi_1(\tau_2)\bigr)^*\hat H(\phi_1(\tau_1)\phi_2(\tau_2)-\phi_2(\tau_1)\phi_1(\tau_2))} \text{.}
\end{equation}
Teraz dosadíme dosadíme \eqref{eq:hm} za $\hat H$ a roznásobíme. Dostaneme 12 členov.
\begin{align}
<H>=&\\ \notag
 &\frac{1}{2}\int{d\tau_1 d\tau_2 \phi^*_1(1)\phi^*_2(2)\hat h_1\phi_1(1)\phi_2(2)}+\\ \notag
 &\frac{1}{2}\int{d\tau_1 d\tau_2 \phi^*_1(1)\phi^*_2(2)\hat h_2\phi_1(1)\phi_2(2)}- \notag
 &\frac{1}{2}\int{d\tau_1 d\tau_2 \phi^*_1(2)\phi^*_2(1)\hat h_1\phi_1(1)\phi_2(2)}-\\ \notag
 &...\\ \notag
  &\frac{1}{2}\int{d\tau_1 d\tau_2 \phi^*_1(1)\phi^*_2(2)V_{12}\phi_1(1)\phi_2(2)}- \notag
  &\frac{1}{2}\int{d\tau_1 d\tau_2 \phi^*_1(2)\phi^*_2(1)V_{12}\phi_1(1)\phi_2(2)}-\\ \notag
  &\frac{1}{2}\int{d\tau_1 d\tau_2 \phi^*_1(1)\phi^*_2(2)V_{12}\phi_1(2)\phi_2(1)}+ \notag
   &\frac{1}{2}\int{d\tau_1 d\tau_2 \phi^*_1(2)\phi^*_2(1)V_{12}\phi_1(2)\phi_2(1)} \notag \text{.}
\end{align}
Integrály obsahujúce $\hat h_1$ a $\hat h_2$ budú rovnaké, pretože sa líšia len zámenou premenných. Členy z $V_{12}$ sa dajú prepísať na sumy
\begin{align}
 \label{eq:sumij}
 &\frac{1}{2} \sum_{i=1}^2{\sum_{i\neq j}^2 \int d\tau_a d\tau_b\phi_i(a)\phi_j(b)V_{ab}\phi_i(a)\phi_j(b)}-\\
 &\frac{1}{2} \sum_{i=1}^2{\sum_{i\neq j}^2 \int d\tau_a d\tau_b\phi_i(a)\phi_j(b)V_{ab}\phi_i(b)\phi_j(a)} \notag \text{.}
\end{align}

V tvare sumy sa potom dá napísať celé $<H>$:
\begin{align}
  \label{eq:hsum}
 &\sum_i^2 \int d\tau_a \phi^*_i(a)\hat h_a \phi_i(a) +\\
 &\frac{1}{2} \sum_{i=1}^2{\sum_{i\neq j}^2 \int d\tau_a d\tau_b\phi_i^*(a)\phi_j^*(b)V_{ab}\phi_i(a)\phi_j(b)}- \notag \\
 &\frac{1}{2} \sum_{i=1}^2{\sum_{i\neq j}^2 \int d\tau_a d\tau_b\phi_i^*(a)\phi_j^*(b)V_{ab}\phi_i(b)\phi_j(a)} \notag \text{.}
\end{align}

Teraz je vhodné sa zbaviť spinovej súradnice. V prvom a v druhom člene \eqref{eq:hsum} dostaneme integrovaním cez spiny 1, lebo tvoria ortogonálnu bázu.
V poslednom člene nám s rovnakých dôvodov   prežijú len paralelné spiny.

\begin{align}
 \label{eq:hsum}
 &\sum_i^2 \int d\vr_a \phi^*_i(a)\hat h_a \phi_i(a) +\\ \notag
 &\frac{1}{2} \sum_{i=1}^2{\sum_{i\neq j}^2 \int d\vr_a d\vr_b\phi_i^*(a)\phi_j^*(b)V_{ab}\phi_i(a)\phi_j(b)}-\\ \notag
 &\frac{1}{2} \sum_{i=1}^2{\sum_{i\neq j;cez || spiny}^2 \int d\vr_a d\vr_b\phi_i^*(a)\phi_j^*(b)V_{ab}\phi_i(b)\phi_j(a)} \text{,} \notag
\end{align}
Tento výsledok možno zovšeobecniť na viac elektrónov.

Teraz ideme nájsť funkcie $\phi_i^*$ ktoré minimalizujú funkcionál $E[\phi^*_i]=<H>$, v Diracovej notácí:
\begin{equation}
 \label{eq:e_func}
 E[\Psi^*]=\expval{H}{\Psi} \text{.}
\end{equation}

V minime funkcionálu musí byť variácia nulová $\delta E[\Psi^*]=0$ , kde
\begin{equation}
 \label{eq:e_var}
 \delta E=\bra{\Psi+\delta \Psi}H\ket{\Psi}-\expval{H}{\Psi} \text{.}
\end{equation}

\newpage
Toto však nieje jediná podmienka pre minimum. Funkcia $\Psi$ musí navyše spĺňať väzobnú podmienku ortogonality:

\begin{equation}
 \label{eq:ortho}
 \bra{\Psi}\ket{\Psi}=1 \text{.}
\end{equation}

Táto podmienka platí aj pre jednotlivé $\phi_i$ a dá sa prepísať do integrálneho tvaru.
\begin{equation}
 \label{eq:ortho_2}
 \int{ d\vr\ \phi^*_j\phi_i }-\delta_{ij}=0 \text{.}
 \end{equation}
Rovnicu \eqref{eq:ortho_2} vynásobíme Lagrangeovým multiplikátorom $\lambda_{ij}$. Potom odpočítame od \ref{eq:e_func} a dostaneme
\begin{equation}
 \label{eq:lagr}
 L[\phi]=\expval{H}{\Psi}-\lambda_{ij}( \int{ d\vr\ \phi^*_j\phi_i }-\delta_{ij})\text{.}
\end{equation}

Lagrangián \eqref{eq:lagr} bude rovný funkcionálu \eqref{eq:e_func} pretože sme od neho odčítali nulový člen. Ak položím variáciu $\delta L$ rovnú nule,
dostanem minimum funkcionálu \eqref{eq:e_func} ktoré navyše spĺňa väzobnú podmienku \eqref{eq:ortho}.

Teraz za $H$ dosadíme hamiltonián \eqref{eq:hm} a za $\Psi$ Slatterov determinant \eqref{eq:Psi2} pre dve funkcie.

Všimnime si, že varírovaním prvého člena sme dostali
\begin{equation}
 \label{eq:var1}
 \sum_i^2 \int d\vr_a \delta\phi^*_i(a)\hat h_a \phi_i(a)\text{,}
\end{equation}
pretože od $L[\Psi+\delta\Psi]$ sa členy s $\phi^*_i+\delta\phi^*_i$ sa nám odčítajú s pôvodnými členmi $L[\Psi]$.

Pri druhom a treťom člene máme roznásobiť $(\phi^*_i+\delta\phi^*_i)(\phi^*_j+\delta\phi^*_j)$. Tu nám prežijú len dva členy $\phi^*_i\delta\phi^*_j + \phi^*_i\delta\phi^*_j$
pretože člen s dvoma deltami je druhého rádu, čo pri variácii neuvažujeme, a člen bez delty sa znova odčíta s $L[\Psi]$. Teda suma bude vyzerať nasledovne
\begin{equation}
 \label{eq:part_sum}
 \frac{1}{2} \sum_{i=1}^2{\sum_{i\neq j}^2 \int d\vr_a d\vr_b\delta\phi^*_i(a)\phi^*_j(b)Vab\phi_i(a)\phi_j(b)+ \int d\vr_a d\vr_b\phi^*_i(a)\delta\phi^*_j(b)Vab\phi_i(a)\phi_j(b)} \text{.}
\end{equation}


Vidíme však, že tieto dva členy sa líšia len integračnými premennými, preto ich môžem napísať ako jeden, čím sa faktor $\frac{1}{2}$ stratí. Veľmi podobne viem zredukovať tretí člen.
Posledne po odčítaní $L[\Psi]$ časti nám člen s Lagrangeovým multiplikátorom prejde na
\begin{equation}
 \label{eq:part_lagr}
 \sum_j{\lambda_{ij}\phi_j(a)} \text{.}
\end{equation}

Celková variácia lagrangiánu $L[\Psi]$ bude
\begin{align}
 \label{eq:var_final}
 &\delta L=\int d\vr_a\delta\phi_i^*(a) \\ \notag
 &\bigl(\sum_i\hat h_a \phi_i(a) +
 \sum_{j\neq i} |\phi_j(a)|^2V_{ab}\phi_i(a)+
 \sum_{j \neq i || sp}\int d\vr_b \phi_i(b)^*V_{ab}\phi_i(b)\phi_j(a)-
 \sum_j \lambda_{ij} \phi_j(a)\bigr)=0 \text{.}
\end{align}

%\subsection{Hartree-Fockove rovnice}
Z nulovosti integrálu vyplýva nulovosť podinitegrálnej funkcie
Rovnosť \eqref{eq:var_final} musí platiť pre ľubovoľné malé $\delta \phi_i^*(a)$, teda
členy v zátvorke musia byť nulové.

\begin{equation}
 \label{eq:fock1}
 \hat{h_a}\phi_i{a}+\sum_{j=1}^2\int{d\vec{r_b}}|\phi_j(b)|^2V_{ab}|\phi_i(a)^2|-\sum_{j=1}^2\int{d\vec{r_b}}\phi_i(b)^*V_{ab}\phi_i(b)\phi_j(a)=\sum_j{\lambda_{ij}\phi_j(a)}\text{.}
\end{equation}

Ľavá strana rovnice sa dá napísať ako operátor pôsobiaci na vlnovú funkciu (Fockov operátor) $\hat{\mathcal{F}}\phi_i$ . Pravá strana je vlastne násobenie maticou $\Lambda$. Keby sme vedeli maticu
diagonalizovať, úloha prejde na hľadanie vlastných hodnôt Fockovho operátora.

Nech $C$ je unitárna transformácia ($CC^\dagger=I$). Taká, že $C\Lambda C^\dagger=\diag(E_1,E_2)$. Fockovu rovnicu vieme jednoducho upraviť na
\begin{equation}
 \label{eq:fock2}
\hat{\mathcal{F}}\phi_i'=E_i\phi_i' \text{,}
\end{equation}
kde $\phi'_i$ je nová báza.

Po rozpísaní $\hat{\mathcal {F}}$ dostaneme Hartree-Fockovu rovnicu \eqref{eq:fock3}, v ktorej už píšeme iba $\phi_i$.




\section{Od kvantovej vodivosti Kuba Greenwooda ku klasickej Drudeho vodivosti}
\label{sec:kubo B}


V tomto dodatku odvodíme z Kubovej-Greenwoodovej formuly \eqref{eq:03sigma2} klasickú Drudeho vodivosť.
 Odvodenie robíme podľa prednášok školiteľa a pôvodnej práce  Thoulessa \cite{Thouless}.
 Potrebujeme vypočítať maticový elemnt $|D_{fi}|^2$, na ktorý  potrebujeme vlnové funkcie $\Phi_i(\vr)$.
 Preto musíme riešiť Schr\"odingerovu rovnicu pre elektrón v kove s disorderom
\begin{align}
\label{eq:03SchrDis}
(\frac{-\hbar^2}{2m}\laplace_{\vr} + V_{dis}(\vr))\Phi_i(\vr)=\mathcal{E}_i\Phi_i(\vr)  \text{.}
\end{align}
Urobíme rozvoj do rovinných vĺn
\begin{equation}
\label{eq:03Phi_i}
\Phi_i(\vr)=\sum_{\vk}a^i_{\vk}\phi_{\vk}(\vr)\text{,}
\end{equation}
kde $\phi_{\vk}(\vr)=\frac{1}{\sqrt \Omega}e^{i\vk\vr}$.
Dosadíme \eqref{eq:03Phi_i} do \eqref{eq:03matrix_element} a počítame modul
\begin{align}
\label{eq:03dfisquared}
|D_{fi}|^2=\int_\Omega d\vr \sum_{\vk_1}a_{\vk_1}^f\phi_{\vk_1}(\vr)\frac{d}{dx}\sum_{\vk_2}a_{\vk_2}^{i*}\phi_{\vk_2}^*(\vr)\int_\Omega d\vr\ ' \sum_{\vk_3}a_{\vk_3}^{f*}\phi_{\vk_3}^*(\vr\ ')\frac{d}{dx'}\sum_{\vk_4}a_{\vk_4}^i \phi_{\vk_4}(\vr\ ')
\end{align}
Využijeme vzťahy $\frac{d}{dx}\phi^{\ast}_{\vk_2}(\vr)=-ik_{2x}\phi^{\ast}_{\vk_2}(\vr)$ a $\frac{d}{dx'}\phi_{\vk_4}(\vr\ ')=ik_{4x}\phi_{\vk_4}(\vr\ ')$, a potom vzťahy
$\int_\Omega d\vr \phi_{\vk_1}(\vr)\phi^*_{\vk_2}(\vr)=\delta_{\vk_1\vk_2}$ a $\int_\Omega d\vr\ '\phi_{\vk_3}^*(\vr\ ')\phi_{\vk_4}(\vr\ ')=\delta_{\vk_3\vk_4}$
Vysumujeme pomocou Kroneckerovych symbolov a dostaneme
\begin{align}
|D_{fi}|^2=\sum_{\vk\vk\ '} a^{f*}_{\vk\ '}a^f_{\vk}a^{*i}_{\vk}a^i_{\vk\ '}k_xk_x' \text{,}
\end{align}
kde sme preznačili sumačný index $\vk_1$ na $\vk$ a sumačný index  $\vk_2$ na $\vk\ '$.
Vystredovaním cez súbor makroskopicky rovnakých vzoriek s makroskopicky rôznym disorderom dostaneme
\begin{align}
\label{eq:03dfisq2}
\overline{|D_{fi}|^2}=\sum_{\vk\vk\ '}\overline{a^{f*}_{\vk\ '}a^f_{\vk}a^{*i}_{\vk}a^i_{\vk\ '}}k_xk_x' \text{.}
\end{align}

Predpokladajúc nekorelovanosť stavov $i$ a $f$ môžeme urobiť zjednodušenie
\begin{align}
\overline{|D_{fi}|^2}=\sum_{\vk\vk\ '}
\
\overline{a^{f*}_{\vk\ '}a^f_{\vk}}
\
\overline{a^{*i}_{\vk}a^i_{\vk\ '}}k_xk_x'
\end{align}
Nakoniec predpokladáme, že aj stavy $\vk$ a $\vk\ '$ sú nekorelované, t.j.
\begin{align}
\label{eq:03dfisq3}
\overline{|D_{fi}|^2}=\sum_{\vk\vk\ '}
\ \overline{a^{f*}_{\vk\ '}a^f_{\vk}}
\ \overline{a^{*i}_{\vk}a^i_{\vk\ '}}k_xk_x'
\ \delta_{kk'}=
\sum_{\vk}\ \overline{a^{f*}_{\vk}a^f_{\vk}}
\ \overline{a^{*i}_{\vk}a^i_{\vk}}k_x^2
\ \ \text{.}
\end{align}
Thouless \cite{Thouless} prijal pre  $\overline{a_{\vk}^{i*}a_{\vk}^i}$  ansatz v tvare Lorentziánu
\begin{align}
\label{eq:03lorenz}
\overline{a_{\vk}^{i*}a_{\vk}^i}=\frac{1}{\pi\rho(\epsilon_i)}\lorenz{\epsilon_i}\text{,}
\end{align}
kde $\tau$ je stredná doba medzi dvomi elektrónovými zrážkami s disorderom.
V nasledujúcom dodatku ukážeme, že vzťah \eqref{eq:03lorenz} je výsledkom tzv. self-konzistentnej Bornovej aproximácie. Teraz vzťah \eqref{eq:03lorenz} prijmeme ako je a dosadíme ho do \eqref{eq:03sigma2}. Dostaneme
\begin{align}
\sigma(\omega)=&
\frac{2\pi}{\hbar}\frac{\hbar^{4}}{m^2}\frac{e^2}{\Omega}
\sum_{f,i}\frac{\hbar\omega}{(\epsilon_f-\epsilon_i)}\delta(\epsilon_f-\epsilon_i-\hbar\omega)(f(\epsilon_i)-f(\epsilon_f))\\ \notag
&\sum_{\vk}\frac{1}{\pi\rho(\epsilon_f)}\lorenz{\epsilon_f}\frac{1}{\pi\rho(\epsilon_i)}\lorenz{\epsilon_i}k_x^2
\end{align}
Prejdeme od sumy cez $\vk$ k integrálu
\begin{align}
\label{eq:03sigma3}
\sigma(\omega)=&
\frac{2\pi}{\hbar}\frac{\hbar^{4}}{m^2}\frac{e^2}{\Omega}
\sum_{f,i}\frac{\hbar\omega}{(\epsilon_f-\epsilon_i)}\delta(\epsilon_f-\epsilon_i-\hbar\omega)(f(\epsilon_i)-f(\epsilon_f))\\ \notag
&\int_{\Omega}d\vk\frac{1}{\pi\rho(\epsilon_f)}\lorenz{\epsilon_f}\frac{1}{\pi\rho(\epsilon_i)}\lorenz{\epsilon_i}k_x^2\text{,}
\end{align}
 Kvôli prehladnosti sa ideme venovať len integrálu cez $\vk$. Prejdeme v ňom do sférických súradníc a máme
\begin{align}
\int_0^{2\pi}d\phi\int_0^\pi d\theta \frac{\sin(\theta)cos^2(\theta)\sin^2(\theta)}{4\pi} \frac{4\pi\Omega}{(2\pi)^{3}}
\int_0^\infty dk k^2 \frac{k^{2}}{\pi^{2}\rho(\epsilon_f)}\rho(\epsilon_i)
\lorenz{\epsilon_f} \lorenz{\epsilon_i}
\end{align}
Výsledok prvých dvoch integrálov je $\frac{1}{3}$, v treťom integráli prejdeme do energetických súradníc. Dostaneme
\begin{align}
\frac{\Omega}{3\pi^2\rho(\epsilon_f)\rho(\epsilon_i)}\int_0^\infty d\epsilon_k \rho^{\frac{1}{2}}(\epsilon_k)k(\epsilon_k)\rho^{\frac{1}{2}}(\epsilon_k)k(\epsilon_k)\lorenz{\epsilon_f}\lorenz{\epsilon_i}\text{.}
\end{align}
Využitím symetrickej aproximácie $\rho(\epsilon_k)^{\frac{1}{2}}k(\epsilon_k)\approx\rho(\epsilon_i)^{\frac{1}{2}}k(\epsilon_i)\approx\rho(\epsilon_f)^{\frac{1}{2}}k(\epsilon_f)$ sa zostávajúci integrál zjednoduší
a dá sa dopočítať nasledovne:
\begin{align}
\label{eq:03integral}
\frac{\Omega}{3}\frac{\rho(\epsilon_f)^{\frac{1}{2}}k(\epsilon_f)\rho(\epsilon_i)^{\frac{1}{2}}k(\epsilon_i)}{\pi^{2}\rho(\epsilon_f)\rho(\epsilon_i)}\int_0^\infty d\epsilon_k\lorenz{\epsilon_f}\lorenz{\epsilon_i}\approx\\ \notag
\frac{\Omega}{3}\frac{\rho(\epsilon_f)^{\frac{1}{2}}k(\epsilon_f)\rho(\epsilon_i)^{\frac{1}{2}}k(\epsilon_i)(\frac{\hbar}{2\tau})^2}{\pi^{2}\rho(\epsilon_f)\rho(\epsilon_i)}\frac{4\pi(\frac{\tau}{\hbar})^3}{1+\tau^2(\frac{\epsilon_f-\epsilon_i}{\hbar})^2}\text{.}
\end{align}
Posledný výsledok platí za  predpokladov
\begin{equation}
\frac{\epsilon_i}{\frac{\hbar}{\tau}} >> 1 \  \  , \ \ \   \frac{\epsilon_f}{\frac{\hbar}{\tau}} >> 1  \text{.}
\end{equation}
Vráťme sa k rovnici pre vodivosť \eqref{eq:03sigma3}
a za integrál cez $\vk$ dosaďme výsledok \eqref{eq:03integral}. Dostaneme výsledok
\begin{align}
\notag
\sigma(\omega)&=
2\frac{e^{2}\tau}{3}\frac{\hbar^{2}}{m^{2}}
\\
&\sum_{f,i}\frac{\rho(\epsilon_f)^{\frac{1}{2}}k(\epsilon_f)\rho(\epsilon_i)^{\frac{1}{2}}k(\epsilon_i)}{\rho(\epsilon_f)\rho(\epsilon_i)}
\frac{\hbar\omega}{(\epsilon_f-\epsilon_i)^{2}}\delta(\epsilon_f-\epsilon_i-\hbar\omega)(f(\epsilon_i)-f(\epsilon_f)) \frac{1}{1+\tau^2(\frac{\epsilon_f-\epsilon_i}{\hbar})^2}
\text{.}
\end{align}
Dvojitú sumu  $\sum_{f,i}$ zameníme za dvojitý integrál $\int_0^\infty d\epsilon_i\rho(\epsilon_i) \int_0^\infty d\epsilon_f\rho(\epsilon_f)$. Jeden z integrálov môžeme zintegrovať hneď vďaka
$\delta$-funkcii. Dostávame
\begin{align}
\sigma(\omega)=
\frac{e^{2}\tau}{3}\frac{\hbar^{2}}{m^{2}}
\frac{2}{1+\tau^2\omega^2}\
\frac{1}{\hbar\omega}
\int_0^\infty d\epsilon_i\rho^{\frac{1}{2}}(\epsilon_i)k(\epsilon_i)\rho^{\frac{1}{2}}(\epsilon_i+\hbar\omega)k(\epsilon_i+\hbar\omega)(f(\epsilon_i)-f(\epsilon_i+\hbar\omega))\text{.}
\end{align}
V limite nulovej teploty sa Fermi-Diracove rozdelenia $f(\epsilon_i)$ a $f(\epsilon_i+\hbar\omega)$ zmenia na skokové $\Theta$ funkcie. Vďaka tomu sa posledná rovnica dá upraviť na tvar
\begin{equation}
\label{eq:03kubo}
\sigma(\omega)=
e^2\ 2\rho(E_F)\ \frac{\hbar^{2}k_{F}^{2}}{m^{2}}\ \frac{\tau}{3}\
\frac{1}{1+\tau^2\omega^2} F(\omega)
\end{equation}
kde
\begin{equation}
\label{eq:Fomega}
F(\omega)\equiv\frac{1}{\hbar\omega}\int_{E_F-\hbar\omega}^{E_F}d\epsilon_i \frac{\rho(\epsilon_i)^{\frac{1}{2}}k(\epsilon_i)\rho(\epsilon_i+\hbar\omega)^{\frac{1}{2}}k(\epsilon_i+\hbar\omega)}{\rho(E_F)k_F^2}
\end{equation}
Zadefinujeme $N(E_F)=2\rho(E_F)$, kde $N(E_F)$ je hustota stavov pre obidve orientácie spinu. Ďalej,
spomenieme si, že
\begin{align}
D(E_F)=\frac{1}{3}v_F^2\tau = \frac{\hbar^{2}k_{F}^{2}}{m^{2}}\ \frac{\tau}{3}
\end{align}
je elektrónový difúzny koeficient. A nakoniec, položíme $F(\omega) \simeq 1$, čo nemusí byť v poriadku pre veľké frekvencie. Dostávame klasickú elektrónovú vodivosť Drudeho - Lorentza:
\begin{equation}
\label{eq:03kubofinal}
\sigma(\omega)=
e^2\ N(E_F) D(E_F)\ \
\frac{1}{1+\tau^2\omega^2} \text{.}
\end{equation}




%%%%koniec dodatok  A   %%%%%%




 \section*{Dodatok C}
 \addcontentsline{toc}{section}{Dodatok C}
 \markboth{Dodatok C}{Dadatok C}

%%%%%%NOVY THOULES
\section{Vlnová funkcia elektrónu v prímesnom disorderi: Self-konzistentná Bornova aproximácia}
\label{sec:thouless}


Hľadáme riešenie stacionárnej Schrodingerovej rovnice
\begin{equation}
\label{eq:04SchrDis2stac}
\left(\frac{-\hbar^2}{2m}\laplace_{\vr} + V_{dis}(\vr)\right)\Phi_i(\vr) = \E_i \Phi_{i}(\vr)
  \text{,}
\end{equation}
kde $V_{dis}(\vr)$ je náhodný potenciál, ktorým na elektrón pôsobí primesový disorder. Pre jednoduchosť prijmeme jednoduchý model bodových porúch,
\begin{align}
\label{eq:04pointdis}
V_{dis}(\vr)=\sum_{j=1}^{N_{imp}}\gamma \delta(\vr-\vec R_j^{imp})\text{,}
\end{align}
kde $N_{imp}$ je počet bodových porúch v kryštáli s objemom $\Omega$, a $\vec R^{imp}_j$ sú náhodné polohy bodových porúch.
Za predpokladu, že disorder je slabý, môžeme vyskúšať riešenie v najnižšom ráde stacionárnej poruchovej teórie,
\begin{align}
\label{nulta}
\Phi_i(\vr)=\phi_{\vk_i}(\vr)+\sum_{\vk} a^i_{\vk} \phi_{\vk}(\vr)\text{,}
\end{align}
kde
\begin{equation}
\label{eq:04SchrDis2stac}
a^i_{\vk} = \frac{V_{\vk\vk_i}}{\epsilon_{\vk_i}-\epsilon_{\vk}}
\end{equation}
je sú koeficienty poruchového rozvoja. Potom
\begin{align}
\label{eq:04coef}
\overline{a^{*i}_{\vk}a^i_{\vk}}=
\frac{\overline{|V_{\vk\vk_i}|^2}}{(\epsilon_{\vk_i}-\epsilon_{\vk})^2}
\ \text{.}
\end{align}
Na chvíľku teraz odbočme a uvažujme Boltzmannovu kinetickú rovnicu v aproximácí relaxačného času,
\begin{align}
\dot{\vk}\nabla_{\vk}f(\vk)=-\frac{f(\vk)-f_0(\vk)}{\tau_{\vk}}
\end{align}
kde $f_0(\vk)$ je Fermi-Diracova distribúcia a $f(\vk)$ je nerovnovážna distribučná funkcia. Relaxačný čas  $\tau_{\vk}$ je definovaný ako
\begin{align}
\label{eq:04goldenrule}
\frac{1}{\tau_{\vk_i}}=\sum_{\vk}W_{\vk_i\vk}(1-\frac{k_x}{k_{ix}}) = \sum_{\vk}\frac{2\pi}{\hbar}|\bra{\vk}V_{dis}(\vr)\ket{\vk_i}|^2\delta(\epsilon_{\vk_i}-\epsilon_{\vk})(1-\frac{k_x}{k_{xi}}) \text{,}
\end{align}
kde $|\bra{\vk}V_{dis}(\vr)\ket{\vk_i}|^2=|V_{\vk\vk_i}|^2$,
 stavy $\ket{\vk}$ a $\ket{\vk_i}$ sú rovinné vlny  a  $\vk_i$ je iniciálny stav.
Do \eqref{eq:04goldenrule} dosadíme náhodný potenciál \eqref{eq:04pointdis}
Počítame maticový element
\begin{align}
\label{eq:04vdisElement}
\notag
\bra{\vk}V_{dis}(\vr)\ket{\vk_i}&=\sum_{j=1}^{N_{imp}}\gamma\bra{\vk}\delta(\vr-\vec R^{imp}_j)\ket{\vk_i}\\
\notag
&=\frac{1}{\Omega^2}\sum_{j=1}^{N_{imp}}\gamma\int_{\Omega}d\vr\delta(\vr -\vec R_j^{imp})e^{i(\vk-\vk_i)\vec{r}}\\
&=\frac{1}{\Omega^2}\sum_{j=1}^{N_{imp}}\gamma e^{i(\vk-\vk_i)R_j^{imp}} \text{,}
\end{align}
 Keď výsledok \eqref{eq:04vdisElement} dosadíme do \eqref{eq:04goldenrule}, máme
\begin{align}
\label{eq:04fgr_expanded}
\frac{1}{\tau_{\vk_i}} = \frac{2\pi}{\hbar} \frac{\gamma^2}{\Omega^2}
\sum_{\vk}\ \sum_{j=1}^{N_{imp}}\ \sum_{j'=1}^{N_{imp}}e^{i(\vk-\vk_i)(\vec R^{imp}_j-\vec R^{imp}_{j'})}\delta(\epsilon_{k_i}-\epsilon_{\vk})(1-\frac{k_x}{k_{xi}}) \text{.}
\end{align}
V \eqref{eq:04fgr_expanded} napíšeme osobitne sumu pre členy, kde $j=j'$. V tejto sume dostaneme v jednotlivých členoch v exponente nulu, teda každý člen je rovný jednej.
Po vysumovaní týchto členov dostaneme $N_{imp}$ a máme
\begin{align}
\label{eq:04fgr_expanded2}
\frac{1}{\tau_{\vk_i}} = \frac{2\pi}{\hbar} \frac{\gamma^2}{\Omega^2}
\left[
N_{imp}+\sum_{j\neq j'=1}^{N_{imp}}\ \sum_{j'=1}^{N_{imp}} \
e^{i(\vk-\vk_i)(\vec R^{imp}_j-\vec R^{imp}_{j'})}
\right]
\delta(\epsilon_{k_i}-\epsilon_{\vk})(1-\frac{k_x}{k_{ix}}) \text{.}
\end{align}
Druhý člen v hranatej zátvorky rovnice
\eqref{eq:04fgr_expanded2}
 môžeme interpretovať ako náhodnú chôdzu v komplexnom priestore,
 takže po vysčítaní dostaneme
\begin{align}
\label{nahodne spocitane}
\sum_{j\neq j'=1}^{N_{imp}} \ \sum_{j'=1}^{N_{imp}} \
e^{i(\vk-\vk_i)(\vec R^{imp}_j-\vec R^{imp}_{j'})} = \
N_{imp}\  Re\lbrace e^{i\alpha}\rbrace
\ \text{,}
\end{align}
kde $\alpha$ je náhodná fáza.
To po vystredovaní \eqref{nahodne spocitane} cez disorder bude
\begin{align}
\label{nahodne vystredovane}
\overline{N_{imp} Re\lbrace e^{i\alpha}\rbrace}
\ = \ 0
\ \text{,}
\end{align}
Ak teraz  rovnicu  \eqref{eq:04fgr_expanded2} stredujem cez disorder
 a využjem \eqref{nahodne vystredovane}, dostanem
\begin{align}
\label{eq:04fgr_mean}
\frac{1}{\tau_{\vk_i}}= \frac{2\pi}{\hbar} \frac{\gamma^2}{\Omega^2}\
 \sum_{\vk}N_{imp}\delta(\epsilon_{\vk_i}-\epsilon_{\vk})(1-\frac{k_x}{k_{ix}})\text{.}
\end{align}
Teraz vykonáme sumu v \eqref{eq:04fgr_mean}.
Delta funkcia je párna funkcia $k_x$ zatiaľčo člen $\propto k_x$ v zátvorke je nepárny, takže príspevok od tohto člena po vysumovaní cez všetky $\vk$ vymizne.
V prvom člene máme
 $\sum_{\vk}\delta(\epsilon_{\vk_i}-\epsilon_{\vk}) = \Omega\rho(\epsilon_{\vk_i})$, kde $\rho(\epsilon_{\vk_i})$ hustota stavov na jednotku objemu a $\Omega$ je objem.
Konečný výsledok pre relaxačný čas je
\begin{align}
\label{eq:04fgr_final}
\frac{1}{\tau_{\vk_i}}
=\
\frac{2\pi}{\hbar}\gamma^2n_{imp}\rho(\epsilon_{\vk_i})
=\
\frac{2\pi}{\hbar}\rho(\epsilon_{\vk_i})\overline{|V_{\vk\vk_i}|^2}
 \text{,}
\end{align}
kde $n_{imp}=\frac{N_{imp}}{\Omega}$ je hustota bodového disorderu. Odtiaľ
\begin{equation}
\label{maticelement fi}
\overline{|V_{\vk\vk_i}|^2}  = \frac{1}{\pi\rho(\epsilon_{\vk_i})} \frac{\hbar}{2\tau} \text{,}
\end{equation}
kde zanedbávame závislosť $ \tau$ od energie. S využitím posledného vzťahu môžeme vzťah \eqref{eq:04coef} zapísať v tvare
\begin{align}
\label{eq:04divergent}
\overline{a_{\vk}^{i*}a_{\vk}^i}=\frac{1}{\pi\rho(\epsilon_{\vk_i})}
\frac{\frac{\hbar}{2\tau}}{(\epsilon_{\vk_i}-\epsilon_{\vk})^2}
\ \text{.}
\end{align}
Tento výsledok sa od Thoulessovho Lorentziánu z dodatku B,
\begin{align}
\label{eq:04lorenz}
\overline{a_{\vk}^{i*}a_{\vk}^i}=\frac{1}{\pi\rho(\epsilon_i)}
\frac{\frac{\hbar}{2\tau}}
{
(\epsilon_{\vk_i}-\epsilon_{\vk})^2
+ (\frac{\hbar}{2\tau})^2
}
\ \text{,}
\end{align}
líši absenciou člena $(\frac{\hbar}{2\tau})^2$ v menovateli.
Preto pre $\vk$ blízke $\vk_i$  diverguje a nespĺňa normalizačnú podmienku
\begin{align}
\sum_{\vk} \overline{|a_{\vk}^{i*}a_{\vk}^i|^2}=1\text{.}
\end{align}
Aby sme splnili normalizačnú podmienku, korekciu $(\frac{\hbar}{2\tau})^2$ by sme mohli do menovateľa vzťahu \label{eq:04divergent} doplniť rukou.
V nasledujúcom ukážeme, že korekcia $(\frac{\hbar}{2\tau})^2$ má aj svoje fyzikálne opodstatnenie.

Budeme riešiť nestacionárnu Schrodingerovu rovnicu
\begin{equation}
\label{eq:04SchrDis2}
i\hbar \frac{\partial}{\partial t}\Phi_{i}(\vec{r} ,t)
\ = \
\left(\frac{-\hbar^2}{2m}\laplace_{\vr} + V_{dis}(\vr)\right)
\Phi_i(\vec{r} ,t)
  \text{.}
\end{equation}
v ktorej predpokladáme, že potenciál $V_{dis}(\vr)$ začal na elektrón pôsobiť v čase $t =0$. 
Do rovnice \eqref{eq:04SchrDis2} dosadíme rozvoj
\begin{equation}
\label{casovy rozvoj}
\Phi_i(\vr,t)=
\sum_{\vk}a^{\vk_i}_{\vk}(t)\phi_{\vk}(\vr)
\ e^{-\frac{i}{\hbar}\epsilon_{\vk}t}\text{,}
\end{equation}
kde $\phi_{\vk}(\vr)$ a $\epsilon_{\vk}$ sú rovinné vlny a k nim prislúchajuce energie.
Potom rovnicu prenásobíme výrazom $ \phi^{\ast}_{\vk_f}(\vr)
\ e^{\frac{i}{\hbar}\epsilon_{\vk_f}t}$ a preintegrujeme
cez normovací objem. Po úpravách dostaneme
\begin{align}
\label{casova1}
i\hbar \frac{\partial}{\partial t}\ a^{\vk_i}_{\vk_f}(t)=\sum_{\vk}
\ \ a^{\vk_i}_{\vk}(t)e^{-\frac{i}{\hbar}(\epsilon_{\vk}-\epsilon_{\vk_f})t}
\ \int d\vec{r} \phi^{\ast}_{\vec{k_f}}(\vec{r})\ V_{dis}(\vec{r})
\phi_{\vec{k}}(\vec{r})
\ \text{,}
\end{align}
V Bornovej aproximácii
 $a^{\vk_i}_{\vk}(t) \simeq a^{\vk_i}_{\vk}(0) =\delta_{\vec{k},\vec{k_i}}$
 rovnica  \eqref{casova1}
prejde na
\begin{equation}
\label{casova2}
i\hbar \frac{\partial}{\partial t}\ a^{\vk_i}_{\vk_f}=
\ V_{fi}
\ e^{-\frac{i}{\hbar}(\epsilon_{\vk_i}-\epsilon_{\vk_f})t}
\ \text{,}
\end{equation}
kde $V_{fi}\equiv \int d\vec{r} \phi^{\ast}_{\vec{k_f}}(\vec{r})\ V_{dis}(\vec{r})
\phi_{\vec{k_i}}(\vec{r})$.
Preintegrovanim cez čas od $0$ po $t$ dostaneme
\begin{align}
\label{eq:04coef_nonstac}
a^{\vk_i}_{\vk_f}=\frac{V_{fi}}{(\epsilon_i-\epsilon_f)}
\left(
e^{-\frac{i}{\hbar}(\epsilon_i-\epsilon_f)t}-1
\right)
\ \text{.}
\end{align}
V tomto vzťahu spoznávame výsledok stacionárnej poruchovej teórie násobený komplexným fázovým faktorom periodickým v čase.
Aby sme sa dostali ďalej, uvažujme znovu časovo závislú rovnicu \eqref{eq:04SchrDis2}, avšak
poruchu $V_{dis}(\vr)$ nahraďme časovo závislo poruchou
\begin{align}
V_{dis}(\vr,t)=V_{dis}(\vr) \ e^{-\delta t} \text{,}
\end{align}
kde $\delta$ je kladná reálna konštanta a faktor  $e^{-\delta t}$ poruchu s narastajúcim časom tlmí.
Rovnica \eqref{casova2}  sa tým zmení na rovnicu
\begin{equation}
\label{casova3}
i\hbar \frac{\partial}{\partial t}\ a^{\vk_i}_{\vk_f}=
\ V_{fi} e^{-\frac{i}{\hbar}(\epsilon_{\vk_i}-\epsilon_{\vk_f})t-\delta t}
\ \text{,}
\end{equation}
ktorá po integrovaní cez čas dáva
 \begin{align}
\label{eq:04coef_nonstac2}
a^{\vk_i}_{\vk_f} \ = \
\frac{V_{fi}}{\epsilon_i-\epsilon_f-i\hbar\delta}
\ \left(
e^{-\frac{i}{\hbar}(\epsilon_i-\epsilon_f- i\hbar\delta)t}-1
\right)
\ \text{.}
\end{align}
V \eqref{eq:04coef_nonstac2} urobíme limitu $ t \to \infty $   a dostaneme
\begin{equation}
\label{limita t nekonecno}
a^{\vk_i}_{\vk_f} \ = \
\frac{V_{fi}}{\epsilon_f-\epsilon_i+ i\hbar\delta}
\ \text{,}
\end{equation}
a odtiaľ
\begin{equation}
\label{a s hviezdickou a}
{a^{\ast}}^{\vk_i}_{\vk_f} a^{\vk_i}_{\vk_f} \
\ = \
\frac{\mid V_{fi} \mid^{2}}
{(\epsilon_f-\epsilon_i)^{2}+(\hbar\delta)^{2}}
\ \text{.}
\end{equation}
Konečne, posledný vzťah vystredujeme cez disorder
\begin{equation}
\label{a s hviezdickou a vystredovany}
\overline{
{a^{\ast}}^{\vk_i}_{\vk_f} a^{\vk_i}_{\vk_f} \
}
\ = \
\frac{
\overline{\mid V_{fi} \mid^{2}}
}
{(\epsilon_f-\epsilon_i)^{2}+(\hbar\delta)^{2}}
\ \text{.}
\end{equation}
Z formuly \eqref{maticelement fi} vyplýva, že
\begin{equation}
\label{vfi na druhu}
\overline{\mid V_{fi} \mid^{2}}
\ = \
\frac{\hbar} {2\pi \rho(\epsilon_i) \tau}
\ \text{.}
\end{equation}
Keď teda dosadíme  \eqref{vfi na druhu}
do \eqref{a s hviezdickou a vystredovany},
dostaneme
\begin{equation}
\label{a s hviezdickou a final}
\overline{
{a^{\ast}}^{\vk_i}_{\vk_f} a^{\vk_i}_{\vk_f} \
}
\ = \
\frac{1}{\pi \rho(\epsilon_i)} \
\frac{ \frac{\hbar}{2\tau}}
{(\epsilon_f-\epsilon_i)^{2}+(\hbar\delta)^{2}}
\ \text{.}
\end{equation}
Aby sa splnila normalizačná podmienka
\begin{equation}
\label{normalizacia}
\sum_{\vec{k_i}}
\overline{
{a^{\ast}}^{\vk_i}_{\vk_f} a^{\vk_i}_{\vk_f} \
}
\ = \
1
\end{equation}
musí byť splnené $\delta =1/2\tau$
a teda
\begin{equation}
\label{thoules ansatz final}
\overline{
{a^{\ast}}^{\vk_i}_{\vk_f} a^{\vk_i}_{\vk_f} \
}
\ = \
\frac{1}{\pi \rho(\epsilon_i)}
\frac{ \frac{\hbar}{2\tau}}
{(\epsilon_f-\epsilon_i)^{2}+(\frac{\hbar}{2\tau})^{2}}
\ \text{,}
\end{equation}
čo je Lorentzián \eqref{eq:04lorenz}. 


%%%%%%%%%%%%%%%%%%%%%%%%%%%%%%%%
%koniec novy thoules%%%%%%%%%%%%%%%%%%%%%%%%%%%%%%%%%
%%%%%%%%%%%%%%%%%%%%%%%%%%