\thispagestyle{empty}
\section*{Abstrakt v štátnom jazyku}

Jenča, Matúš: Hustota elektróvých stavov v kove so slabým disorderom a slabou elektrón-elektrónovou interakciou: Jav Altshulera - Aronova [Diplomová práca], Univerzita Komenského v Bratislave, Fakulta matematiky, fyziky a informatiky, Katedra experimentálnej fyziky; školiteľ: Doc. RNDr. Martin Moško, DrSc., Bratislava, 2022,  45 s.

Dominantné interakcie vodivostných elektrónov v kovoch pri nízkych teplotách sú interakcia s prímesným disorderom a elektrón-elektrónová tienená coulombovská interakcia. 
Kombinácia týchto dvoch interakcií potláča
hustotu elektrónových stavov v blízkom okolí Fermiho
energie v porovnaní s hustotou stavov v čistom kove. Toto lokálne potlačenie hustoty stavov (pod blízkym okolím Fermiho energie sa má na mysli interval energie daný približne ako súčin Planckovej konštanty a frekvencie elektrónových zrážok s disorderom) predpovedali teoretické práce Altshulera a Aronova a predpoveď bola experimentálne mnoho krát potvrdená pomocou tunelovej
spektroskopie. V posledných dvoch desaťročiach sa podarilo experimentálne pozorovať aj zmeny hustoty stavov mimo
blízkeho okolia Fermiho energie, kde už teória Altshulera-Aronova neplatí a kde teoretici zvyknú hustotu stavov aproximovať konštantnou hustotou stavov čistého kovu.
Niektoré experimenty však ukazujú, že mimo blízke okolie Fermiho energie hustota elektrónových stavov s rastúcou vzdialenosťou od Fermiho energie najprv narastie výrazne nad hodnotu hustoty stavov v čistom kove a až neskôr na túto hodnotu poklesne.
V tejto práci teoreticky študujeme, ako vplýva e-e interakcia a disorder na 
hustotu stavov na tých vzdialenostiach od Fermiho energie, na ktorých už teória Altshulera-Aronova neplatí a používa sa aproximácia konštantnej  hustoty stavov čistého kovu, ktorá 
sa nezdá byť v súlade s nedávnymi experimentami. Altshuler a Aronov predpokladali, že elektrónová vlnová funkcia v disorderi popisuje semiklasickú difúziu na dlhých časoch, čo ich teóriu obmedzilo na blízke okolie Fermiho energie. V našom výpočte je vlnová funkcia elektrónu v disorderi popísaná v selfkonzistentnej Bornovej aproximácii, ktorá naopak platí pre časy kratšie ako elektrónový zrážkový čas a teda pre stavy ďaleko od Fermiho energie. Naviac, neporušenú hustotu stavov v čistom kove neaproximujeme konštantou ale explicitne započítavame Fockovu tienenú e-e interakciu. Získané výsledky porovnávame s experimentom.

\begin{flushleft}
\textbf{Kľúčové slová:} hustota elektrónových stavov, disorder, elektrón-elektrónová interakcia,  jav Altshulera-Aronova
\end{flushleft}