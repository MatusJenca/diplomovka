\section{Odvodenie Kubovej Formuly}
V tejto kapitole odvodíme Kubovu Formulu pre optickú vodivosť, ktorú neskôr použijeme na ďalšie výpočty.
%fakt neviem presne na ake, tuto vetu treba prepisat
Narozdiel od štandartne používanej Boltzmanovej Kinetickej rovnice (BKR), ktorá využiva semiklasický 
formalizmus vlnových balíkov, v Kubovej formule uvažujeme čisto kvantový prístup. V tejto kapitole 
ukážeme, že v istom priblížení výsledok pre vodivosť z Kubovej formuly korešponduje s Drudeho formulou:
\begin{align}
\label{eq:03drude}
\sigma=\frac{ne^2\tau}{m*}\mathrm{,}
\end{align}
kde $n$ je číslo pásu, $\tau$ je relaxačný čas a $m*$ je efektívna hmotnosť. Drudeho formula je odvodená z 
BKR, teda je semiklasická, z Kubovej formuly vieme dostať jej kvantovú analógiu.
Uvažujme disorderovaný kov napojený na zdroj napätia s periodickou časovou závislosťou
\begin{align}
\label{eq:03potential}
V(t)=V\cos \omega t = -eEx \cos \omega t \mathrm{.}
\end{align}

Predpokladajme, že v čase $t=t_0$ je pole nulové. Platí bezčasová \schr
\begin{align}
\label{eq:03schr}
\hat{H} \Phi_j(\vr)= \epsilon_j\Phi_j(\vr) \mathrm{,}
\end{align}
ktorej Hamiltoníán
\begin{align}
\label{eq:03Hamiltonian}
\hat{H}= - \frac{\hbar^2}{2m}\laplace \vr + V_{dis}(\vr) \mathrm{,} 
\end{align}
kde $V_{dis}(\vr)$ je náhodný potenciál disorderu. Pre vlnovú funkciu $\Psi(\vr,t)$ v iných časoch platí. 
\begin{align}
\label{eq:03timeschr}
\frac{\partial}{\partial t} \Psi(\vr,t) = (\hat{H} + V(t))\Psi(\vr,t)\mathrm{.}
\end{align}

Rovnicu \eqref{eq:03timeschr}  riešime pomocou časovej poruchovej teórie. Funkciu $\Psi(\vr,t)$ rozvinieme do stacionárnych stavov $\phi_j(\vr)$. 
\begin{align}
\label{eq:03stac}
\Psi(\vr,t)=\sum_j c_{ji}(t)\Phi_j(\vr)e^{-\frac{i\epsilon_j}{\hbar}} \mathrm{,}
\end{align}
kde $c_{ji}(t)$ je koeficient prechodu, prenásobením komplexne združeným   $c_{ji}*(t)$ dostaneme pravdepodobnosť prechodu $|c_{ji}(t)|^2$ z počiatočného stavu $i$ do nového stavu $j$. Je zrejmé, že v čase $t=t_0$ je tento koeficient rovný Kronekerovmu symbolu
\begin{align}
\label{eq:03cji0}
c_{ji}(t_0)=\delta_{ji}
\end{align}  
%tuna to budem potrebovat lepsie vysvetlit.
V ďalších výpočtoch budeme potrebovať koeficient prechodu medzi počiatočným stavom $i$  a finálnym stavom $f$. Ten dostaneme nasledovným spôsobom:

Rozvoj \eqref{eq:03stac} dosadíme do \eqref{eq:03timeschr} obe strany prenásobíme $\Phi_f(\vr)e^{\frac{i\epsilon_f}{\hbar}}$ - kde $\Phi_f(\vr)$ a $\epsilon_f$ sú vlnová funkcia a energia finálneho stavu $f$ - a integrujeme cez normovaný objem $\Omega$. Po dosadení bude ľavá strana rovnice \eqref{eq:03timeschr} 
\begin{align}
i\hbar\frac{\partial}{\partial t} \sum_j c_{ij}(t)\int_{\Omega} d\vr \Phi_f^{*}(\vr)\Phi_j(\vr)e^{\frac{i(\epsilon_f-\epsilon_j}){\hbar}} \text{,}
\end{align}
kde využijúc ortogonalitu bázy $\{\Phi_j(\vr)\}$ môžeme ľavú stranu vysumovať. Rovnica \eqref{eq:03timeschr} prejde na 
\begin{align}
\label{eq:03timeschr_expanded}
i\hbar\frac{\partial}{\partial t}c_{fi}(t)=\sum_j c_{ji} V_{fi}(t)e^{\frac{-i \epsilon_{fi} t}{\hbar}} \mathrm{,}
\end{align}
kde sme zaviedli nasledovné označenia:
\begin{align}
V_{fi}(t) &\equiv \int_{\Omega}d\vr \Phi_f(\vr)V(t)\Phi_i(\vr) \\
\epsilon_{fi} &\equiv \epsilon_f - \epsilon_i \text{.}
\end{align}

Teraz použijeme Bornovu aproximáciu $c_{ij}(t)=\c_{ij}(t_0)$, čo je podľa \eqref{eq:03cji0} Kronekerov symbol, teda \eqref{eq:03timeschr_expanded} prejde na
\begin{align}
\label{eq:03born_appr}
i\hbar\frac{\partial}{\partial t}c_{fi}(t)=V_{fi}e^{\frac{-i\epsilon_{fi} t}{\hbar}}\mathrm{.}
\end{align}
Túto diferenciálnu rovnicu vieme narozdiel od \eqref{eq:03timeschr} a \eqref{eq:03timeschr_expanded} riešiť jednoducho integrovaním:
\begin{align}
\label{eq:03cfi}
c_{fi}(t) = \frac{1}{i\hbar} \int_0^t dt' V_{fi}(t')e^{\frac{-i\epsilon_{fi} t'}{\hbar}} \text{.}
\end{align}.

Maticový element $V_{fi}(t)$ prepíšeme ako
\begin{align}
V_{fi}(t)&=V_{fi}(e^{i\omega t}+e^{-i\omega t})\\
\label{eq:03Vfi}
V_{fi}&\equiv \int_{\Omega} d\vr \Phi_f^{*}(\vr)(\frac{-eEx}{2})\Phi_i(\vr) \text{.}
\end{align}
V rovinici \eqref{eq:03cfi} vykonáme integrál cez čas dostaneme
\begin{align}
\label{eq:03cfi_final}
c_{fi}(t)=V_{fi}[\frac{e^{\frac{\epsilon_{fi} - \hbar\omega}{\hbar}}-1}{\frac{i}{\hbar}(\epsilon_{fi}-\hbar\omega)}+\frac{e^{\frac{\epsilon_{fi} + \hbar\omega}{\hbar}}-1}{\frac{i}{\hbar}(\epsilon_{fi}+\hbar\omega)}] \text{.}
\end{align}
Dostali sme koeficient $c_{fi}(t)$ rozvoja  stavu $\Psi(t,\vr)$ do ortonormálnej bázy $\{\Phi_j(\vr)\}$ s počiatočnou podmienkou $\Psi(t_0,\vr)=\Phi_i(\vr)$ pre jeden konkrétny vektor $\Phi_f(\vr)$. Modul tohoto koeficientu je pravdepodobnosť prechodu z {\it iniciálneho} stavu $\Phi_i(\vr)$ do {\it finálneho} stavu $\Phi_f(\vr)$.

Prenásobením \eqref{eq:03cfi_final} komplexne združeným dostaneme
\begin{align}
\label{eq:03prob}
|c_{fi}(t)|^2=c_{fi}(t)c_{fi}^*(t)&=[\sinc(\frac{\epsilon_{fi}-\hbar\omega}{\hbar}t)+\sinc(\frac{\epsilon_{fi}+\hbar\omega}{\hbar}t)\\
&+2\cos(\omega t)\sinc(\frac{\epsilon_{fi}-\hbar\omega}{\hbar}t)\sinc(\frac{\epsilon_{fi}+\hbar\omega}{\hbar}t)]\text{,} \notag
\end{align}
kde sme zaviedli označenie 
\begin{align}
\sinc(x)=\frac{\sin(x)}{x}\text{.}
\end{align}

V nasledujúcich výpočtoch budeme potrebovať Bornovskú pravdepodobnosť prechodu za jednotku času $\frac{|c_{fi}(t)|^2}{t}$, ktorá nás zaujíma v limite nekonečného času.
\begin{align}
\label{eq:03wfi}
W_{fi}=\lim_{t\to \infty} \frac{|c_{fi}(t)|^2}{t} \text{.}
\end{align}
Dosadíme \eqref{eq:03prob} do  \eqref{eq:03wfi}. Tretí člen bude v limite nulový, na prvé dva použijeme nasledujúci vzťah:
\begin{align}
\delta(x)=\lim_{t\to \infty}t\ \sinc(tx) \text{.}
\end{align}
Dostávame nasledujúce vzťahy
\begin{align}
W_{fi}&=\frac{2\pi}{\hbar}|V_{fi}|^2[\delta(\epsilon_{fi}-\hbar\omega)+\delta(\epsilon_{fi}+\hbar\omega)]\\
W_{fi}^{\mathrm{ABS}}&=\frac{2\pi}{\hbar}|V_{fi}|^2\delta(\epsilon_{fi}-\hbar\omega)\\
W_{fi}^{\mathrm{EMIS}}&=\frac{2\pi}{\hbar}|V_{fi}|^2\delta(\epsilon_{fi}+\hbar\omega)\text{,}
\end{align}
kde sme zadefinovali pravdepodobnosti zvlášť pre absorbciu a emisiu $W_{fi}^{\mathrm{ABS}}$ a $W_{fi}^{\mathrm{EMIS}}$.

Odvodili sme kvantovú pravdepodobnosť prechodu medzi stavmi, z ktorej vieme určiť prenesený výkon, ako rozdiel absorbovaného a emitovaného výkonu vysumovaný cez všetky počiatočné a konečné stavy
\begin{align}
\label{eq:03pw_quant}
A=2[\sum_{f,i}\hbar\omega W_{fi}^{\mathrm{ABS}}f(\epsilon_i)(1-f(\epsilon_f))-\sum_{f,i}\hbar\omega W_{fi}^{\mathrm{EMIS}}f(\epsilon_i)(1-f(\epsilon_f))] \
\end{align}
Kde $f(\epsilon_i)$ sú Fermi-Diracove distribúcie a faktor 2 je kvôli spinu. Výraz \eqref{eq:03pw_quant} zjednodušíme nasledovnými úpravami
\begin{align}
A&=\frac{4\pi}{\hbar}[\sum_{f,i}\hbar\omega|V_{fi}|^2\delta(\epsilon_{fi}-\hbar\omega)f(\epsilon_i)(1-f(\epsilon_f))-\sum_{f,i}\hbar\omega|V_{fi}|^2\delta(\epsilon_{fi}+\hbar\omega) f(\epsilon_i)(1-f(\epsilon_f))]\\
A&=\frac{4\pi}{\hbar}[\sum_{f,i}\hbar\omega|V_{fi}|^2\delta(\epsilon_f-\epsilon_i-\hbar\omega)f(\epsilon_i)(1-f(\epsilon_f))-\sum_{i,f}\hbar\omega|V_{if}|^2\delta(\epsilon_i-\epsilon_f+\hbar\omega) f(\epsilon_f)(1-f(\epsilon_i))]\\
\label{eq:03pw_quant_final}
A&=\frac{4\pi}{\hbar}\sum_{f,i}\hbar\omega|V_{fi}|^2\delta(\epsilon_f-\epsilon_i-\hbar\omega)(f(\epsilon_i)-f(\epsilon_f))
\end{align}
Kde v druhom riadku sme vymenili sčítacie indexy v druhej sume a v treťom riadku využili symetriu maticového elementu $|V_{fi}|=|V_{if}|$ a  párnosť delta funkcie $\delta(x)=\delta(-x)$. Kvantový vzťah pre prenesený výkon \eqref{eq:03pw_quant_final} porovnám s klasickým 
\begin{align}
A=\frac{1}{T}\int_0^T\sigma(\omega)E^2\cos^2(\omega t)dt=\frac{1}{2}\sigma(\omega)E^2\omega \text{,}
\end{align}

kde $\sigma(\omega)$ je optická vodivosť. Po dosadení za maticový element podľa \eqref{eq:03Vfi} môžme optickú vodivosť vyjadriť ako
\begin{align}
\label{eq:03sigma}
\sigma(\omega)&=\frac{2\pi}{\hbar \Omega}\sum_{f,i}\hbar\omega|v_{fi}|^2\delta(\epsilon_f-\epsilon_i-h\omega)(f(\epsilon_i)-f(\epsilon_f))\\
v_{fi}&\equiv \int_{\Omega}d\vr \Phi_f^*(\vr)x\Phi_i(\vr)
\end{align}
Nový maticový element je lepšie prepísať využitím komutačného vzťahu $[x,H]=\frac{i\hbar}{m}\hat{p_x}$ ako
\begin{align}
v_{fi}&=-\frac{\hbar}{m(\epsilon_f-\epsilon_i)}D_{fi}\\
\label{eq:03matrix_element}
D_{fi}&\equiv \int_{\Omega}d\d\vr\Phi_f^*(\vr)\frac{d}{dx}\Phi_i(\vr)\text{,}
\end{align}
Vzťah \eqref{eq:03sigma} bude teda
\begin{align}
\label{eq:03sigma2}
\sigma(\omega)=\frac{2\pi\hbar e^2}{m^2\Omega} \sum_{f,i}\frac{\hbar\omega}{(\epsilon_f-\epsilon_i)^2}|D_{fi}|^2\delta(\epsilon_f-\epsilon_i-\hbar\omega)(f(\epsilon_i)-f(\epsilon_f))\text{.}
\end{align}

Teraz potrebujeme vypočítať maticový elemnt $|D_{fi}|^2$ na ktorý ale potrebujeme vlnové funkcie $\Phi_i(\vr)$. Preto musíme riešiť Schr\"odingerovu rovnicu pre elektrón v kove s disorderom
\begin{align}
\label{eq:03SchrDis}
(\frac{-\hbar^2}{2m}\laplace_{\vr} + V_{dis}(\vr))\Phi_i(\vr)=\mathcal{E}_i\Phi(\vr)  \text{.}
\end{align}
Túto rovnicu budeme riešiť poruchovou teóriou, neporušenými stavmi budú rovinné vlny
\begin{align}
\label{eq:03Phi_i}
\Phi_i(\vr)=\sum_{\vk}a^i_{\vk}\phi_{\vk}(\vr)\\
\label{eq:03rov}
\phi_{\vk}(\vr)=\frac{1}{\sqrt \Omega}e^{i\vk\vr}\text{.}
\end{align}
Ešte predtým však dosadím rozvoj \eqref{eq:03Phi_i} do maticového elementu \eqref{eq:03matrix_element}. Modul maticového elementu potom bude
\begin{align}
|D_{fi}|^2=\int_\Omega d\vr \sum_{\vk_1}a_{\vk_1}^f\phi_{\vk_1}(\vr)\frac{d}{dx}\sum_{\vk_2}a_{\vk_2}^{i*}\phi_{\vk_2}^*(\vr)\int_\Omega d\vr\ ' \sum_{\vk_3}a_{\vk_3}^{f*}\phi_{\vk_3}^*(\vr\ ')\frac{d}{dx'}\sum_{\vk_4}a_{\vk_4}^i \phi_{\vk_4}(\vr\ ')
\end{align}
Teraz využijeme nasledovné vzťahy:
\begin{align}
\frac{d}{dx}\phi_{\vk_2}(\vr)&=-ik_{2x}\phi_{\vk_2}(\vr)\\
\frac{d}{dx'}\phi_{\vk_4}(\vr\ ')&=-ik_{4x}\phi_{\vk_4}(\vr\ ')\\
\int_\Omega d\vr \phi_{\vk_1}^*(\vr)\phi_{\vk_2}(\vr)&=\delta_{\vk_1\vk_2}\\
\int_\Omega d\vr\ '\phi_{\vk_3}^*(\vr\ ')\phi_{\vk_4}(\vr\ ')&=\delta_{\vk_3\vk_4} \text{.}
\end{align}
Vysumujeme cez Kroneckerove symboly, a dostaneme
\begin{align}
|D_{fi}|^2=\sum_{\vk\vk\ '} a^{f*}_{\vk\ '}a^f_{\vk}a^{*i}_{\vk}a^i_{\vk\ '}k_xk_x' \text{,}
\end{align}
kde sme preznačili sumačné indexy $\vk\equiv \vk_1$ a $\vk\ '\equiv \vk_2$.

Doteraz sme v tejto kapitole prezentovali presné výsledky. Teraz urobíme prvé aproximácie. Maticový element $|D_{fi}|^2$ je pre jeden konkrétny disorder, teda jedno náhodné usporiadanie porúch v kryštáli. V reálnom prípade nás zaujíma maticový element pre {\it stredný disorder}, ktorý dostaneme ako strednú hodnotu všetkých možných disorderov.
\begin{align}
\overline{|D_{fi}|^2}=\sum_{\vk\vk\ '}\overline{a^{f*}_{\vk\ '}a^f_{\vk}a^{*i}_{\vk}a^i_{\vk\ '}}k_xk_x' \text{.} 
\end{align}

Predpokladáme, že disorder je {\it slabý}, preto môžme urobiť aj druhú aproximáciu, ktorá predpokladá nekorelovanosť stavov $i$ a $f$, preto môžme písať.
\begin{align}
\overline{|D_{fi}|^2}=\sum_{\vk\vk\ '}\overline{a^{f*}_{\vk\ '}a^f_{\vk}}\overline{a^{*i}_{\vk}a^i_{\vk\ '}}k_xk_x' 
\end{align}
Nakoniec predpokladáme, že stavy $\vk$ a $\vk\ '$ sú nekorelované, teda
\begin{align}
\overline{|D_{fi}|^2}=\sum_{\vk\vk\ '}\overline{a^{f*}_{\vk\ '}a^f_{\vk}}\overline{a^{*i}_{\vk}a^i_{\vk\ '}}k_xk_x'\delta_{kk'}=\sum_{\vk}\overline{a^{f*}_{\vk}a^f_{\vk}}\overline{a^{*i}_{\vk}a^i_{\vk}}k_x^2\text{.}
\end{align}
Koeficienty $a^f_{\vk}$,$a^i_{\vk}$ získame riešením \eqref{eq:03SchrDis} stacionárnou poruchovou metódou do 1 rádu. Pre $\Phi_i(\vr)$ dostávame
\begin{align}
\Phi_i(\vr)=\sum_{\vk}\frac{V_{\vk\vk\ '}}{\epsilon_{\vk_i}-\epsilon_{\vk_f}}\phi_{\vk}(\vr)\text{,}
\end{align}
čo sú presne koeficienty $a^i_{\vk}$, preto dostávame
\begin{align}
\label{eq:03coef}
\overline{a^{*i}_{\vk}a^i_{\vk}}=\frac{|V_{\vk\vk_i}|^2}{(\epsilon_{\vk_i}-\epsilon_{\vk})^2}\text{.}
\end{align}

Vráťme sa teraz k BKR. 
\begin{align}
\dot{\vk}\nabla_{\vk}f(\vk)=<\frac{\partial f}{\partial t}>_{coll}=\frac{-f(\vk)-f_0(\vk)}{\tau_{\vk}}
\end{align}
kde $f_0(\vk)$ je Fermi-Diracova distribúcia a $\tau_{\vk}$ je relaxačný čas. Použitím Fermiho Zlatého Pravidla dostávame
\begin{align}
\label{eq:03goldenrule}
\frac{1}{\tau_{\vk_i}}=\sum_{\vk}W_{\vk_i\vk}(1-\frac{k_x}{k_{ix}}) = \sum_{\vk}\frac{2\pi}{\hbar}|\bra{\vk}V_{dis}(\vr)\ket{\vk_i}|^2\delta(\epsilon_{\k_i}-\epsilon_{\vk})(1-\frac{k_x}{k_{ix}}) \text{,}
\end{align}
kde $\vk_i$ je iniciálny stav. Stavy $\ket{\vk}$, $\ket{\vk_i}$ sú rovinné vlny \eqref{eq:03rov}.

Do \eqref{eq:03goldenrule} dosadíme bodový model disorderu.
\begin{align}
\label{eq:03pointdis}
V_{dis}(\vr)=\sum_{j=1}\gamma \delta(\vr-\vec R_j^{imp})\text{,}
\end{align}
kde $N_{imp}$ je počet bodových porúch v kryštáli, a $\vec R^{imp}_j$ sú náhodné polohy daných bodov. Takýto model  je pre {\it slabý} disorder postačujúci. 
\eqref{eq:03pointdis} dosadíme do \eqref{eq:03goldenrule}. Pre prehľadnosť textu sa venujeme len časti $\bra{\vk}V_{dis}(\vr)\ket{\vk_i}2$
\begin{align}
\bra{\vk}V_{dis}(\vr)\ket{\vk_i}&=\sum_{j=1}^{N_{imp}}\gamma\bra{\vk}\delta(\vr-\vec R^{imp}_j)\ket{\vk_i}\\
\bra{\vk}V_{dis}(\vr)\ket{\vk_i}&=\frac{1}{\Omega^2}\sum_{j=1}^{N_{imp}}\gamma\int_{\Omega}d\vr\delta(\vr -\vec R_j^{imp})e^{i(\vk-\vk_i)}\\
\label{eq:03vdisElement}
\bra{\vk}V_{dis}(\vr)\ket{\vk_i}&=\frac{1}{\Omega^2}\sum_{j=1}^{N_{imp}}\gamma e^{i(\vk_i-\vk)R_j^{imp}} \text{,}
\end{align}
kde v poslednom riadku sme využili Fourierovu transformáciu delta funkcie. Výsledok \eqref{eq:03vdisElement} dosadíme do \eqref{eq:03goldenrule}
\begin{align}
\label{eq:03fgr_expanded}
\frac{1}{\tau_{\vk_i}} = \frac{2\pi}{\Omega^2\hbar}\gamma^2\sum_{j=1}^{N_{imp}}\sum_{j'=1}^{N_{imp}}e^{i(\vk-\vk_i)(\vec R^{imp}_j-\vec R^{imp}_{j'})}\delta(\epsilon_{\k_i}-\epsilon_{\vk})(1-\frac{k_x}{k_{ix}}) \text{.}
\end{align}
V \eqref{eq:03fgr_expanded} napíšeme osobitne sumu pre členy, kde $j=j'$. V tejto sume dostaneme v exponente nulu, teda celý výsledok je rovný jednej. Po vysumovaní takýchto členov dostanem $N_{imp}$.
\begin{align}
\label{eq:03fgr_expanded2}
\frac{1}{\tau_{\vk_i}} = \frac{2\pi}{\Omega^2\hbar}\gamma^2[N_{imp}+\sum_{j\neq j'=1}^{N_{imp}}e^{i(\vk-\vk_i)(\vec R^{imp}_j-\vec R^{imp}_{j'})}]\delta(\epsilon_{\k_i}-\epsilon_{\vk})(1-\frac{k_x}{k_{ix}}) \text{.}
\end{align}
Sumu $\sum_{j\neq j'=1}^{N_{imp}}e^{i(\vk-\vk_i)(\vec R^{imp}_j-\vec R^{imp}_{j'})}$ môžme interpretovať ako náhodnú chôdzu v komplexonom priestore. Po vysčítaní dostanem
\begin{align}
\sum_{j\neq j'=1}^{N_{imp}}e^{i(\vk-\vk_i)(\vec R^{imp}_j-\vec R^{imp}_{j'})} =N_{imp}e^{i\alpha}\text{,}
\end{align}
kde $\alpha$ je náhodná fáza. Teraz znova uvedieme výpočet pre stredný disorder, pri stredovaní dostanem
\begin{align}
\overline{N_{imp}e^{i\alpha}}=0\text{,}
\end{align}
po dosadení do \eqref{eq:03fgr_expanded2}  a vystredovaní teda dostaneme 
\begin{align}
\label{eq:03fgr_mean}
\frac{1}{\overline{\tau_{\vk_i}}}=\frac{2\pi}{\Omega^2\hbar}\gamma^2\sum_{\vk}N_{imp}\delta(\epsilon_{\vk_i}-\epsilon_{\vk})(1-\frac{k_x}{k_{ix}})\text{.}
\end{align}
Teraz vysumujeme \eqref{eq:03fgr_mean}. Delta funkcia je párna a druhý člen v zátvorke je nepárny. Ich súčin bude tiež nepárny, a teda po vysumovaní cez párny interval všetkých $\vk$ dostaneme nulu. Ostáva nám sumovať $\sum_{\vk}\delta(\epsilon_{\vk_i}-\epsilon_{\vk})$ čo je z definície hustota stavov $\rho(\epsilon_{\vk_i})$. Finálny výsledok pre relaxačný čas teda bude
\begin{align}
\label{eq:03fgr_final}
\frac{1}{\overline{\tau_{\vk_i}}}=\frac{2\pi}{\Omega\hbar}\gamma^2n_{imp}\rho(\epsilon_{\vk_i}) \text{,}
\end{align}
kde sme zaviedli pojem hustoty bodového disorderu $n_{imp}=\frac{N_{imp}}{\Omega}$. Z rovníc \eqref{eq:03fgr_final} a \eqref{eq:03coef} dostaneme 
\begin{align}
\label{eq:03divergent}
\overline{a_{\vk}^{i*}a_{\vk}^i}=\frac{1}{\pi\rho(\epsilon_{i})}\frac{\frac{\hbar}{2\tau}}{(\epsilon_{i}-\epsilon_{k})^2}\text{.}
\end{align}
Dostali sme vzťah \eqref{eq:03divergent}, ktorý je však divergentný, teda nespĺňa normalizačnú podmienku. Navrheneme teda korekciu, extra člen v menovateli $(\frac{\hbar}{2\tau})^2$, a dostaneme Lorenzián
\begin{align}
\label{eq:03lorenz}
\overline{a_{\vk}^{i*}a_{\vk}^i}=\frac{1}{\pi\rho(\epsilon_i)}\lorenz{\epsilon_i}\text{.}
\end{align}
Túto korekciu vysvetlíme neskôr, teraz ju dosadíme do \eqref{eq:03sigma2}.
\begin{align}
\sigma(\omega)=&\frac{2\pi e^2\hbar^3}{m^2\Omega}\sum_{f,i}\frac{\hbar\omega}{(\epsilon_f-\epsilon_i)}\delta(\epsilon_f-\epsilon_i-\hbar\omega)(f(\epsilon_i)-f(\epsilon_f))\\ \notag
&\sum_{\vk}\frac{1}{\pi\rho(\epsilon_f)}\lorenz{\epsilon_f}\frac{1}{\pi\rho(\epsilon_i)}\lorenz{\epsilon_i}k_x^2
\end{align}
Prejdeme od sumy cez $\vk$ k integrálu
\begin{align}
\label{eq:03sigma3}
\sigma(\omega)=&\frac{2\pi e^2\hbar^3}{m^2(2\pi)^3}\sum_{f,i}\frac{\hbar\omega}{(\epsilon_f-\epsilon_i)}\delta(\epsilon_f-\epsilon_i-\hbar\omega)(f(\epsilon_i)-f(\epsilon_f))\\ \notag
&\int_{\Omega}d\vk\frac{1}{\pi\rho(\epsilon_f)}\lorenz{\epsilon_f}\frac{1}{\pi\rho(\epsilon_i)}\lorenz{\epsilon_i}k_x^2\text{,}
\end{align}
 Kvôli prehladnosti sa ideme venvať len integrálu cez $\vk$. Prejdeme do sférických súradníc, dostaneme
\begin{align}
\int_0^{2\pi}d\phi\int_0^\pi d\theta \frac{\sin(\theta)cos^2(\theta)\sin^2(\theta)}{4\pi} \frac{\Omega}{2\pi} \int_0^\infty dk k^2 \frac{1}{\pi\rho(\epsilon_f)}\lorenz{\epsilon_f} \frac{1}{\pi\rho(\epsilon_i)}\lorenz{\epsilon_i} 
\end{align}
Výsledok prvých dvoch integrálov je $\frac{1}{3}$. V poslednom integráli vieme prejsť do energetických súradníc. Napíšeme ho nasledovne
\begin{align}
\frac{1}{3}\frac{1}{\pi^2\rho(\epsilon_f)\rho(\epsilon_i)}\int_0^\infty d\epsilon_k \rho^{\frac{1}{2}}(\epsilon_k)k(\epsilon_k)\rho^{\frac{1}{2}}(\epsilon_k)k(\epsilon_k)\lorenz{\epsilon_f}\lorenz{\epsilon_i}\text{.}
\end{align}
Teraz využijeme rovnosť $\rho(\epsilon_k)^{\frac{1}{2}}k(\epsilon_k)=\rho(\epsilon_i)^{\frac{1}{2}}k(\epsilon_i)=\rho(\epsilon_f)^{\frac{1}{2}}k(\epsilon_f)$. Integrál teda bude
\begin{align}
\label{eq:03integral}
\frac{1}{3}\frac{\rho(\epsilon_f)^{\frac{1}{2}}k(\epsilon_f)\rho(\epsilon_i)^{\frac{1}{2}}k(\epsilon_i)}{\pi\rho(\epsilon_f)\rho(\epsilon_i)}\int_0^\infty d\epsilon_k\lorenz{\epsilon_f}\lorenz{\epsilon_i}\approx\\ \notag
\frac{1}{3}\frac{\rho(\epsilon_f)^{\frac{1}{2}}k(\epsilon_f)\rho(\epsilon_i)^{\frac{1}{2}}k(\epsilon_i)(\frac{\hbar}{2\tau})^2}{\pi\rho(\epsilon_f)\rho(\epsilon_i)}\frac{4\pi(\frac{\tau}{\hbar})^3}{1+\tau^2(\frac{\epsilon_f-\epsilon_i}{2})^2}\text{.}
\end{align}
Tento približný výsledok platí za nasledovných predpokladov
\begin{align}
\label{eq:03cond1}
\frac{\epsilon_i}{\frac{\hbar}{\tau}} >> 1\\
\label{eq:03cond2}
\frac{\epsilon_f}{\frac{\hbar}{\tau}} >> 1\text{.}
\end{align}
Teraz sa vrátime k rovnici pre optickú vodivosť \eqref{eq:03sigma3} a za integrál cez $\vk$ dosadíme približný výsledok \eqref{eq:03integral}
\begin{align}
\sigma(\omega)=\frac{e^2}{4\pi^2\hbar}\frac{1}{3}\frac{\hbar^2}{m^2}\tau\hbar^2\frac{2\pi}{\hbar}4\pi\sum_{f,i}&\frac{1}{3}\frac{\rho(\epsilon_f)^{\frac{1}{2}}k(\epsilon_f)\rho(\epsilon_i)^{\frac{1}{2}}k(\epsilon_i)}{\pi\rho(\epsilon_f)\rho(\epsilon_i)}\frac{\hbar}{\epsilon_f-\epsilon_i}\delta(\epsilon_f-\epsilon_i-\hbar\omega)(f(\epsilon_i)-f(\epsilon_f))\\ \notag
&\frac{1}{1+\tau^2(\frac{\epsilon_f-\epsilon_i}{2})^2}
\text{.}
\end{align}
Znova prejdeme od sumy k integrálu cez energie pre $f$ a $i$. Jeden z integrálov vieme vykonať hneď kvôli delta funkcii.
\begin{align}
\sigma(\omega)=e^2\frac{1}{3}(\frac{\hbar^2}{m^2}\tau)\frac{1}{1+\tau^2\omega^2}\int_0^\infty d\epsilon_i\rho^{\frac{1}{2}}(\epsilon_i)k(\epsilon_i)\rho^{\frac{1}{2}}(\epsilon_i+\hbar\omega)k(\epsilon_i+\hbar\omega)(f(\epsilon_i)-f(\epsilon_i+\hbar\omega))\text{.}
\end{align}
Pri výpočte uvažujeme nulovú teplotu, Fermi-Diracove rozdelenia budú $\Theta$ funkcie, ktoré obmedzia integračné hranice. Dostávame finálny výsledok, Kubovu Formulu
\begin{align}
\label{eq:03kubo}
\sigma(\omega)=e^2\frac{1}{3}\frac{\hbar^2k_F^2}{m^2}\tau 2\rho(E_F)\frac{1}{1+\tau^2\omega^2}F(\omega)\\
F(\omega)\equiv\frac{1}{\hbar\omega}\int_{EF-\hbar\omega}^{E_F}d\epsilon_i \frac{\rho(\epsilon_i)^{\frac{1}{2}}k(\epsilon_i)\rho(\epsilon_i+\hbar\omega)^{\frac{1}{2}}k(\epsilon_i+\hbar\omega)}{\rho(E_F)k_F^2}
\end{align}
kde dvojka pred hustotou stavov je kvôli spinu. Pre $\omega=0$ platí $F(0)=1$. To znamená, že pre nulovú frekvenciu dostávame Drudeho formulu
\begin{align}
\sigma(0)=e^2\rho(E_F)D(E_F)=\sigma_{drude} \text{,}
\end{align}
kde $D(E_F)$ je difúzny koeficient
\begin{align}
D(E_F)=\frac{1}{3}v_F^2\tau
\end{align}
Odvodili sme Kubovu formulu a z nej Drudeho formulu. Naše odvodenia boli rýdzo kvantové, bez semiklasického prístupu BKR. Ostáva nám už len zdôvodniť korekciu, ktorú sme urobili pri prechode z divergentného vzťahu \eqref{eq:03divergent} na lorenzián \eqref{eq:03lorenz}.

Výsledok \eqref{eq:03divergent} sme dostali riešením \eqref{eq:03SchrDis} stacionárnou poruchovou teóriou, teraz ideme riešiť rovnaký problém nestacionárne.
\begin{align}
\label{eq:03SchrDis2}
(\frac{-\hbar^2}{2m}\laplace_{\vr} + V_{dis}(\vr))\Phi_i(\vr)=\mathcal{E}_i\Phi(\vr)  \text{.}
\end{align}
Podobne, ako sme riešili \eqref{eq:03timeschr}, do rovnice \eqref{eq:03SchrDis2} dosadíme rozvoj 
\begin{align}
\Phi_i(\vr)=\sum_{\vk}a^{\vk_i}_{\vk}(t)\phi_{\vk}(\vr)e^{\frac{\epsilon_{\vk}t}{\hbar}}\text{,}
\end{align}
po úpravách dostaneme
\begin{align}
\hbar \frac{\partial}{\partial t}a^{\vk_i}_{\vk_f}=\sum_{\vk}a^{\vk_i}_{\vk}(t)\phi_{\vk}(\vr)e^{\frac{(\epsilon_{\vk}-\epsilon_{\vk_i})t}{\hbar}}V_{fi}\text{,}
\end{align}

použijeme Bornovu aproxímáciu a výsledok je
\begin{align}
\label{eq:03coef_nonstac}
a^{\vk_i}_{\vk_f}=\frac{-V_{fi}}{(\epsilon_f-\epsilon_i)}(e^{\frac{i}{\hbar}(\epsilon_i-\epsilon_f)}-1)\text{.}
\end{align}
V tomto vzťahu spoznávame koeficient z nestacionárnej teórie.Finálny výsledok bude identícký, pretože náš problém nie je časovo závislý. Teraz zakomponujeme časovú  závislosť
\begin{align}
V_{dis}(t)=V_{dis}e^{\frac{-t}{\tau}} \text{.}
\end{align}
Do rovnice sme vložili konečné zapínanie poruchy v čase $\tau$. Pre koeficienty dostaneme
\begin{align}
\label{eq:03coef_nonstac2}
a^{\vk_i}_{\vk_f}=\frac{-V_{fi}}{(\epsilon_f-\epsilon_i)-\frac{i\hbar}{2\tau}}(e^{\frac{i}{\hbar}(\epsilon_i-\epsilon_f-\frac{\hbar)}{2\tau}}-1)\text{.}
\end{align}
z čoho po prenásobení komplexne združeným dostaneme lorenzián \eqref{eq:03lorenz}.
