\newcommand{\bra}[1]{<#1|}
\newcommand{\ket}[1]{|#1>}
\newcommand{\expval}[2]{<#2|#1|#2>}
\newcommand{\laplace}{\Delta}
\newcommand{\diag}{\text{diag}}
\newcommand{\ftk}[3]{\frac{1}{(2\pi)^3}\int d#2\ #3 e^{i(#1#2)}}
\newcommand{\ftkvec}[3]{\frac{1}{(2\pi)^3}\int d#2\ #3 e^{i#1\cdot#2}}
\newcommand{\ftr}[3]{\int d#2\ #3 e^{-i(#1\cdot#2)}}
\newcommand{\vr}{\vec{r}}
\newcommand{\vR}{\vec{R}}
\newcommand{\vq}{\vec{q}}
\newcommand{\vk}{\vec{k}}
\newcommand{\vrp}{\vec{r'}}
\newcommand{\vkp}{\vec{k'}}
\newcommand{\E}{\mathcal{E}}
\newcommand{\dos}{N}
\newcommand{\schr}{Schr\"odingerova Rovnica}
\newcommand{\lorenz}[1]{\frac{\frac{\hbar}{2\tau}}{(#1-\epsilon_{k})^2+(\frac{\hbar}{2\tau})^2}}
\newcommand{\sinc}{\mathrm{sinc}}
\newcommand{\insertgraph}[2]{
    \begin{figure}[H]
        \centering
        \includegraphics[angle=-90,origin=c,scale=0.5,trim={1cm 0 1cm 0},clip]{grafy/#1}
        \vspace{-22mm}
        \caption{#2}
        \label{fig:#1}
    \end{figure}
}

